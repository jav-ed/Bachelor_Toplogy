
\chapter{3D Adapt Results}
In the field of 3D topology optimization,
visualization of the optimization progress and results is very important
in order to understand how a set of parameter influences
the evolution of the optimization and to evaluate the result in a qualitative way. Matlab as a post processor is not suitable for this, reasons for this
statement and advantages which are offered by the chosen post processor ParaView are described along with handling instructions in the section
 \ref{section_paraview}. 
All shown results were generated with Maltab and displayed with ParaView. 
First of all, it will be shown how the Adapt
was extended from 2D to 3D. For this purpose, a 2D load case has been
defined in 3D with on element in the depth in order to be
similar to each other. To show the results,
 one 2D result, one 3D result, one overlap
 and the inverse of an overlap are shown with 
 the usage of two different thresholds . Afterwards results with more than
 one element in the depth will be presented in section
  \ref{section_3d_loadcases}.

\section{Compare 2D and 3D results }
In order to find, whether the Adapt is correctly extended to 3D some investigations were made. A reasonable assumption is, that a 3D mesh with only one element in the depth-direction (\textbf{z}) should lead
to similar results to those obtained with
a 2D mesh.
A comparison between a 2D mesh and a
 3D mesh with only one depth- element was tested 
 with three different load-cases.

\subsection{2D and 3D load-cases}

The first load-case is known as the Cantilever,
where the left
 lower edge and the left upper edge is clamped and a force is applied vertically. In figure \ref{fig_loadcase_canti_2d} a force is applied at 80\% of the horizontally length at the bottom in vertical direction. In order to be able to get similar results, the 3D load-case is defined as follow:
 the length-details remains unchanged, however, two forces with the half magnitude of the origin force magnitude are applied, see figure \ref{fig_loadcase_canti_3d}.

\begin{figure}[!h]
\begin{minipage}{0.45\textwidth}
\centering
 \def\svgwidth{\textwidth}
\input{path_Image/load_case_2d_canti.pdf_tex} 
\caption{Load-case Cantilever in 2D.}  
\label{fig_loadcase_canti_2d}
\end{minipage}
\hfill
\begin{minipage}{0.45\textwidth}
\centering
 \def\svgwidth{\textwidth}
\input{path_Image/load_case_3d_canti.pdf_tex} 
\caption{Adjusted load-case Cantilever for 3D.}  
\label{fig_loadcase_canti_3d}
\end{minipage}
\end{figure}

The second for this purpose used load-case is similar to the the first one, therefore this load-case will be defined as Cantilever\textsubscript{2}. The force is applied vertically at 50\% of the whole vertically length and at 100\% of the horizontally length, see figure \ref{fig_loadcase_canti2_2d}. In figure \ref{fig_loadcase_canti2_3d} is shown which load-case was used to approximate the 2D load-case for 3D.\\

\begin{figure}[!h]
\begin{minipage}{0.45\textwidth}
\centering
 \def\svgwidth{\textwidth}
\input{path_Image/load_case_2d_canti2.pdf_tex} 
\caption{Load-case Cantilever\textsubscript{2} in 2D.}  
\label{fig_loadcase_canti2_2d}
\end{minipage}
\hfill
\begin{minipage}{0.45\textwidth}
\centering
 \def\svgwidth{\textwidth}
\input{path_Image/load_case_3d_canti2.pdf_tex} 
\caption{Adjusted load-case Cantilever\textsubscript{2} for 3D.}  
\label{fig_loadcase_canti2_3d}
\end{minipage}
\end{figure}

The last load-case is called MBB (Messerschmitt-Bölkow-Blohm),
where a force is applied at the top center. See figure \ref{fig_loadcase_mbb_2d} for the 2D load-case and the used 3D load case in figure \ref{fig_loadcase_mbb_3d}.
\begin{figure}[!h]
\begin{minipage}{0.45\textwidth}
\centering
 \def\svgwidth{\textwidth}
\input{path_Image/load_case_2d_mbb.pdf_tex} 
\caption{Load-case MBB in 2D.}  
\label{fig_loadcase_mbb_2d}
\end{minipage}
\hfill
\begin{minipage}{0.45\textwidth}
\centering
 \def\svgwidth{\textwidth}
\input{path_Image/load_case_3d_mbb.pdf_tex} 
\caption{Adjusted load-case MBB for 3D.}  
\label{fig_loadcase_mbb_3d}
\end{minipage}
\end{figure}

\newpage
\subsection{2D and 3D comparison results}
The upcoming figures, which are presented from figure \ref{fig_2dbasis_03} to
 figure \ref{fig_3dbasis_re_03}, show the 2D Basis, the 3D Basis, an overlapping of
 the 2D and 3D Basis and also one figure with highlighted elements which
 are not common in both models.
 The 2D Basis (figure \ref{fig_2dbasis_03}) is generated with the 88
  lines of code by \cite{Andreassen.2011} and the 3D Basis (figure \ref{fig_3dbasis_03}) is
  obtained with the 169 lines of code by \cite{Liu.2014}. 
  Since it is hardly possible 
   to see the difference between the both Bases without any support of a tool, an
 overlapping figure \ref{fig_3dbasis_overl_03} and a figure, which highlights
 especially the not matching elements (figure \ref{fig_3dbasis_re_03}), which
 are coloured yellow.  
  Both Bases
  are obtained with a $volfrac = 0.2$, a mesh-resolution of $100 \times 50$
  respectively $100 \times 50 \times 1$, $r_{min} = 1.5,\; r_{b} = 1.5,\; ep = 2$, with a
   penal-exponent of $p = 3$ and the load-case Cantilever, 
   2D-(see figure
   \ref{fig_loadcase_canti_2d}) respectively 3D-load case (see figure
 \ref{fig_loadcase_canti_3d}).
  All the shown figures starting from figure \ref{fig_2dbasis_03} to figure \ref{fig_3dbasis_re_03}
  are presented with the restriction of a threshold of 0.3
 (see subsection \ref{subsection_paraview_threshold} for more details about thresholds).
It can be observed that, there is a difference between the 2D Basis
(figure \ref{fig_2dbasis_03}) and the 3D Basis (figure \ref{fig_3dbasis_03}).
The reason for that lies in the
optimization procedure or optimization history of each element
and furthermore the simulations accuracy is limited to the computers numerically precision. \\

With the same working procedure the extension from the 2D Adapt to 3D is examined. Figure \ref{fig_2dadapt_03} shows a 2D Adapt, generated with the 2D optimiser Adapt, figure \ref{fig_3dbadapt_03} shows a 3D Adapt,
with only
 one element in the third axis (\textbf{z}), the figure \ref{fig_3dadapt_overl_03}
 shows an overlapping and figure \ref{fig_3dadapt_re_03} shows the difference
  between the 2D and 3D Adapt. It can be observed, that the difference
  between the 2D Adapt and 3D Adapt is similar to the difference between the 2D Basis and 3D Basis.
  Note that, the figures \ref{fig_2dbasis_03} - 
  \ref{fig_3dadapt_re_03} are presented with a threshold of 0.3 and the figures \ref{fig_2dbasis_05} - \ref{fig_3dadapt_re_05} are
presented 
 with a threshold of 0.5.\\

The reason, why
  some results
  with a threshold of \emph{0.5} show elements, which 
  are completely alone in the space without being connected with
   other elements is the threshold. The purpose of a threshold is to prohibit the display
   of elements which do not satisfy the value of the threshold.
   A threshold of 0.5 means that every element in the matrix, which do not fulfil the
    criteria of exhibiting a density value, which is $\rho_e \geq 0.5$ is not going to
    be displayed. \\
       
     The comparison of the 2D and 3D Adapts with the other two load-cases Cantilever\textsubscript{2} and MBB exhibits a smaller difference. The least difference, with only 5 elements was observed with Cantilever\textsubscript{2} as the load-case.
   
%:::::::::::::::::2D and 3D comparison :::::::::::::::
\begin{figure}[!h]
\begin{minipage}{0.45\textwidth}
%_____2D BASIS_________
  \includegraphics[width = \textwidth]{path_Image/pngs/Aufgabe_1/Uberlappungen/2d_basi_3D_basis/2d_basis_03.png}
	\caption{Threshold: 0.3, 2D green Basis generated with 88 lines of code \protect\cite{Andreassen.2011}} 
	\label{fig_2dbasis_03}
\end{minipage}
\hfill
\begin{minipage}{0.45\textwidth}
%K____3D_Basis__________________
  \includegraphics[width = \textwidth]{path_Image/pngs/Aufgabe_1/Uberlappungen/2d_basi_3D_basis/3d_basis_03.png}
	\caption{Threshold: 0.3, 3D red Basis generated with 169 lines of code \protect\cite{Liu.2014}.} 
	\label{fig_3dbasis_03}
\end{minipage}
\end{figure}
\begin{figure}[!h]
\begin{minipage}{0.45\textwidth}
%______2D und 3D Basis Uberlappung______
  \includegraphics[width = \textwidth]{path_Image/pngs/Aufgabe_1/Uberlappungen/2d_basi_3D_basis/2d_3d_basi_ov_03.png}
    \vspace*{-10mm}
	\caption{Threshold: 0.3, overlapped green 2D and yellow 3D Basis.} 
	\label{fig_3dbasis_overl_03}
\end{minipage}
\hfill
\begin{minipage}{0.45\textwidth}
%______2D und 3D Basis RE______
  \includegraphics[width = \textwidth]{path_Image/pngs/Aufgabe_1/Uberlappungen/2d_basi_3D_basis/2d_3d_basi_re_03.png}
    \vspace*{-10mm}
	\caption{Threshold: 0.3, overlapping 2D and 3D Basis.} 
	\label{fig_3dbasis_re_03}
	\end{minipage}
\end{figure}

%
%__________________ADAPT --- 0.3________________________________________
%
%
\begin{figure}[!h]
\begin{minipage}{0.45\textwidth}
%_____2D BASIS_________
  \includegraphics[width = \textwidth]{path_Image/pngs/Aufgabe_1/Uberlappungen/2D_Adapt_2DBasis_Adapt/2d_adapt_03.png}
  \vspace*{-10mm}
	\caption{Threshold: 0.3, 2D green Adapt generated with the 2D Adapt.} 
	\label{fig_2dadapt_03}
\end{minipage}
\hfill
\begin{minipage}{0.45\textwidth}
%K____3D_Basis__________________
  \includegraphics[width = \textwidth]{path_Image/pngs/Aufgabe_1/Uberlappungen/2D_Adapt_2DBasis_Adapt/3d_adapt_03.png}
    \vspace*{-10mm}
	\caption{Threshold: 0.3, current 3D Adapt.}
	\label{fig_3dbadapt_03}
\end{minipage}
\end{figure}
\begin{figure} [H]
\begin{minipage}{0.45\textwidth}
%______2D und 3D Basis Uberlappung______
  \includegraphics[width = \textwidth]{path_Image/pngs/Aufgabe_1/Uberlappungen/2D_Adapt_2DBasis_Adapt/2d_3d_adapt_th_03_ov.png}
    \vspace*{-10mm}
	\caption{Threshold: 0.3, overlapped green 2D and yellow 3D Adapt.} 
	\label{fig_3dadapt_overl_03}
\end{minipage}
\hfill
\begin{minipage}{0.45\textwidth}
%______2D und 3D Basis RE______
  \includegraphics[width = \textwidth]{path_Image/pngs/Aufgabe_1/Uberlappungen/2D_Adapt_2DBasis_Adapt/2d_3d_adapt_th_03_re.png}
    \vspace*{-10mm}
	\caption{Threshold: 0.3, overlapping 2D and 3D Adapt.} 
	\label{fig_3dadapt_re_03}
	\end{minipage}
\end{figure}

%_______________________________BASIS 2D 3D 0.5___________________________
\begin{figure}[!h]
\begin{minipage}{0.45\textwidth}
%_____2D BASIS_________
  \includegraphics[width = \textwidth]{path_Image/pngs/Aufgabe_1/Uberlappungen/2d_basi_3D_basis/2d_basis_05.png}
      \vspace*{-10mm}
	\caption{Threshold: 0.5, 2D green Basis generated with 88 lines of code \protect\cite{Andreassen.2011}} 
	\label{fig_2dbasis_05}
\end{minipage}
\hfill
\begin{minipage}{0.45\textwidth}
%K____3D_Basis__________________
  \includegraphics[width = \textwidth]{path_Image/pngs/Aufgabe_1/Uberlappungen/2d_basi_3D_basis/3d_basis_05.png}
      \vspace*{-10mm}
	\caption{Threshold: 0.5, 3D red Basis generated with 169 lines of code \protect\cite{Liu.2014}.} 
	\label{fig_3dbasis_05}
\end{minipage}
\end{figure}
\begin{figure}[!h]
\begin{minipage}{0.45\textwidth}
%______2D und 3D Basis Uberlappung______
  \includegraphics[width = \textwidth]{path_Image/pngs/Aufgabe_1/Uberlappungen/2d_basi_3D_basis/2d_3d_basi_ov_05.png}
      \vspace*{-10mm}
	\caption{Threshold: 0.5, overlapped green 2D and yellow 3D Basis.} 
	\label{fig_3dbasis_overl_05}
\end{minipage}
\hfill
\begin{minipage}{0.45\textwidth}
%______2D und 3D Basis RE______
  \includegraphics[width = \textwidth]{path_Image/pngs/Aufgabe_1/Uberlappungen/2d_basi_3D_basis/2d_3d_basi_re_05.png}
      \vspace*{-10mm}
	\caption{Threshold: 0.5, overlapping 2D and 3D Basis.} 
	\label{fig_3dbasis_re_05}
	\end{minipage}
\end{figure}

%
%__________________ADAPT --- 0.5________________________________________
%
%
\begin{figure}[H]
\begin{minipage}{0.45\textwidth}
%_____2D BASIS_________
  \includegraphics[width = \textwidth]{path_Image/pngs/Aufgabe_1/Uberlappungen/2D_Adapt_2DBasis_Adapt/2d_adapt_05.png}
      \vspace*{-10mm}
	\caption{Threshold: 0.5, 2D green Adapt generated with the 2D Adapt.} 
	\label{fig_2dadapt_05}
\end{minipage}
\hfill
\begin{minipage}{0.45\textwidth}
%K____3D_Basis__________________
  \includegraphics[width = \textwidth]{path_Image/pngs/Aufgabe_1/Uberlappungen/2D_Adapt_2DBasis_Adapt/3d_adapt_05.png}
	\caption{Threshold: 0.5, current 3D Adapt.}
	\label{fig_3dbadapt_05}
\end{minipage}
\end{figure}
\begin{figure} [!h]
\begin{minipage}{0.45\textwidth}
%______2D und 3D Basis Uberlappung______
  \includegraphics[width = \textwidth]{path_Image/pngs/Aufgabe_1/Uberlappungen/2D_Adapt_2DBasis_Adapt/2d_3d_adapt_th_05_ov.png}
      \vspace*{-10mm}
	\caption{Threshold: 0.5, overlapped green 2D and yellow 3D Adapt.} 
	\label{fig_3dadapt_overl_05}
\end{minipage}
\hfill
\begin{minipage}{0.45\textwidth}
%______2D und 3D Basis RE______
  \includegraphics[width = \textwidth]{path_Image/pngs/Aufgabe_1/Uberlappungen/2D_Adapt_2DBasis_Adapt/2d_3d_adapt_th_05_re.png}
      \vspace*{-10mm}
	\caption{Threshold: 0.5, overlapping 2D and 3D Adapt.} 
	\label{fig_3dadapt_re_05}
	\end{minipage}
\end{figure}

%____________3D results with more than one depth-elemt________________-

\section{3D Adapt results with more than one elements in depth}
\label{section_3d_loadcases}

Before presenting the main 3D Adapt results,
some new 3D load cases are introduced.
 Figure \ref{fig_load_case_clamped_cant} 
 shows the load-case Cantilever\textsubscript{3}. The difference between the Cantilever and
 Cantilever\textsubscript{3} is in the boundary 
conditions. The Cantilever\textsubscript{3}'s left side is 
 clamped fully, whereas the Cantilever has only 4 corner  
 nodes, which are clamped. The second difference 
 is that, the applied force is not multiplied by the 
 factor $\frac{1}{2}$, each node in the depth (\textbf{z}),
  is stressed with the same force \textbf{$F = 1$}. The 
 topology will not change by increasing the 
 magnitude of $F$, however, the compliance will increase.
 The second 3D load-case is a modified version 
 (see figure \ref{fig_load_case_MBB_modified}) of the 
 MBB load-case (figure \ref{fig_loadcase_mbb_3d}).
 Each node in the depth (z) is clamped, which means
 that the DOFs are blocked in all three axes. Furthermore
 the force equals $F = 1$, instead of $F = \frac{1}{2}$ and is applied on
 each node in the depth. The 
 load-case Cantilever\textsubscript{2} 
 (figure \ref{fig_loadcase_canti2_3d}) is also 
 deployed with some modifications for this section. 
 The boundary conditions for this lode-cases remains 
 unchanged, however the applied magnitude of the 
 forces are not going to be reduced by half and also 
 the force is applied on each 
 node in the depth (\textbf{z}). Therefore the 
 modified MBB, Cantilever\textsubscript{2} and
  Cantilever\textsubscript{3}  will be 
 defined as depth load-cases.\\
 
 The 3D Adapt results are shown 
in figure \ref{fig_load_canti_02} - \ref{fig_load_mitchell_04_t} 
 with two different thresholds,
 where the left figure is presented with a displaying threshold of
 \emph{0.2} and the right figure has a threshold of
 \emph{0.4}. The Basis is gray coloured and the
 added AMs are coloured with the colormap 
 \emph{Blue to Red Rainbow} with
 blue representing the lowest densitiy respectively
 red the highest. The table \ref{tabe_3d_load_cases} 
 offers more informations about
 the figures \ref{fig_load_canti_02} - \ref{fig_load_mitchell_04_t}.
 The compliance \emph{c} of the Basis and of
 the Adapt from table \ref{tabe_3d_load_cases}
 are not completely correct for their presented threshold.
 The reason for this is, if a threshold is
  defined, then all elements that do not fulfil this threshold
   are considered as non-existent elements (no material) and 
   all elements with a density above this threshold are set to \emph{1}.
    This means that several elements are completely lost, but 
    also that the density of many elements is artificially 
    increased to \emph{1}. 
    Increasing the density to 1 is necessary because
     in practice there are no intermediate densities. 
     There is either material or there is none. In order 
     to be able to give an exact c-value, it would be
      essential to load each Adapt result with
       the desired threshold into e.g. Optistruct and 
       perform a recalculation. However, in order to provide a comparison value, 2 
       Adapt results with different thresholds were 
       loaded and the variation to the \emph{c}-values given in 
       the table
       \ref{tabe_3d_load_cases}
        is about 2\%. To emphasize it again,
        the values in the table
        \ref{tabe_3d_load_cases} 
        are not entirely 
        correct for their thresholds and therefore no guarantee is given on 
        the accuracy of the \emph{c}-values.

 
 
% _______________Load-Case______________________
 \begin{figure} [!h]
\begin{minipage}{0.45\textwidth}
 \centering
 \def\svgwidth{\textwidth}
 \input{path_Image/load_case_3d_canti_voll.pdf_tex}
 \caption{3D load-case: Cantilever\textsubscript{3}.}    % Bildunterschrift 
 \label{fig_load_case_clamped_cant}          % Label für Verweise f
\end{minipage}
\hfill
\begin{minipage}{0.45\textwidth}
 \centering
 \def\svgwidth{\textwidth}
 \input{path_Image/load_case_3d_mbb_2.pdf_tex}
 \vspace*{6mm}
 \caption{3D load-case: MBB modified.}    % Bildunterschrift 
 \label{fig_load_case_MBB_modified}          % Label für Verweise f
\end{minipage}
\end{figure} ~\\

% __________________::::::::::::::::::::::The 3D Results_________________________-
\begin{figure}[!h]
\begin{minipage}{0.45\textwidth}
%________________CANTIlever______________________
\centering
  \includegraphics[width=  \textwidth]{path_Image/pngs/Aufgabe_1/3D_Ergebnisse/canti.png}
	\caption{Cantilever\textsubscript{3}, threshold 0.2, $100\times 50 \times 8$.}  
	\label{fig_load_canti_02}
\end{minipage}
\hfill
\begin{minipage}{0.45\textwidth}
\centering
  \includegraphics[width= \textwidth]{path_Image/pngs/Aufgabe_1/3D_Ergebnisse/canti_04.png}
	\caption{Cantilever\textsubscript{3}, threshold 0.4, $100\times 50 \times 8$.} 
	\label{fig_load_canti_04}
\end{minipage}\\

\vspace{0.75cm}
%____________________Cantilever_2________________
\begin{minipage}{0.45\textwidth}
\centering
  \includegraphics[width=  \textwidth]{path_Image/pngs/Aufgabe_1/3D_Ergebnisse/mitchell.png}
	\caption{Cantilever\textsubscript{2}, threshold 0.2, $100\times 50 \times 8$.} 
	\label{fig_load_mitchell_02}
\end{minipage}
\hfill
\begin{minipage}{0.45\textwidth}
\centering
  \includegraphics[width= \textwidth]{path_Image/pngs/Aufgabe_1/3D_Ergebnisse/mitchell_04.png}
	\caption{Cantilever\textsubscript{2}, threshold 0.4, $100\times 50 \times 8$.} 
	\label{fig_load_mitchell_04}
\end{minipage}\\

\vspace{0.75cm}
%______________________MBB_________________
\begin{minipage}{0.45\textwidth}
\centering
  \includegraphics[width=  \textwidth]{path_Image/pngs/Aufgabe_1/3D_Ergebnisse/mbb.png}
	\caption{MBB modified, threshold 0.2, $100\times 50 \times 8$.} 
	\label{fig_load_mbb_02}
\end{minipage}
\hfill
\begin{minipage}{0.45\textwidth}
\centering
  \includegraphics[width= \textwidth]{path_Image/pngs/Aufgabe_1/3D_Ergebnisse/mbb_04.png}
	\caption{MBB modified, threshold 0.4, $100\times 50 \times 8$.} 
	\label{fig_load_mbb_04}
\end{minipage}
\end{figure}

\begin{figure}[!h]
%_______________________Canti1 -Tiefen________________
\begin{minipage}{0.45\textwidth}
\centering
  \includegraphics[width=  \textwidth]{path_Image/pngs/Aufgabe_1/3D_Ergebnisse/canti_t.png}
	\caption{Cantilever\textsubscript{3}, threshold 0.2, $100\times 50 \times 14$.} 
	\label{fig_load_canti_02_t}
\end{minipage}
\hfill
\begin{minipage}{0.45\textwidth}
\centering
  \includegraphics[width= \textwidth]{path_Image/pngs/Aufgabe_1/3D_Ergebnisse/canti_t_04.png}
	\caption{Cantilever\textsubscript{3}, threshold 0.4, $100\times 50 \times 14$.} 
	\label{fig_load_canti_04_t}
\end{minipage}\\

\vspace{0.75cm}
%%______________________CANTI 2_Tiefen__________________--
\begin{minipage}{0.45\textwidth}
\centering
  \includegraphics[width=  \textwidth]{path_Image/pngs/Aufgabe_1/3D_Ergebnisse/mitchell_t.png}
	\caption{Cantilever\textsubscript{2}, threshold 0.2, $100\times 50 \times 14$.} 
	\label{fig_load_mitchell_02_t}
\end{minipage}
\hfill
\begin{minipage}{0.45\textwidth}
\centering
  \includegraphics[width= \textwidth]{path_Image/pngs/Aufgabe_1/3D_Ergebnisse/mitchell_t_04.png}
	\caption{Cantilever\textsubscript{2}, threshold 0.4, $100\times 50 \times 14$.}
\label{fig_load_mitchell_04_t}
\end{minipage}
\end{figure}
%%::::::::::::::::::::.3D REsults-Table::::::::::::::::::::::::::::::::::
With the table from the table \ref{tabe_3d_load_cases} it
 can be observed that each \emph{c}-Basis is higher than 
 \emph{c}-Adapt. This proves that with the usage of the
Adapt the compliance is reduced, or the stiffness
is increased. Furthermore it can be seen that the
improvement regarding stiffness is highly depended on the
load-case, e.g. in case of the load-case Cantilever\textsubscript{3} the
stiffness is increased by the factor \emph{1.57}, whereas the
the load-case Cantilever\textsubscript{2} exhibits an improvement
of the factor \emph{1.19}.\\

\begin{table}[!h]
\begin{tabular}{|c|c|c|c|c|c|c|c|}
\hline 
\rowcolor{Green}
Figure & Mesh resolution & TA & \emph{c} Basis &
\emph{c} Adapt& NI & sec/iteration \\ 
\hline 
\ref{fig_load_canti_02} &$100\times 50 \times 8$&0.2& 448.9 &286.34 &  167& 34 \\ 
\hline 
\ref{fig_load_canti_04} &$100\times 50 \times 8$&0.4& 448.9&  286.34 & 167 & 34 \\ 
\hline 
\ref{fig_load_mitchell_02}  &$100\times 50 \times 8$& 0.2&874.28 &  737.775 & 250 & 42 \\ 
\hline 
\ref{fig_load_mitchell_04} &$100\times 50 \times 8$& 0.4 &874.28 &  737.775  & 250 & 42 \\ 
\hline 
\ref{fig_load_mbb_02} &$100\times 50 \times 8$& 0.2&72.15& 59.5 & 146 & 32 \\ 
\hline 
\ref{fig_load_mbb_04} &$100\times 50 \times 8$&0.4&72.15 &  59.5 & 146 & 32 \\ 
\hline 
\ref{fig_load_canti_02_t}  &$100\times 50 \times 14$&0.2& 2699.62&1750.9 & 250 & 36 \\ 
\hline 
\ref{fig_load_canti_04_t} &$100\times 50 \times 14$&0.4&2699.62& 1750.9 & 250 & 36 \\ 
\hline 
\ref{fig_load_mitchell_02_t} &$100\times 50 \times 14$& 0.2 & 7930.95 & 7084.74 & 193 & 38 \\ 
\hline 
\ref{fig_load_mitchell_04_t} &$100\times 50 \times 14$& 0.4&7930.95& 7084.74 & 193 & 38 \\ 
\hline 
\end{tabular} 
\caption{Shows the compliance \emph{c} with the usage of the optimiser Adapt and without (Basis).\\
TA = Threshold Adapt
NI = Number of iterations.\\
\emph{c} = compliance\\
Parameters of the figures \protect\ref{fig_load_canti_02} to \protect\ref{fig_load_mitchell_04_t}\\
Basis-volfrac = 0.2, Adapt-volfrac = 0.3, ep = 2, r\textsubscript{min} = r\textsubscript{b} = 1.5.
}
\label{tabe_3d_load_cases}
\end{table}

%______________________________________________________________________________________________________
%::::::::::::::::::::::3D Results -- Parameter::::::::::::::::::::::::::::::::::::::::::::::::
\section{Parameter investigation}

This section will investigate how changing Adapt specific parameters
 affects the compliance and display the Adapt results.
  For this,
the figure \ref{fig_comp_orig} is defined as the
origin Adapt result, which serves for comparison.
The load case is named as the Cantilever\textsubscript{3} (see figure
\ref{fig_load_case_clamped_cant}), the mesh resolution is 
defined as $100 \times 50 \times 8$, $ep = 2$, $r_{min} = r_b = 1.5$, the \textit{volfrac}
 of the Basis is \emph{0.2} and the \textit{volfrac} of the Adapt is
 \emph{0.3}. 
 However, the
 Re-filter radius r\textsubscript{min} and the BZ-filter 
 radius r\textsubscript{b} have always the same value. First 
 the influence of the parameter \emph{ep} is shown. Afterwards with
  constant \emph{ep}-value the filter radii are changed. The results
  are shown in figures \ref{fig_comp_ep_3} - \ref{fig_comp_re_3},
  all with a display threshold of \emph{0.2}.
  A summary of these results can be obtained from the
  table \ref{tabel_compare_parameters}. \\
  
  The figures \ref{fig_comp_orig} - \ref{fig_comp_ep_5} exhibit an increasing \emph{ep}-value.
  It can be observed that with the 
  increase of \emph{ep} the distance of the \emph{AMs}
  and the Basis is increased as expected. This effect can be seen in
  the left upper area of the figures 
  \ref{fig_comp_orig} - \ref{fig_comp_ep_5}.
  The reason for this occurrence is that, \emph{ep}
  defines the thickness of the BZ, in which the elements
  are penalized with the higher exponent of \emph{3}. This
  means only the elements with a high sensitivity can remain in the
  BZ. In case of the actual Basis and the Adapt, the Basis load-case
  and the Adapt load-case are equal, which
  signifies that the sensitivity around the Basis is 
  higher than elsewhere. With increasing distance to the
  Basis the sensitivity decreases. However, in conclusion it
  can be stated that, because
  of the penalization exponent of 
  \emph{3} only few elements with a high
  sensitivity can remain and 
most of the elements are penalized to void. Furthermore, in case of 
  a too high 
 penal exponent, it may be that, the AMs cannot
  connect to the Basis. Therefore the penalization exponent
  of \emph{3} is recommended for the BZ.\\
  
   The figures
  \ref{fig_comp_re_2} - \ref{fig_comp_re_3} exhibits
  a variation of the sensitivity filter radii wit a common vlue of
  \emph{ep}. The filter defines the
  minimal member size of the AMs. With an increasing 
  filter radius the minimal member size is also increased, which can
  be seen well, by comparing the figures \ref{fig_comp_re_2} and
  \ref{fig_comp_re_3}. Narrow bars in the lower middle area
  are removed and the already existing bars are thickened.
  Another comprehension that can be obtained by
comparing the figures \ref{fig_comp_re_2} - \ref{fig_comp_re_3}
  is that many small struts have a positive impact 
  for high stiffness then few wide struts. This can be stated, since
  the compliance (see table \ref{tabel_compare_parameters}) 
  increases with an increase of the filter radii.\\
 
%_____________Parameter-INvestigation-Table__________________----  
\begin{table}[!h]
\centering
\begin{tabular}{|c|c|c|c|c|c|c|c|}
\hline 
Figure & ep & r\textsubscript{min} = r\textsubscript{b} &
\emph{c} &NI & sec/iteration  \\ 
\hline 
\ref{fig_comp_orig} & 2 & 1.5  & 286.34 & 168 & 33 \\ 
\hline 
\ref{fig_comp_ep_3} & 3 & 1.5 & 284.6 & 198 & 40 \\ 
\hline 
\ref{fig_comp_ep_4} & 4 & 1.5  & 291.78 & 195 & 45 \\ 
\hline 
\ref{fig_comp_ep_5} & 5 & 1.5  & 293.62 & 215 & 48 \\ 
\hline 
\ref{fig_comp_re_2} & 2 & 2  & 237.95 & 102 & 35 \\ 
\hline 
\ref{fig_comp_re_2_5} & 2 & 2.5  & 298.57 & 128 & 41 \\ 
\hline 
\ref{fig_comp_re_3} & 2 & 3  & 313.8 & 74 & 36 \\ 
\hline 
\end{tabular} 
\caption{Shows the impact of the parameter \emph{ep, r\textsubscript{min}
and r\textsubscript{b}}
on the compliance.\\
NI = Number of Iterations.}
\label{tabel_compare_parameters}
\end{table}

\begin{figure}[!h]
\centering
  \includegraphics[width = \textwidth]{path_Image/pngs/Aufgabe_1/Parameter/orig.png}
	\caption{$ep = 2\; r_{min} = r_{b}=1.5$.} 
	\label{fig_comp_orig}
\end{figure}

%________________________________________________________________
\begin{figure}[!h]
\begin{minipage}{0.45\textwidth}
%EPS
\centering
  \includegraphics[width = \textwidth]{path_Image/pngs/Aufgabe_1/Parameter/re15_ep3.png}
	\caption{$ep = 3\; r_{min} = r_{b}=1.5$.} 
	\label{fig_comp_ep_3}
\end{minipage}
\hfill
\begin{minipage}{0.45\textwidth}
\centering
  \includegraphics[width = \textwidth]{path_Image/pngs/Aufgabe_1/Parameter/re15_ep4.png}
	\caption{$ep = 4\; r_{min} = r_{b}=1.5$.} 
	\label{fig_comp_ep_4}
	\end{minipage}
	\vspace{1cm}
	\begin{minipage}{0.45\textwidth}
	\centering
  \includegraphics[width = \textwidth]{path_Image/pngs/Aufgabe_1/Parameter/re15_ep5.png}
	\caption{$ep = 5\; r_{min} = r_{b}=1.5$.} 
	\label{fig_comp_ep_5}
\end{minipage}
\hfill
%_______________________RES__________________________________
\begin{minipage}{0.45\textwidth}
\centering
  \includegraphics[width = \textwidth]{path_Image/pngs/Aufgabe_1/Parameter/re2_ep2.png}
	\caption{$ep = 2\; r_{min} = r_{b}=2$.} 
	\label{fig_comp_re_2}
	\end{minipage}\\
	
	\begin{minipage}{0.45\textwidth}
	\centering
  \includegraphics[width = \textwidth]{path_Image/pngs/Aufgabe_1/Parameter/re25_ep2.png}
	\caption{$ep = 2\; r_{min} = r_{b}=2.5$.} 
	\label{fig_comp_re_2_5}
	\end{minipage}
	\hfill
	\begin{minipage}{0.45\textwidth}
	\centering
  \includegraphics[width = \textwidth]{path_Image/pngs/Aufgabe_1/Parameter/re3_ep2.png}
	\caption{$ep = 2\; r_{min} = r_{b}=3$.} 
	\label{fig_comp_re_3}
\end{minipage}\\
\end{figure}~\\




