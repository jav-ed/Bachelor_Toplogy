\chapter{Finite element reanalysis}
\label{chapter_fea_reanalysis}
In order to find out whether the 3D Adapt delivers
 meaningful reinforcements, a FE calculation is carried out
 in this section with 
a commercial software in order to obtain
access to further output values,
such as stress distribution. The selected commercial  
software is \emph{Hyperworks OptiStruct}.
However, the structures, which are added by the Adapt, the Adapt Members (AMs),
are to be understood as meaningful extension, if the AMs
lead to an increase of the stiffness of the entire structure. 
 
\section{Finite element analysis with OptiStruct}
For applying a FE reanalysis with OptiStruct 
 a function called \emph{OSSmooth} is used. 
 It takes two input parameter,
 one is a 
 \emph{.fem}-file and the other is a \emph{.sh}-file.
These files are required in order to load a topology
optimized structure from Matlab into \emph{Hyperworks} by
using \emph{OSSmooth}.
 The \emph{.fem}-file stores information 
 about the discretization e.g. mesh size and type of element to discretize.
 The \emph{.sh}-file is composed of
 the obtained density
 values from the optimization,
 which are between \emph{0} and \emph{1}.
 Each finite element from the \emph{.fem}-file obtains its belonging  density from the \emph{.sh}-file. 
  Furthermore the OSSmooth function offers a selectable OSSmooth threshold
  in order to select the elements which will be considered
  for the interpretation and smoothing og the geometry.\\
 
 Through some investigations it was
 determines that not all value of OSSmooth threshold lead to reasonable results.
 In case of a 
 high OSSmooth threshold, e.g. 0.9, most of the elements 
 are not included in the smoothing process. 
  This leads to elements, which have no connection with other elements, an example can be
  obtained by the figure \ref{fig_2dadapt_05}. 
  These isolated elements lead to an increase in compliance and also distort the grid.
  A OSSmooth threshold $\leq 0.25$ was found to be
  a value, which delivered interpreted geometries
  close to the results obtained from the optimization.\\
  
    As mentioned, the\emph{ .sh}- and\emph{ .fem}-files are required in
     order to make use of the OSSmooth and
  to generate a \emph{.fem}-file. It is necessary to be familiar with
  \emph{OptiStructs} node numbering convention,
  which can be obtained through the figure
   \ref{fig_optistruct_node_1}.
   To make sure that each element in the \emph{.fem} file is assigned the correct density, the command from Listing \ref{lst_sh_data} for Matlab
   is used, where \textit{density\_matrix} can be a 2D or 3D Matlab default matrix which contains the density values. 
 
 \begin{figure} [!h]
 \centering
 \def\svgwidth{0.70\textwidth}
 \input{path_Image/3d_nodes_opti.pdf_tex}
 \caption{OptiStruct node numbering convention.}    % Bildunterschrift 
 \label{fig_optistruct_node_1}          % Label für Verweise 
\end{figure} 
%______________matlab 3d BZ_____________
\begin{lstlisting}[
style=Matlab-editor,
basicstyle=\mlttfamily,
escapechar=`,
label=lst_sh_data,
caption={Helpful command for generating a .sh-file.}
]
sh_densities = (density_matrix(:))';
\end{lstlisting} 

\section{Reanalysis of MBB}
In this section an example for a reanalysis is given. The chosen Adapt for this purpose is generated with modified MBB version, see figure
\ref{fig_load_case_MBB_modified}.
This \textit{ MBB Adapt} is loaded into OptiStruct using OSSmooth and a 
OSSmooth threshold of \emph{0.5} is used.
 However note that a threshold of 0.5
 for most other Adapt results, is going to be too high. A OSSmooth 
 threshold greater than \emph{0.25} will most likely cause a distorted grid.\\

 The element size was selected to 0.5, the feature angle to 30,
  the force equals 1000, the discretization was 
  selected to \textit{mixed}, and the options \textit{connection detect} and \textit{iso surface}
  were chosen. The results can be obtained
  from the figures
 \ref{fig_mbb_re_1} and \ref{fig_mbb_re_2}.
One requirement that the Adapt has to meet is that it should not 
show any local stress peaks. Continuous stress curves, which should be in the middle range of the stress-range, are
 desired. In principle, the
 higher the stress in the Adapt Members (AMs), the greater their
 influence on increasing stiffness, or deceasing the compliance. A high
 stress in the AMs would ensure that
 placement of the AMs increases the stiffness of the whole
 structure. It can be observed that the Basis 
 (gray coloured in figure \ref{fig_edge_modi_mbb_05}) 
is exposed to a lower stress in most places than the AMs. The reason for
this occurrence is that, the added AMs take over a part of the total load. Therefore the Basis has to take up a lower load. Furthermore, it can be 
seen that the stress distribution 
does not show any erratic behaviour, the transitions
 from low to high stress are clear and the the AMs 
 are mostly not stressed to the red
 maximum stress level (see figures \ref{fig_mbb_re_1} and \ref{fig_mbb_re_2}).
  Because the stress in the AMs is clearly visible, the
  Adapt performed a useful material distribution in order to support
  the Basis.

\begin{figure}[!h]
 \centering
 \begin{minipage}{\textwidth}
  \centering
    \includegraphics[width= 0.8 \textwidth]{path_Image/pngs/Aufgabe_2/alles.png}
 	\caption{FEA renalysis of MBB Adapt.} 
 	\label{fig_mbb_re_1}
 \end{minipage}\\
 
 \vspace{0.75cm}
  \begin{minipage}{\textwidth}
 \centering
   \includegraphics[width= 0.6 \textwidth]{path_Image/pngs/Aufgabe_2/Schnitt.png}
 	\caption{FEA renalysis of MBB Adapt, cut view.} 
 	\label{fig_mbb_re_2}
 \end{minipage}

 \end{figure}

