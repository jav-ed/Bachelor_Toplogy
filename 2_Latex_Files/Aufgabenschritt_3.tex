%Explain, why the current 2d adpt does not work for our purpouses
%Explain what we want to have
\chapter{Extension of the 2D Adapt for slightly different loads between Basis and Adapt}
\label{chapter_features_adapt}
 The Adapt was created to strengthen already existing structures, which are called Basis. 
 The Basis must not be changed, only new structures are allowed to be added to ensure  that the Basis gains stiffness.
 The connections of the added structures may 
 only be created at the border of the Basis (in the BZ).  
 Desired are small connections between Basis and the AMs.
  Up to now, the Basis and the AMs are obtained with the same load case.
  This means that the Basis was obtained with a load-case, as shown in
  figure \ref{fig_equal_load} and the Adapt was then
  obtained with the same load case, which only differs in the higher
  requirement as shown in figure
  \ref{fig_equal_load_adapt}.
  If the load case for the Adapt is slightly changed,
  few and large-area connections are created between Basis
   and AMs, which are unwanted and can be
   seen in figure 
   \ref{fig_adapt_large_Connec}.
   The following section introduces and describes further restrictions and parameters 
   that provide more control over the Adapt in
    order to obtain small connections between the Basis and the AMs.
    These restrictions are going to be called \emph{features} and are going to
    be presented in the upcoming sections. The first feature
    (section \ref{section_feature_1})
    enables the control of the tangential thickness of
    a connection between Basis and the AMs.
    The second feature enables to define
    the minimal distance between each found
    connection
    (see section \ref{section_dealing_with_overlap})
    and the third feature enables it
    to remove local reinforcements
    (see section \ref{section_feature_3}).
    Why these features are
    needed and how they work is explained in their
    respective section. After having discussed the features
    a table with the brief explanations of
    the new Adapt specific terms is provided in
    the section
    \ref{section_fundamentals}.
    Finally in
    section \ref{section_fea_features} one
     Adapt result with these features is
    used for a FE reanalysis.\\
    
     In order to explain, what slightly
     different load-case means,
    the figure \ref{fig_slighlty_different_load} 
    gives an overview about the definitions of slightly different load-case,
    where the origin Basis load-case
    can be seen in figure \ref{fig_equal_load}.
    A slightly different Adapt load-case results in an additional force or
    slightly shift of the original load and a new Adapt load-case can
    be obtained from figure \ref{fig_new_load}.

 
\begin{figure}[!h]
\begin{minipage}{0.45\textwidth}
\centering
  \includegraphics[width= \textwidth]{path_Image/pngs/Aufgabe_3/grbegri_1.png}
	\caption{Resulting from optimization.} 
	\label{fig_equal_load}
\end{minipage}
\hfill
\begin{minipage}{0.45\textwidth}
\centering
  \includegraphics[width= \textwidth]
  {path_Image/pngs/Aufgabe_3/grbegri_2.png}
	\caption{Same loadcase, higher requirements.} 
	\label{fig_equal_load_adapt}
\end{minipage}\\

\vspace{0.75cm}
\begin{minipage}{0.45\textwidth}
\centering
  \includegraphics[width= \textwidth]{path_Image/pngs/Aufgabe_3/grbegri_3.png}
	\caption{Slightly different load.} 
	\label{fig_slighlty_different_load}
\end{minipage}
\hfill
\begin{minipage}{0.45\textwidth}
\centering
  \includegraphics[width= \textwidth]{path_Image/pngs/Aufgabe_3/grbegri_4.png}
	\caption{New load.} 
	\label{fig_new_load}
\end{minipage}\\

\vspace{0.75cm}
\centering
\begin{minipage}{0.7\textwidth}
   \includegraphics[width= \textwidth]
   {path_Image/pngs/Aufgabe_3/last_pos_79_ohne.png}
 	\caption {Adapt with slightly different load-case.} 
 	\label{fig_adapt_large_Connec}
 	\end{minipage}
\end{figure}

    
\section{Feature 1: control of tangential length connection}
\label{section_feature_1}

    The goal of the Adapt is to generate small connections
    between the Basis and the Adapt while increasing
    the stiffness of the whole structure. To create thin
    connections, it is necessary to control both the normal
     and tangential width of the
    connection. The figure \ref{fig_ep_cz_tangen} shows a gray coloured Basis
    and the yellow zone represents elements of a generated
    connection. Furthermore, it  
    can be observed that the parameter \textit{ep} defines the 
     normal length and the parameter \textit{CZ\textsubscript{mind}}
      controls the tangential length of the connection.
      \newpage
 
 \begin{figure} [!h]
 \begin{minipage}{0.45 \textwidth}
The idea to define a connections tangential thickness
is shown in the figure 
\ref{fig_penal_domain}.
 Within  \textit{CZ\textsubscript{mind}}, which
 is the parameter that defines the
 maximal thickness of a connection,
 the sensitivity of the elements
 will not be reduced or the elements will not be penalized.
  However, every element between \textit{CZ\textsubscript{mind}} and
    \textit{CZ\textsubscript{max}} will be punished. Note that only
    elements
     whose centre lies within \textit{CZ\textsubscript{mind}} will be penalized (compare with figure \ref{fig_rmin}).
     This ensures that no connection is larger than \textit{CZ\textsubscript{mind}}.
     In order to be able to apply this process, it is first necessary to find out where the individual connections arise. Each connection consist of elements and
      because the Adapt consist of multiple connections, the 
      term \emph{Connection Zone Group} is introduced (CZG).
      A CZG can only contain elements, which is
      located in the BZ, because the BZ is the zone in which a
      possible connection can be made. However a CZG only contains
      those elements from the BZ, which lead to a connection.
 \end{minipage}
 \hfill
  \begin{minipage}{0.45 \textwidth}
  \centering
 \def\svgwidth{\textwidth}
 \input{path_Image/cz_ep_tangential.pdf_tex}
 \caption{Gray Basis, yellow CZG elements, $ep = 3, CZ_{mind} = 4$.}    % Bildunterschrift 
 \label{fig_ep_cz_tangen}          % Label für Verweise 
 \end{minipage}
\end{figure} 
%
 \begin{figure} [!h]
 \begin{minipage}{0.45 \textwidth}
  The
      sum of all the CZGs is going to be called
      the Connection Zone (CZ), which also has
      to be located within the BZ. \\
      
             After two technical terms have 
      been explained, it shall be mentioned
      that the CZGs are extracted after 10 iterations,
      in order to give enough time for
      the CZGs to arise, before identifying them and
      manipulating their
      properties.
      With that being said,
      the procedure of finding CZGs shall be introduced.
            Since the CZ lies within the BZ, it is only required to consider the BZ elements during the search.
       Furthermore a \textit{CZ-threshold} is employed: if the
        considered BZ element fulfils the \textit{CZ-threshold} it
         is taken as a CZ-element.
         Next it must be checked, if the considered
          element is a neighbour of an element that is already in a CZG.
            In case it is the first element, which was found, a new CZG
            for himmust be generated and in every 
 \end{minipage}
 \hfill
  \begin{minipage}{0.45 \textwidth}
 \centering
 \def\svgwidth{\textwidth}
 \input{path_Image/cz_mind_max.pdf_tex}
 \caption{Orange CZ\textsubscript{max}, purple CZ\textsubscript{mind} and
 gray penal area \emph{AVA}.}    % Bildunterschrift 
 \label{fig_penal_domain}          % Label für Verweise 
 \end{minipage}
\end{figure}
 other case, the elements must be examined.
 This examination
      is 
      as already mentioned
      based on neighbourhood: it verifies whether the current element is a 
      neighbour element of the already found elements, which are stored in
      CZGs.
       By taking advantage of the numbering convention
       (square and regular mesh)
        of the elements, which is from left to right and top to bottom, only 
        maximal 4 search directions need to be taken into 
       account. The search directions can be obtained in figure \ref{fig_look_around},
       where the gray
       element can be taken as the current element in the BZ
       and the green arrows as the 4 search directions-neighbour-elements.
       Each of the neighbour element is obligated to be 
       within the BZ and  meet the \textit{CZ-threshold}.
       
       The reason why only maximal 
       4 search directions need to be taken into account,
       is that two for loops are used, the first one changes the x-number 
       of the element
       and the second loop changes the y-number of the considered element
       in the matrix. 
       Because the x-loop comes first and the y-loop comes second, the column 
       stays constant and after all rows with the constant column
       were examined the column will change to $column_{new} = column +1$.
       Thus, the elements without arrows inside
       (figure \ref{fig_look_around})
        have not been checked ye and rhus can not be
        n any CZG and therefore 
       there is no need for a neighbourhood-examination. In case the
       current element was assigned to a group with the
       first direction of the neighbourhood search, the remaining
       search directions are not going to be considered any more
       in order to save computational time. Neighbour groups will
       be merged subsequently. 
            \begin{figure}[!h]
 \centering
 \def\svgwidth{0.45\textwidth}
 \input{path_Image/look_around.pdf_tex}
 \caption{Search directions for the CZG examination }    % Bildunterschrift 
 \label{fig_look_around}          % Label für Verweise 
      \end{figure}

       \subsection{Geometrical and weighted centre}
       \label{subsection_geometrical_weighted}
       After it is possible to save the CZGs, the centre of the CZG needs to be 
       located
       in order to define the maximal tangential thickness 
       CZ\textsubscript{mind}.
       The centre of a 
       CZG is the point, where the two radii
       CZ\textsubscript{mind} and CZ\textsubscript{max} are
       generated in order to penalize all the elements
       within CZ\textsubscript{max} and outside CZ\textsubscript{}mind
       ( \emph{AVA}, see figure \ref{fig_penal_domain}).
       There are two possible ways, which are going to be described.\\
       
       The first method in order to calculate the centre of
       each CZG can be expressed as 
       the geometrical centre. To calculate the geometrical centre, two mean 
       values are required. Each CZG consists of elements, which
       have x and y-coordinates and with this coordinates the x-mean
       and y-mean can be obtained.
       \begin{align*}
       x_{mean} = \dfrac{1}{N} \sum_{i=1}^N x_i, \quad 
       y_{mean} = \dfrac{1}{N} \sum_{i=1}^N y_i,
       \end{align*}
       where \textbf{$N$} is the number of the elements which are stored
       in the considered CZG\textsubscript{i}, 
       \textbf{$x_i$} and \textbf{$y_i$} are the x- and 
       y-coordinates of the CZG elements.\\
       
       The second method is a weighted average method and can be expressed
       as follows:
          \begin{align*}
       x_{mean} = \dfrac{1}{\sum_{i=1}^N \rho_i} \sum_{i=1}^N x_i  \rho_i, \quad 
       y_{mean} = \dfrac{1}{\sum_{i=1}^N \rho_i} \sum_{i=1}^N y_i \rho_i,
       \end{align*}
		where \textbf{$N$} is the number of the elements which are stored
       in the considered CZG\textsubscript{i},
        \textbf{$x_i$} and \textbf{$y_i$} are the x- and 
       y-coordinates of the CZG elements and \textbf{$\rho_i$} are their
       densities.
       The advantage of the weighted average method is, that the elements 
       densities are considered in order to calculate the mean x- and y-coordinates.
       Therefore the weighted average method allows to come
       closer to the centre of the elements with a higher density. The
       elements with a higher density have a bigger impact on increasing
       the stiffness then the low densities. Since the
       maximal tangential thickness is defined with CZ\textsubscript{mind},
       and the elements outside of CZ\textsubscript{mind} and
       inside CZ\textsubscript{max} are penalized,
       the closer CZ\textsubscript{mind} gets to the centre of the
       densities with higher values the better the impact
       on increasing the stiffness becomes. 
      
       \subsection{Penalization of elements within\textit{ CZ\textsubscript{mind}} and \textit{CZ\textsubscript{max}}}
       \label{subsection_distance_i}
       
       After having found the CZGs and their centre ($x_{mean}$ and $y_{mean}$)
       all elements which
       lie between \textit{CZ\textsubscript{mind}} and
        \textit{CZ\textsubscript{max}}
        or within AVA
       must be penalized (see figure \ref{fig_penal_domain}. The
       reason why there is no difference between penalizing
       between \textit{CZ\textsubscript{mind}} and
        \textit{CZ\textsubscript{max}} and AVA is based
        on the definition of
        $CZ_{max} = AVA + CZ_{mind}$, where
        CZ\textsubscript{mind} and AVA are
        parameter which are defined by the operator of
        the Adapt and CZ\textsubscript{max} is calculated
        then by a summation. In case of $AVA = 0$,
        $CZ_{max} = CZ_{mind}$ and then the penalization area
        would not exist (see figure
        \ref{fig_penal_domain}).\\

       However, to reduce the sensitivities
       within AVA, the
       penalization task needs to be 
       accomplished after the sensitivity analysis and
       it must be performed 
       for each CZG with its respective x- and y-mean
        coordinates. In order to penalize the AVA 
        elements
		the distance ($dist_i$) between each element in
		 CZG and the centre$(x_{mean}|y_{mean})$ of the CZG needs
		to be $CZ_{mind} < dist_i \leq CZ_{max}$.
		
		\begin{align*}
		dist_i = \sqrt{(x_{mean_i}-x_i)^2+(y_{mean_i}-y_i)^2} \quad,
		\end{align*}
where \textbf{$x_{mean_{i}}$} and\textbf{ $y_{mean_{i}}$} are the centre coordinates 
in x- respectively
y-direction of the current CZG\textsubscript{i},  \textbf{$x_i$ }
and \textbf{$y_i$} are the coordinates in x- and y-direction of the
 current CZG\textsubscript{i} element.\\
 
 To penalize the AVA elements $(CZ_{mind} < dist_i \leq CZ_{max})$ the penalization function from
\cite{Dienemann.2018} is deployed.

\begin{align*}
P_i(d_i) = \dfrac{1- \tanh(a \; [2 \; d_i / CZ_{mind} -1)}{1+ \tanh(a)} \quad,
\end{align*}

where \textbf{$d_i = \dfrac{CZ_{mind}}{2}$}
and a defines the discreetness of the function. A low discreetness leads to
insufficient penalization. A value of $a = 4$ will be used
as recommended in \cite{Dienemann.2018}.\\


Analogously to finding elements between \textit{CZ\textsubscript{mind} }
and \textit{CZ\textsubscript{max}}, elements 
within \textit{CZ\textsubscript{mind}} can be
located with $dist_i \leq  CZ_{mind}$ with
a if statement that checks, whether the distance $d_i < CZ_{mind}$.
This can be useful, when it is desired to increase the sensitivity of
the mentioned elements.
However, there was no need to implement the former yet.

\section{Feature 2: Definition of minimal distance between Connection Zone Groups (CZGs)}
\label{section_dealing_with_overlap}
In this section it is going to be explained how to control
the minimal distance between all CZGs. The idea is it to
find overlapping CZGs and in case of an overlapping, the
overlapping CZGs are combined to one CZG. In
case the CZGs do not overlap, the distance between the
CZGs is greater then the minimal distance between 
CZGs parameter.\\

It can be observed that some overlapping CZGs can
cause unintentional penalization.
In general a distinction between two types of 
overlapping can be made. To provide a good understanding, the figure \ref{fig_CZG_overlapping_1}
  shows two CZGs, which do not overlap. 
  The figure \ref{fig_CZG_overlapping_2} shows the first type
  of an overlapping of two CZGs. The first case results in penalizing the elements twice, once by the
left CZG and once by the right CZG within AVA.
\vspace{0.2cm}
 \begin{figure} [!h]
 \def\svgwidth{\textwidth}
 \input{path_Image/overlapping_1.pdf_tex}
 \caption{Overlapping of CZGs type 1.}    % Bildunterschrift 
 \label{fig_CZG_overlapping_1}          % Label für Verweise 
\end{figure} 
%
 \begin{figure} [!h]
 \centering
 \def\svgwidth{\textwidth}
 \input{path_Image/overlapping_2.pdf_tex}
 \caption{Two CZGs with no overlapping.}    % Bildunterschrift 
 \label{fig_CZG_overlapping_2}          % Label für Verweise 
\end{figure} 
 \begin{figure} [!h]
 \centering
 \def\svgwidth{\textwidth}
 \input{path_Image/overlapping_3.pdf_tex}
 \caption{Overlapping of CZGs type 2 and calcualting new centre of two overlapped CZGs.}    % Bildunterschrift 
 \label{fig_CZG_overlapping_3}          % Label für Verweise 
\end{figure} 
However, since no \textit{CZ\textsubscript{mind}} element is punished,
 the maximum tangential thickness of the CZGs
is not changed. The AVA elements of each CZG is 
penalized, in case of a complete identical AVA
as shown in figure \ref{fig_CZG_overlapping_2}.
The complete AVA
is the whole overlapping area, which is then penalized twice
and in case only some parts of the AVAs overlaps 
only these overlapping parts are penalized twice,
the rest of the AVAs is penalized once.
The second type of overlapping CZG occurs, when the \textit{CZ\textsubscript{max}} of at least two CZGs overlaps,
which would penalize elements within \textit{CZ\textsubscript{mind}} (see figure \ref{fig_CZG_overlapping_3}). Penalizing \textit{CZ\textsubscript{mind}} elements affects
the maximum tangential thickness of the CZG. In this way the parameter \textit{CZ\textsubscript{mind}} is not respected
any more and does not describe the maximum tangential thickness of a CZG. In order to avoid
such an occurrence, the distance ($dist_{cz}$) of all CZGs is calculated. If
$dist_i < 2 \; CZ_{mind} +AVA $
at least two \textit{CZ\textsubscript{max}} are overlapping,
which results in penalizing \textit{CZ\textsubscript{mind}} elements. Therefore if
$dist_i < 2 \; CZ_{mind} +AVA$,
new $x_{mean}$ and $y_{mean}$ of the overlapping CZGs are calculated.
Because of the mentioned $dist_i$-condition
it is possible to define a minimal distance
between CZGs. The defining parameter is 
  This \emph{AVA}, which lets define
  \emph{AVA} not only the penalization area,
  but also the minimal distance between CZGs.
  The calculation of the centre
can be accomplished by means of the geometrical or weighted centre,
 introduced in subsection \ref{subsection_geometrical_weighted}.
Again, because of the already 
named benefits the calculation is suggested with the weighted centre.\\

Due to its simple comprehension, the figure 
\ref{fig_CZG_overlapping_3} shows a geometrical centre calculation of a new
CZG\textsubscript{mind\textsubscript{1\_2}} and a
 new CZG\textsubscript{max\textsubscript{1\_2}} of two overlapped CZGs. The new CZG\textsubscript{1\_2} is replaced in the
working storage by the two overlapped CZGs,
CZG1 and CZG2 in order to save memory space,
keep a better overview of the CZGs and it also facilitates the later required modifications on the
BZ-filter, which will be explained in section 
\ref{section_cgz_bz_filter}.
 The new CZG\textsubscript{1\_2} contains all the informations
of the two overlapped CZGS, CZG\textsubscript{1}
 and CZG\textsubscript{2}, with the difference that it is one CZG and its centre
of gravity consists of the coordinates of the overlapping CZGs. \\

The figure \ref{fig_matlab_1_CZG} shows a struct from
Matlab, which stores all the found CZGs with
 the names of the CZGs (\textit{Gruppe\_\textsubscript{i}}),
  the x-, y-coordinates and 
 the densities of their elements as a vector and the center 
of gravity in x and y-direction of each found CZG as \emph{Doubles}
named \textit{X\_mean}
and \textit{Y\_mean}. These informations are sufficient in order to
collect all essential information about the elements
in a CZG\textsubscript{i}. The figure \ref{fig_matlab_2_CZG} shows
the same struct after finding overlapping CZGs: in this case
three overlapping CZGS are found. Therefore CZGS, CZG\textsubscript{4},
CZG\textsubscript{5} and CZG\textsubscript{6} were replaced by the 
CZG\textsubscript{4\_5\_6}.

\begin{figure}[!h]
\begin{minipage}{0.45\textwidth}
\centering
  \includegraphics[width= \textwidth]
  {path_Image/pngs/Aufgabe_3/matlab_1.png}
	\caption{List of found CZGs.} 
	\label{fig_matlab_1_CZG}
\end{minipage}
\hfill
\begin{minipage}{0.45\textwidth}
\centering
  \includegraphics[width= \textwidth]
  {path_Image/pngs/Aufgabe_3/matlab_2_overlapped.png}
	\caption{List of found CZGs after overlapping examination.} 
	\label{fig_matlab_2_CZG}
\end{minipage}
\end{figure} 
%
In order to get a better overview of
the progress Matlabs plot function
in 2D is used.
This allows to project 
\textit{CZ\textsubscript{mind}} and 
\textit{CZ\textsubscript{max}} on the Adapt at the centre of gravity of
 CZGs, which
is shown in figure \ref{fig_circles_CZG}. The dark gray coloured structure
represents the Basis, the lighter gray color stands for the BZ
($\rho_e < 0.1$), the Adapt without
the Basis is rainbow coloured, where dark blue stands for
void material.

\begin{figure}[!h]
\centering
  \includegraphics[width = 0.75\textwidth]
  {path_Image/pngs/Aufgabe_3/czgs_bsp_1.png}
	\caption{Circles for \textit{CZ\textsubscript{mind}} and 
	\textit{CZ\textsubscript{max}}.} 
	\label{fig_circles_CZG}
\end{figure}

\section{ Feature 3: Prohibition of local reinforcements}
\label{section_feature_3}
The Adapt is coded in order to create connections
between the Basis and the AMs.
 As a further restriction, the Adapt will be given to apply material 
 within the BZ, only if it contributes to a \emph{used} connection.
 A \emph{used} connection does not cause local
 reinforcements and is given when the Basis is connected to
 the REs through the BZ, wherein the REs have a
 struts like developed AMs structure.
  The aim is to allow material to be applied to the BZ,
  if there 
  is a  struts like developed AM structure in the vicinity of the found CZG.
  The reason for the extension of this feature is:
  in practise it would be an additional effort to
  manufacture local reinforcements.
  In case there is no formed AM structure, this would only result
  into thickening of the Basis. The term \emph{unused connection}
  is going to be used as local reinforcement.\\
  
  In order to find an unused connection the centre of the
  CZGs is required. This serves as the starting point of the examination,
  because it makes investigations within \textit{AVA} easier.
  Within \textit{AVA} of the CZGs each element must
  be tested, whether it belongs to the REs and in case it does, 
 the density of the matching element needs to be compared
 with a user predefined RE-threshold. If the densities of the matched 
 elements are lower then the RE-threshold, then all the elements inside
 the CZG or the CZG itself needs to be penalized in order
 to prohibit its development. Because \textit{AVA} is a small area, it is 
 sufficient to find one single element, which does not fulfil
 the RE-threshold in order to penalize the CZG.
To make certain that the penalization is 
sufficient, the penalization factor is held variable. With some
if statement it is possible to implement a method, which
can perform the penalization after a number of iterations and
change the beginning magnitude of the penalization repeatedly.
Such if statements are beneficial when the Adapt
faces
unused connections in which elements exhibits high sensitivities.
Therefore high and variable increasable penalisation factors
are required in order to remove unused connections.
The beginning penalisation
of the sensitivity is set to 0.5, which is
performed after having found CZGs. As mentioned the search for
CZGs is performed from the 10th iteration and after each 5 iterations
the RE penalization factor is increased by the factor \emph{0.5} or 
the penalization is increased by the factor 2
(decreasing the 
sensitivity results in a increasing punishment).\\

The figures \ref{fig_nr_iteration_13} to \ref{fig_nr_iteration_16} shows
a topology optimization by the Adapt employing the explained 
method in order to prohibit the occurrence of unused CZGs. 
The small circles represents \textit{CZ\textsubscript{mind}}, the
big circles are \textit{CZ\textsubscript{max}}.
The figures shows that the unused connections 
or the CZGS are removed
in iteration 16 (figure \ref{fig_nr_iteration_16}).
 
\begin{figure}[!h]
\begin{minipage}{0.48\textwidth}
\centering
  \includegraphics[width= \textwidth]{path_Image/pngs/Aufgabe_3/13.png}
	\caption{Iteration number 13.} 
	\label{fig_nr_iteration_13}
\end{minipage}
\hfill
\begin{minipage}{0.48\textwidth}
\centering
  \includegraphics[width= \textwidth]{path_Image/pngs/Aufgabe_3/14.png}
	\caption{Iteration number 14.} 
	\label{fig_nr_iteration_14}
\end{minipage}\\

\vspace{0.3cm}
\begin{minipage}{0.48\textwidth}
\centering
  \includegraphics[width= \textwidth]{path_Image/pngs/Aufgabe_3/14.png}
	\caption{Iteration number 15.} 
	\label{fig_nr_iteration_15}
\end{minipage}
\hfill
\begin{minipage}{0.48\textwidth}
\centering
  \includegraphics[width= \textwidth]{path_Image/pngs/Aufgabe_3/16.png}
	\caption{Iteration number 16.} 
	\label{fig_nr_iteration_16}
\end{minipage}
\end{figure}

\section{Modifications on the BZ-filter}
\label{section_cgz_bz_filter}
Since all the CZGs are located in the BZ, the BZ-filter needs to be modified.
The important parameter for the filtering is 
\textit{CZ\textsubscript{mind}}: it defines the maximal
tangential thickness of the CZGs. As long as $CZ_{mind} \geq ep$ 
the rest of BZ elements, which are within AVA or even further away
 can be neglected. It is also essential not to filter the 
 CZGs which consist a unused connection. This can be performed
 by removing these CZGs or explicitly not including them in
 the filtering process. Furthermore a CZG can exhibit elements, which
 are located within \textit{CZ\textsubscript{mind}}, however,
 are not stored in the CZG. The reason is that, these missing elements
 do not fulfil the CZ-threshold.\\
 To avoid a checkerboard,
    each element within \textit{CZ\textsubscript{mind}} is filtered.
   To find these elements, the distance between the centre of gravity
    of the CZG\textsubscript{i} and its elements is required. 
    The calculation already has been explained in 
    subsection \ref{subsection_distance_i}.
   
   
\section{Comparison results with and without features}
\label{section_compare_results_with_without_feature}

This section shows results with 2D adapts
 with and without the implemented features.
 In particular, it is examined what the Adapt
  performs when its load differs from that of the Basis. 
  For this purpose two versions were presented, 
  the Adapt version with the 
  features and the Adapt version without the features. 
 In order to compare the Adapt versions
  all parameters, such as \textit{ep, r\textsubscript{min}, r\textsubscript{b}},
 the  mesh resolution and the load-case, 
  are equal for the examination with and without the 
  features. However, it is important to understand,
  that the load cases are only equally selected 
  for the two versions of the Adapts, not 
  for the Basis and the Adapt, since the main
  object of this section is to see, what
  the Adapt performs in case its load
  case is not equal with the
  Basis load case. However, the Basis is equal
  for all the Adapt results.\\
  
 The figures on the left show the results without
   the integration of the features and the figures on the
    right show the results with the integration of the features.
    The Basis is gray, the blue elements
    represents void material and the AMs are
    rainbow coloured.\\
    
Figure \ref{fig_comp_fea_l10_o} and \ref{fig_comp_fea_l10_m}: 
 the Basis and the Adapts are obtained with the 
same load in y direction (see figure \ref{fig_loadcase_canti_2d}),
 in x direction, where x represents the
 \emph{horizontal} direction and y stands for
 the \emph{vertical} direction. 
 The Adapts, however, receive a different load. 
The load in x direction is defined as +10\% (tensile load) of the x load,
which results in an increase of the resulting load.
The left figure \ref{fig_comp_fea_l10_o}  shows the result
 without the features and the right figure 
 \ref{fig_comp_fea_l10_m} with the features.\\
 
The sole difference 
between figures \ref{fig_comp_fea_l10_o} and 
\ref{fig_comp_fea_l10_m} to
the figures \ref{fig_comp_fea_ln10_o} and \ref{fig_comp_fea_ln10_m} 
is that the load in the figures
\ref{fig_comp_fea_ln10_o} and
\ref{fig_comp_fea_ln10_m}
 is selected in negative x direction.
 The figures \ref{fig_comp_fea_l50_o} and
\ref{fig_comp_fea_l50_m} are obtained with nearly the same
conditions as in the introductory load case
(figures \ref{fig_comp_fea_l10_o} 
and \ref{fig_comp_fea_l10_m},
 however, the only
difference is that the load in x direction is defined as 50\% of the y load
(tensile load).
 The figures \ref{fig_comp_fea_ln50_o} and
\ref{fig_comp_fea_ln50_m} show the latter Adapts (50\%)
with a negative x-direction. \\

The figures \ref{fig_comp_fea_l79_o} to \ref{fig_comp_fea_l81_m}
 present the results for slightly changing the 
 load position, but with the same load magnitude. The results from figure
 \ref{fig_comp_fea_l79_o} and \ref{fig_comp_fea_l79_m} are obtained 
 with the a load position of $0.79 L$ instead of $0.8 L$,
 which is the load position for the Basis
  (see figure \ref{fig_loadcase_canti_2d}) and the
 figures \ref{fig_comp_fea_l81_o} and
 \ref{fig_comp_fea_l81_m} are obtained with
 a load position of $0.81 L$. This change of the position
 is performed in x-direction.\\
 
It can be observed that the features fulfil their purpose.
The keeping of the maximum tangential
connection thickness can be seen especially well
when comparing the figures \ref{fig_comp_fea_l10_o}
and \ref{fig_comp_fea_l10_m} .
 At the upper left area, can be seen how
 the area of the connection from the figure
 \ref{fig_comp_fea_l10_o}
decreases in figure
\ref{fig_comp_fea_l10_m}.
The work of the third feature,
removing local reinforcements can be
seen when comparing, the two figures
\ref{fig_comp_fea_ln10_o} and
\ref{fig_comp_fea_ln10_m}.
The upper left local reinforcement in
figure \ref{fig_comp_fea_ln10_o}
disappears in figure \ref{fig_comp_fea_ln10_m}, because of the
usage of the features. The same observations are made for the
other examples\\

\vspace{0.75cm}

 \begin{figure}[!h]
 \begin{minipage}{0.45\textwidth}
 \centering
   \includegraphics[width= \textwidth]
   {path_Image/pngs/Aufgabe_3/last_10_ohne.png}
 	\caption{Without features, +10\%.} 
 	\label{fig_comp_fea_l10_o}
 \end{minipage}
 \hfill
 \begin{minipage}{0.45\textwidth}
 \centering
   \includegraphics[width= \textwidth]
   {path_Image/pngs/Aufgabe_3/last_10_mit.png}
 	\caption{With features, +10\%.} 
 	\label{fig_comp_fea_l10_m}
 \end{minipage}\\
 
 \vspace{0.75 cm}
  \begin{minipage}{0.45\textwidth}
 \centering
   \includegraphics[width= \textwidth]
   {path_Image/pngs/Aufgabe_3/last_n10_ohne.png}
 	\caption{Without features, -10\%.} 
 	\label{fig_comp_fea_ln10_o}
 \end{minipage}
 \hfill
 \begin{minipage}{0.45\textwidth}
 \centering
   \includegraphics[width= \textwidth]
   {path_Image/pngs/Aufgabe_3/last_n10_mit.png}
 	\caption{With features, -10\%.} 
 	\label{fig_comp_fea_ln10_m}
 \end{minipage}
 \end{figure}
 \begin{figure}[!h]
 \vspace{0.75 cm}
  \begin{minipage}{0.45\textwidth}
 \centering
   \includegraphics[width= \textwidth]
   {path_Image/pngs/Aufgabe_3/last_50_ohne.png}
 	\caption{Without features, +50\%.} 
 	\label{fig_comp_fea_l50_o}
 \end{minipage}
 \hfill
 \begin{minipage}{0.45\textwidth}
 \centering
   \includegraphics[width= \textwidth]
   {path_Image/pngs/Aufgabe_3/last_50_mit.png}
 	\caption{With features, +50\%.} 
 	\label{fig_comp_fea_l50_m}
 \end{minipage}\\

  \vspace{0.75 cm}
    \begin{minipage}{0.45\textwidth}
 \centering
   \includegraphics[width= \textwidth]
   {path_Image/pngs/Aufgabe_3/last_n50_ohne.png}
 	\caption{Without features, -50\%.} 
 	\label{fig_comp_fea_ln50_o}
 \end{minipage}
 \hfill
 \begin{minipage}{0.45\textwidth}
 \centering
   \includegraphics[width= \textwidth]
   {path_Image/pngs/Aufgabe_3/last_n50_mit.png}
 	\caption{With features, -50\%.} 
 	\label{fig_comp_fea_ln50_m}
 \end{minipage}\\
  \end{figure}~\\
  
  
  \begin{figure} [!h]
  \begin{minipage}{0.45\textwidth}
 \centering
   \includegraphics[width= \textwidth]
   {path_Image/pngs/Aufgabe_3/last_pos_79_ohne.png}
 	\caption {Without features.} 
 	\label{fig_comp_fea_l79_o}
 \end{minipage}
 \hfill
 \begin{minipage}{0.45\textwidth}
 \centering
   \includegraphics[width= \textwidth]
   {path_Image/pngs/Aufgabe_3/last_pos_79_mit.png}
 	\caption{With features.} 
 	\label{fig_comp_fea_l79_m}
 \end{minipage}\\
 
   \vspace{0.75 cm}
  \begin{minipage}{0.45\textwidth}
 \centering
   \includegraphics[width= \textwidth]
   {path_Image/pngs/Aufgabe_3/last_pos_81_ohne.png}
 	\caption{Without features.} 
 	\label{fig_comp_fea_l81_o}
 \end{minipage}
 \hfill
 \begin{minipage}{0.45\textwidth}
 \centering
   \includegraphics[width= \textwidth]
   {path_Image/pngs/Aufgabe_3/last_pos_81_mit.png}
 	\caption{With features.} 
 	\label{fig_comp_fea_l81_m}
 \end{minipage}
 \end{figure}

 
 \section{Fundamental terms}
 \label{section_fundamentals}
Since the previous presented 
three new
features required some new technical words, the table \ref{table_third_task}
shall serve as a short and fast Information source. 
%_____________________TABELLE 1 ________________________________________
\begingroup
\renewcommand{\arraystretch}{2} % Default value: 1
 \begin{longtable}{L{0.2\textwidth} L{0.8\textwidth}}
 \caption{Brief explanations about often used terms in this chapter.}\\
  \hline 
  \rowcolor{Green}
 \multicolumn{1}{c}{Term}  &  \multicolumn{1}{c}{Explanation}  \\ 
 \hline
   \rowcolor{Gray1}  
 Connection-Zone (CZ) & The CZ is the domain in which the Adapt generates
 connections to the Basis. It is located within the BZ and 
 represents the sum of all the generated connections.
  In order to be regarded as a CZ-element, it must must 
  fulfil the CZ-threshold. Each CZ-element is also a BZ-element, but
  not vice versa.
  Note, 
 BZ elements can have a density smaller than the CZ-threshold \\ 
\hline
 CZ-threshold &  The CZ-threshold is a density-threshold
 parameter, which defines, whether the
 elements in the BZ can be considered as CZ elements or not.\\
 \hline 
    \rowcolor{Gray1}  
 RE-threshold & The RE-threshold is a density-threshold
 parameter, which defines, whether 
 the REs exhibit a local reinforcement, see section
 \ref{section_feature_3}.\\
 \hline 
 Connection-Zone-Group (CZG) & The Adapt generates multiple
 connections in order to connect the AMs to the Basis. In order to
 be able to modify single connections, CZGs are introduced. 
 A CZG is one single connection, which consists of all the elements
 which are in the CZ and
are required to build one connection between
 the Basis and the AMs.  The yellow elements or the yellow zone in figure 
 \ref{fig_ep_cz_tangen} represents a CZG.
 The sum of
 all the CZGs is the CZ.\\

 \hline
     \rowcolor{Gray1}  
CZ\textsubscript{mind} and CZ\textsubscript{max} & \textit{CZ\textsubscript{mind}}
defines the maximal tangential thickness of the connection between AMs and Basis. $CZ_{max} = AVA + CZ_{mind}$ (see figure \ref{fig_penal_domain})
is required in order to penalize all the 
elements between \textit{CZ\textsubscript{max}} and 
\textit{CZ\textsubscript{mind}} or
within \textit{AVA}.
With the two parameter \textit{CZ\textsubscript{max}} and \textit{CZ\textsubscript{mind}} it is possible to control the tangential thickness of 
the connection between AMs and Basis, where
CZ\textsubscript{mind} and AVA are user provided parameter and
CZ\textsubscript{max} is calculated with $CZ_{max} = AVA + CZ_{mind}$ .

It is important to choose $CZ_{mind} \geq ep$, otherwise
BZ elements which
are normal to the Basis and  
 form the connection between
Basis, BZ and AMs, are penalized unintentionally.

The figure \ref{fig_ep_geq_ep} shows a gray coloured Basis,
yellow  elements, which are normal to the Basis and
lies between the Basis and the red AM elements. 
$CZ_{mind} < ep$ and the dark green  coloured area presents the 
penalization area. In case $CZ_{mind} < ep$ a connection
between Basis, BZ elements and AM elements might
be disturbed with void elements, because of the
penalization. Therefore the following must
chosen: $CZ_{mind} \geq ep$.\\

     \rowcolor{Gray1}  
 CZ\textsubscript{mind} and CZ\textsubscript{max} & Without \textit{CZ\textsubscript{max}}  the elements around 
\textit{CZ\textsubscript{mind}} can not be penalized (gray colored in
figure \ref{fig_penal_domain}),
which results in none control over the tangential connection thickness.

Note: the normal thickness or length of the
a CZG is defined by the parameter \textit{ep}. 
\\
 \hline 
AVA & \textit{AVA} (see figure \ref{fig_penal_domain})
 represents the penalization area, within \textit{AVA}, elements
from CZGs will be penalized in order to achieve demanded maximal
tangential CZ thickness \textit{CZ\textsubscript{mind}}. It
is a user predefined parameter and since 
the $dist_i $-condition from 
section \ref{section_dealing_with_overlap}, is defined as 
$dist_i < 2 \; CZ_{mind} +AVA $
\textit{AVA} controls the minimal distance between all
overlapped CZGs.
The effect of \textit{AVA} can be seen in figure 
\ref{fig_CZG_overlapping_2} .  \\
\hline
\label{table_third_task}
\end{longtable}
\endgroup 

\begin{figure}[!h]
  \centering
 \def\svgwidth{0.45\textwidth}
 \input{path_Image/cz_geq_ep.pdf_tex}
 \caption{ $ CZ_{mind} < ep$.}    % Bildunterschrift 
 \label{fig_ep_geq_ep}          % Label für Verweise 
\end{figure}

%_______________
 
  
 \section{Finite element reanalysis of the Adapt with features}
 \label{section_fea_features}
 
The reasons why a FE reanalysis can be considered as meaningful 
were already mentioned in chapter \ref{chapter_fea_reanalysis}
and will therefore not be discussed again. Section
\ref{section_compare_results_with_without_feature} offers
6 different load cases, however, one Adapt with features
will be used for the FE reanalysis. Since the figures
\ref{fig_comp_fea_l79_o} and \ref{fig_comp_fea_l79_m} exhibit
the strongest influence of the implemented features, these
Adapt results will
be used for
the FE reanalysis.\\

 The Adapt is restricted to three features.
These features help to increase the manufacturability
 of the Adapt structures in practice.
 The restrictions can have a positive as 
 well as a negative effect on the compliance.
 In case of the requirement of a maximum tangential
  thickness of the CZGs, a connection larger than
   \textit{CZ\textsubscript{mind }} is usually broken down into several 
   smaller connections, whereby it must meet the
    minimum distance of \textit{AVA}. The reason why 
    one thick connection is broken into several smaller
     ones instead of applying the obtained material 
     elsewhere is that the sensitivity in the vicinity of 
     the connection location may be higher than in another location.
     Furthermore, this requirement does not prohibit 
     the appearance of any structure completely, it is limited in 
     its tangential thickness by  \textit{CZ\textsubscript{mind }} and
      in its normal thickness by \textit{ep}. 
      Thus it is possible that several smaller 
      formed compounds increase stiffness. \\
       
       In the case of the requirement that no unused 
       connection should be created or that the 
       base should not only be thickened, the 
       creation of a local reinforcement is completely
        forbidden. Since these structures, due to 
        their high sensitivities, have required high
         punishment factors to be removed, it is 
         very likely that the obtained material
          cannot be applied in a better place . 
          Therefore, the application of this feature
           is expected to result in a loss of stiffness.\\
                        
       Both figure \ref{fig_comp_fea_rean_alt} and 
       \ref{fig_comp_fea_rean_neu} do not  exhibit any obvious
       differences in the maximum strain (red areas).
        Both also  display no sharp transitions from high to low 
        stress and vice versa. Furthermore, the stress
         curves are in the middle range of the
         stress range, which 
         is  desirable.
          Through the two figures it  can be 
          observed that both AMs, with and without the features,
           are exposed to stress, 
           which leads to the conclusion that they 
           reduce the compliance of the Adapt structure.\\
           
  \begin{figure} [!h]
 \centering
   \includegraphics[width= \textwidth]
   {path_Image/pngs/Aufgabe_3/079_nachrech_alt.png}
 	\caption {FE reanalysis of a Adapt without features.} 
 	\label{fig_comp_fea_rean_alt}
 	 \end{figure}
 
  \begin{figure} [!h]
 \centering
   \includegraphics[width= \textwidth]
   {path_Image/pngs/Aufgabe_3/079_nachrech_neu.png}
 	\caption{FE reanalysis of a Adapt with features.} 
 	\label{fig_comp_fea_rean_neu}
 \end{figure}
 
% _______________________________________




