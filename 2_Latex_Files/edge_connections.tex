\chapter{Prohibit connections at edges and corners}
\label{chapter_prohibit_edge}

The 3D Adapt exhibits results with corner- and edge-connections in the BZ (see figure \ref{fig_edge_edge_connec}
and \ref{fig_edge_corner_connec}), which
can be considered as a \emph{3D checker-board}.
Since such connections are not possible to manufacture in reality, they need to be prohibited. 

\begin{figure}[!h]
\begin{minipage}{0.45\textwidth}
\centering
 \includegraphics[width = \textwidth]{path_Image/pngs/Aufgabe_1/edges/edges.png}
	\caption{Edge connection in 3D, gray Basis.} 
	\label{fig_edge_edge_connec}
\end{minipage}
\hfill
\begin{minipage}{0.45\textwidth}
\centering
 \includegraphics[width = \textwidth]{path_Image/pngs/Aufgabe_1/edges/corner.png}
	\caption{Corner connections in 3D, gray Basis.} 
	\label{fig_edge_corner_connec}
\end{minipage}
\end{figure}
 
 To prevent the appearance of edge- and corner-connected elements,
 the search of the BZ needs to be modified. The 2D discretized
 element exhibits 4 Nodes and 4 Edges, this results in 4 possible 
 edge-connections 
 and 4 possible node-connections. Since the Adapt is 
 not programmed for letting the Basis directly
 connected to REs, the BZ needs a reasonable definition.
 In case $ep = 1$, the distance between the border
 elements of the Basis and the REs is only one element, which causes corner-connections (see figure \ref{fig_2dcorner_conections}).
 The Basis is directly connected with the REs at the corners. Therefore the Adapt is not obligated to fill the BZ with material and
 the BZ-elements can remain void without
 impacting the overall stiffness.
 In case $ep = 2$, the distance between the border elements of the Basis and the REs is 2 elements, which forces the Adapt to supply the BZ-elements with material. 
 The reason for this behaviour can be explained by closer examination of the load-case. Since the Basis was generated with the same load-case as the Adapt, the highest Sensitivities are next to the Basis. Furthermore not only the Adapt, but also the 88 lines of code 
are written in a way that elements are formed along
a load path and are connected to each other (requirement of topology
optimization). No elements
should generated, which are not bound to other elements.
The reason, why elements need to be connected to each other
along a load path is, that this increases the stiffness (see figure \ref{fig_2dcorner_conections}).\\

Up to now, the knowledge about the procedure of \textit{ep} was sufficient in
 order to get acceptable results and therefore modifications of the BZ-search were
 not required in 2D. However with the 3D-Adapt two problems, edge-and corner-connection, were encountered, as shown in figure
 \ref{fig_edge_edge_connec} and \ref{fig_edge_corner_connec}.\\


\begin{figure}[!h]
 \centering
 \def\svgwidth{\textwidth}
 \input{path_Image/2d_connections.pdf_tex} 
 \caption{2D connections by varying  \emph{ep}.} % Label für Verweise 
 \label{fig_2dcorner_conections}
 \end{figure}
 
In 3D there are three possible ways to create a connection between two
 elements: edge-connections, node-connections and surface-connections. The
 discretized voxel has 8 nodes, 12 edges and 6 surfaces,which can lead to 8 node-connections, 12 edge-connections and 6 surface-connections
 (figure \ref{fig_edge_3d_search}). This results in 26 possible 
 connections which can be made by a single element.
To prohibit the 26 possible connections between all border elements of the Basis
 and the REs, the 3D BZ-search methods needs to be extended. Instead of
 only considering right, left, upper, lower, elements in $z + 1$ and $z - 1$ elements
 as neighbour elements, which represent
 the 6 potential surface connections
 (see figure \ref{fig_find_BZ_2D}), the diagonal elements
 in $z-1, z$ and $z+1$ need also to be counted as neighbour elements. The figure
 \ref{fig_edge_3d_search} shows the modified 3D BZ-search with $ep =1$, where
 $\rho_e$ represents the current element and all the green, turquoise and
 yellow elements defines the search directions in order to find the 3D BZ.\\
 
  By taking the diagonal elements into account, it is possible to perform the Adapt
 with $ep = 1$ without
 getting a checkerboard type of connection.
 A comparison of the modified 3D BZ-search with $ep = 2$
 and the previous 3D BZ-search method is provided in figure \ref{fig_edges_comp_old_new_ep_1_2}, where the green BZ represents the modified BZ-search with $ep = 1$ and the blue BZ represents the prevoius BZ-search with $ep = 2$.
 Figure \ref{fig_edges_comp_new_ep_1}
 shows the modfied 3D BZ-search with $ep = 1$, where 
 the Basis is gray and the BZ is yellow.
  However, even if $ep = 2$ in 2D works fine, it is not guaranteed that 
 it is going to avoid checkerboard like connections for every problem.
 With a little effort the 
 2D BZ-search was then also modified as shown in 
 figure \ref{fig_edge_2d_search}, where $\rho_e$ represents 
 the current element and all the yellow elements are
 the search area in order to find the BZ.
 With the explained modification, creating a connection
 at corners, even with $ep = 1$ is not
 possible any more. This means border elements 
 of the Basis must be connected to the REs with at least one BZ element in between.\\
 
A comparison between the previous 3D BZ-search and the modified 3D BZ-search can
 be obtained with the figures \ref{fig_edge_3d_BZsearch_modi} to
 \ref{fig_edge_3d_2compare_mod_old}.
 Their results are generated with a mesh resolution 
 of $100 \times 50 \times 8, r_{min} = r_{b} = 1.5, ep = 2$ and $ volfrac = 0.3$.
 

%___________________3D BZ-Search-Modified_________________
\begin{figure}[!h]
\centering
\begin{minipage}{0.8\textwidth}
 \def\svgwidth{\textwidth}
 \input{path_Image/3d_diag.pdf_tex} 
 \caption{Modified BZ-search method for 3D,  \emph{ep} = 1.} % Label für Verweise 
 \label{fig_edge_3d_search}
\end{minipage}\\
\vspace{0.5cm}
%\end{figure}
% \begin{figure}[!h]
 \begin{minipage}{0.45\textwidth}
 \centering
 \includegraphics[width= \textwidth]{path_Image/pngs/Aufgabe_1/edges/compare_old_new_ep_1_2.png}
 	\caption{ Green modified BZ-search with $ep = 1$ and blue previous BZ-search with $ep = 2$ .} 
 	\label{fig_edges_comp_old_new_ep_1_2}
 \end{minipage}
 \hfill
 \begin{minipage}{0.45\textwidth}
 \centering
 \includegraphics[width= \textwidth]{path_Image/pngs/Aufgabe_1/edges/basis_new_ep_1.png}
 	\caption{ Yellow modified BZ-search with $ep = 1$ and gray Basis.}
 	\label{fig_edges_comp_new_ep_1}
 \end{minipage}
 \end{figure}   
 It can be observed that the modified 3D BZ-search exhibits a 
 bigger BZ. Therefore the 2D and 
 3D modified search-methods
 penalize more elements in the BZ than the previous search methods.\\
 %____________________________BOUNDARY-3D_search_______________________________
%________________________________________________________________

\begin{figure}[!h]
\vspace{0.75cm}
\begin{minipage}{0.45\textwidth}
\centering
 \includegraphics[width = \textwidth]{path_Image/pngs/Aufgabe_1/edges/new_BZ_ep_2.png}
	\caption{Modified 3D BZ-search.} 
	\label{fig_edge_3d_BZsearch_modi}
\end{minipage}
\hfill
\begin{minipage}{0.45\textwidth}
\centering
 \includegraphics[width = \textwidth]{path_Image/pngs/Aufgabe_1/edges/old_BZ_ep_2.png}
	\caption{Previous 3D BZ-search.} 
	\label{fig_edge_3d_BZsearch_ohne}
\end{minipage}\\

\vspace{0.75 cm}
\begin{minipage}{0.45\textwidth}
\centering
 \includegraphics[width = \textwidth]{path_Image/pngs/Aufgabe_1/edges/new_BZ_Basis_ep_2.png}
	\caption{Modified 3D BZ-search, gray Basis and yellow BZ.} 
	\label{fig_edge_3d_BZBasissearch_modi}
\end{minipage}
\hfill
\begin{minipage}{0.45\textwidth}
\centering
 \includegraphics[width = \textwidth]{path_Image/pngs/Aufgabe_1/edges/old_BZ_Basis_ep_2.png}
	\caption{Previous 3D BZ-search, gray Basis and yellow BZ.} 
	\label{fig_edge_3d_BZBasissearch_ohne}
\end{minipage}
\end{figure}

\begin{figure}[!h]
\vspace{0.75 cm}
\begin{minipage}{0.45\textwidth}
\centering
 \includegraphics[width = \textwidth]{path_Image/pngs/Aufgabe_1/edges/compare_old_new_ep_2.png}
	\caption{Yellow modified 3D BZ-search and blue previous search.} 
	\label{fig_edge_3d_compare_mod_old}
\end{minipage}
\hfill
\begin{minipage}{0.45\textwidth}
\centering
 \includegraphics[width = \textwidth]{path_Image/pngs/Aufgabe_1/edges/2_compare_old_new_ep_2.png}
	\caption{Difference between modfied and previous 3D BZ-search.} 
	\label{fig_edge_3d_2compare_mod_old}
\end{minipage}
\end{figure}
\newpage
 
 \begin{figure} [!h]
 \vspace{0.75cm}
\centering
\begin{minipage}{0.5\textwidth}
 \def\svgwidth{\textwidth}
 \input{path_Image/BZ_2_2d.pdf_tex} 
 \caption{Modified BZ search method for 2D, $ep = 1$.} % Label für Verweise 
 \label{fig_edge_2d_search}
\end{minipage}
\hfill
\begin{minipage}{0.45\textwidth}
Owing to the different penalization factors which are used for the REs and the BZ,
 the elements are not imposed with equal penalization. The penalization factor for
 the BZ is set to 3, whereas the penalization factor for the REs is set to 2. To
 understand the arising difference by using different penalization factors,
 respectively to see the functions behaviour by means of different penalization
 factor, see figure \ref{fig_SIMP_sceme_modified}. In 
 conclusion, due to the more severe penalization in the BZ 
 and deploying the modified BZ-search-method, not only 
 more elements are found to be penalized, but also the 
 penalization is heavier. Therefore even if the modified BZ-search
 finds a greater BZ than the previous BZ-search, however
 due to the more servere penalization,
 the BZ exhibits less elements with a high density.
 With this combination it is possible 
 to prohibit corner- and edge-connection to occur. 
 The described effect can be observed with a comparison, 
 the figure \ref{fig_edge_3d_micth2_modi} and
 \ref{fig_edge_3d_micth2_ohne} are obtained 
 with the same predefining and the modified load-case
 Cantilever\textsubscript{2} from
 section \ref{section_3d_loadcases}, which stresses each node in depth.
\end{minipage}
\end{figure}


%_____Adapt Mitchell-3D_with|edge without_______________________________
%________________________________________________________________

\begin{figure}[!h]
\begin{minipage}{0.45\textwidth}
\centering
 \includegraphics[width = \textwidth]{path_Image/pngs/Aufgabe_1/edges/adapt_mitchel_02_new.png}
	\caption{Modified 3D BZ-search, gray Basis.} 
	\label{fig_edge_3d_micth_modi}
\end{minipage}
\hfill
\begin{minipage}{0.45\textwidth}
\centering
 \includegraphics[width = \textwidth]{path_Image/pngs/Aufgabe_1/edges/adapt_mitchel_02_old.png}
	\caption{Previous 3D BZ-search, gray Basis.} 
	\label{fig_edge_3d_micth_ohne}
\end{minipage}\\

\vspace{0.75cm}
\begin{minipage}{0.45\textwidth}
\centering
 \includegraphics[width = 0.75\textwidth]{path_Image/pngs/Aufgabe_1/edges/adapt_mitchel_02_new_2.png}
	\caption{Modified 3D BZ-search, gray Basis.} 
	\label{fig_edge_3d_micth2_modi}
\end{minipage}
\hfill
\begin{minipage}{0.45\textwidth}
\centering
 \includegraphics[width = 0.75\textwidth]{path_Image/pngs/Aufgabe_1/edges/adapt_mitchel_02_old_2.png}
	\caption{Previous 3D BZ-search, gray Basis.} 
	\label{fig_edge_3d_micth2_ohne}
\end{minipage}
\end{figure}~\\


\section{Results with modfied 3D BZ-search method}
\label{section_results_modified_3D_BZ_search}
In this section two results with the modified 3D BZ-search method are given. The
 load-cases are Cantilever\textsubscript{2} and the MBB from section
 \ref{section_3d_loadcases}. The figures
 \ref{fig_edge_modi_canti_03} - \ref{fig_edge_modi_mbb_05} are all
 generated with a mesh resolution of $100 \times 50 \times 8, 
 \; r_{min} = r_b = 1.5, 
 \; ep = 2$ and $volfrac = 0.3$. 
 The figures \ref{fig_edge_modi_canti_03} and \ref{fig_edge_modi_canti_05} are
	generated with the loadcase Cantilever\textsubscript{2} and are 
 shown with a display threshold of \emph{0.3} and \emph{0.5}.
 The figures \ref{fig_edge_modi_mbb_03} and 
 \ref{fig_edge_modi_mbb_05} are obtained with the load-case 
 MBB modified and are also shown 
 with a threshold of \emph{0.3} and \emph{0.5}. The
 Basis is gray coloured and the AMs have the colormap
 \emph{Blue to Red Rainbow}.

\begin{figure}[!h]
\vspace{0.75cm}
\begin{minipage}{0.45\textwidth}
\centering
 \includegraphics[width = \textwidth]{path_Image/pngs/Aufgabe_1/edges/mitchell_03.png}
	\caption{Gray Basis, threshold 0.3.} 
	\label{fig_edge_modi_canti_03}
\end{minipage}
\hfill
\begin{minipage}{0.45\textwidth}
\centering
 \includegraphics[width = \textwidth]{path_Image/pngs/Aufgabe_1/edges/mitchell_05.png}
	\caption{Gray Basis, threshold 0.5.} 
	\label{fig_edge_modi_canti_05}
\end{minipage}\\

\vspace{0.75 cm}
\begin{minipage}{0.45\textwidth}
\centering
 \includegraphics[width = \textwidth]{path_Image/pngs/Aufgabe_1/edges/mbb_03.png}
	\caption{Gray Basis, threshold 0.3.} 
	\label{fig_edge_modi_mbb_03}
\end{minipage}
\hfill
\begin{minipage}{0.45\textwidth}
\centering
 \includegraphics[width = \textwidth]{path_Image/pngs/Aufgabe_1/edges/mbb_05.png}
	\caption{Gray Basis, threshold 0.5.} 
	\label{fig_edge_modi_mbb_05}
\end{minipage}
\end{figure}


