\chapter{Introduction}

The construction of a mechanical component comes along with mechanical requirements, such as
stiffness, solidity, mass-restrictions, costs and
whether the desired component offers manufacturability.
In order to meet these challenges, optimization algorithms are useful tools. 
Most algorithms are deployed iteratively and affect the
geometrical properties.\\

The so called \emph{topology optimisation} is a well established method
used by engineers,
for example in the field of aeronautical, civil, materials and mechanical structural optimisation.
One example for
topology optimisation can be
stated as, for a given mass-restriction and a 
design domain, topology optimisation distributes
 material in order to minimize an objective function.
 The design domain can be an area  or a volume.
The optimisation algorithms discretize
the geometry and decide for each
element whether it needs to be void (holes) or filled with material.\\

The Adapt is a topology optimiser, written in Matlab and up to now,
it can be used to increase the stiffness of a 2D structure.
Therefore the optimiser obtains a
structure, which is a matrix from Matlab.
Each entry of the matrix represents a discretized element 
containing values between 0 and 1.
These values stand for the the density of each element.
After having loaded a structure into
the Adapt, the following tasks
are performed, locating non design- and design-space,
structural analysis, with a defined load-case, sensitivity analyse,
sensitivity filtering and update of
density distribution using the Optimality Criterion (OC). The Adapt serves
to generate connection-zones on the Basis and connect
multiple connection-zones with
struts instead of thicken the loaded structure or Basis. \\

The Adapt can be used in every discipline for which it is desired
to increase the overall stiffness of a structure, e.g.
the racing department at the company BMW would like to use the bodywork of an existing production
car for motor sports. In order to make this possible
the higher loads that occur during racing must
be  absorbed by the body of the series car.
To ensure that the existing body series does not fail,
its stiffness needs to be increased.
It would be conceivable to create a completely new
body series, however, this would be inefficient for economic reasons.\\

In order to be able to use the Adapt on a complete vehicle,
the Adapt needs various modifications. First it needs
to work in 3D, it should not only be able to load
structures generated with Matlab, but any given 2D or
3D structure from a CAD-software. It would also be
conceivable to control the distance of the individual
attached bindings, as well as the thickness of the individual bindings.\\

The main aim of this work is to extend the Adapt for 3D load cases,
verify whether the 3D Adapt ensures an increase
of stiffness, load any 2D or 3D structure into the Adapt and enable
the optimization of an Adapt load case, which is slightly different from
the Basis load case.



