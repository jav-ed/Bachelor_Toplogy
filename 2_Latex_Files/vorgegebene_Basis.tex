
\chapter{Topology optimization with an external 2D and 3D Basis}
\label{chapter_external_basis}

Until now, it was only possible to
 take structures created with
  Matlab as a Basis. Since the Adapt is
  supposed to optimise an entire vehicle, it 
  is desirable
   to include  any given  structure as
   the Basis. For this purpose, it is going to be explained, how to
    import a 2D or 3D structure, generated with a common
     CAD sofwares, e.g.CATIA , into Matlab and use the imported
     structure as the Basis. Afterwards it is going to be clarified,
     why optimising with an external Basis requires
     the already presented features and finally 
     two applications are shown.
     Note that the CAD program must be able to save the CAD drawing as
      a .stl file and the discretization takes place exclusively as
      squared elements in 2D and as voxels in 3D. 
      
      \section{External Basis}
      \begin{figure}[H]
      \begin{minipage}{0.45\textwidth}
            In the previous chapter \ref{chapter_features_adapt} three Adapt
      features were presented. These features enable to get the control
      of the maximal tangential thickness of the CZGs, remove
      locally reinforcements and define a minimal distance between
      the CZGs. The reason why these features were implemented,
      is because, it was desired to use the Adapt for slightly different and 
      new load cases, which can be seen in figures
      \ref{fig_slighlty_different_load} and
      \ref{fig_new_load}. An external Basis as shown in figure
      \ref{fig_ext_bas_1} is obtained without any predefined load-case.
      The Basis, which was used up to now, was always
      generated with a Basis load-case though the topology
      optimisation codes \cite{Andreassen.2011} and \cite{Liu.2014}.
          \end{minipage}
      \hfill
            \begin{minipage}{0.45\textwidth}
      	\centering
  \includegraphics[width= \textwidth]{path_Image/pngs/Aufgabe_3/grbegri_5.png}
	\caption{Random external Basis.} 
	\label{fig_ext_bas_1}
      \end{minipage}
      \end{figure}
      
      \begin{figure}[!h]
      \begin{minipage}{0.45\textwidth}
\centering
  \includegraphics[width= \textwidth]{path_Image/pngs/Aufgabe_3/grbegri_6.png}
	\caption{Load close to loadpaths.} 
	\label{fig_ext_bas_2}
\end{minipage}
\hfill
\begin{minipage}{0.45\textwidth}
	\centering
  \includegraphics[width= \textwidth]{path_Image/pngs/Aufgabe_3/grbegri_7.png}
	\caption{Load far from loadpaths.} 
	\label{fig_ext_bas_3}
\end{minipage}
      \end{figure}
	An external
      Basis is not generated
      though topology optimisation and
      therefore it does not have a load-case and also there is no such
      event, in which the Adapt could have the same load-case and loadpaths as
      an external Basis. However, there is something similar to
      a slightly different load-case, the event is called
     \emph{ load close to the load paths} and can be seen in
      the figure \ref{chapter_external_basis}.
      The 
      previous new load is similar to a load, which is\emph{ far
      away from the load paths} of the external Basis, which
      can be obtained from the figure
      \ref{fig_ext_bas_3}. Since in case of a slightly different load-case
      and without the features,
      the Adapt produces
      results with wide connections and local reinforcements
      and therefore a calculation with an external Basis
      is requiring the features in order to prohibit the
      occurrence of wide connections and local reinforcements.
	The following sections are going to explain
	which method is used
      in order to load an external Basis into Matlab.
      
\section{Discretize external 2D and 3D Basis}
The idea for working with an external Basis is to load any
\emph{.stl}-file in Matlab. This is done by
discretizing the \emph{.stl} 2D or 3D object.
The code for this
purpose is called Mesh voxelisation and 
 is available free of charge on MathWorks File Exchange. Note
 the Mesh voxelisation is not written by the author of this bachelor thesis.
 MathWorks File Exchange also explains  the principles of the code,
 therefore the reader is referred to it.
  In order to use Mesh voxelisation in Adapt, 
  the Matrix-Ouput, which is 
  obtained after running
  Mesh voxelisation,
  must be converted from the type  \textit{Boolean}
   to type \textit{Double}. The
    matrix can then be regarded as a
    completely usual Matlab matrix consisting 
    exclusively of 0 and 1. 
    In order to discretize in 2D, the depth parameter must be set to \textit{1}.

\section{Results with external Basis}
  \begin{figure}[!h]
  \begin{minipage}{0.45\textwidth}
  The selected external 2D and 3D Basis
 have been drawn with CATIA. The 3D Basis
 is extruded in the depth, which is the only difference between
 the 2D and the 3D Basis. The 2D Basis
 is presented in
 \ref{fig_externe_2d_basis}.
  It consists of a frame, which occupies 20\%
   of the available construction space 
  in both 2D and 3D. 
  \end{minipage}
  \hfill
    \begin{minipage}{0.5\textwidth}
    \centering
    \includegraphics[width=  \textwidth]
    {path_Image/pngs/Aufgabe_4/externe2d_basis.png}
  	\caption{Gray external 2D Basis.} 
  	\label{fig_externe_2d_basis}
  \end{minipage}
  \end{figure}
  
  The figures \ref{fig_externe_basis_02} and
  \ref{fig_externe_basis_04} are obtained with the 
  load case in figure \ref{fig_loadcase_canti_2d} and the
  use of the features, where \ref{fig_externe_basis_02}
  show the results with a display threshold of \emph{0.2} and
  the figure \ref{fig_externe_basis_04} show the
  results with a threshold of \emph{0.4}. The Basis is gray and 
  the AMs are coloured with
  the colormap \emph{Blue to Red Rainbow}. Since the features are
  not implemented in 3D,
  the latest 3D Adapt, which prohibits the appearance of
  connections at the edges is employed. The load case is
  similar to Cantilever\textsubscript{3}, with the sole difference,
  that the force is applied on only the central node in the depth
  direction.
  The figures \ref{fig_externe_basis3d_02}
  and \ref{fig_externe_basis3d_02_2} show the results
  with a threshold of \emph{0.2} and
  the figures \ref{fig_externe_basis3d_04_2}
  and \ref{fig_externe_basis3d_04_2}
  show the figure with a threshold of \emph{0.4}.
  
    \begin{figure}[!h]
  \begin{minipage}{0.45\textwidth}
  \centering
    \includegraphics[width= \textwidth]
    {path_Image/pngs/Aufgabe_4/externe2d_basis_th_02.png}
  	\caption{Threshold: 02, gray external 2D Basis.} 
  	\label{fig_externe_basis_02}
  \end{minipage}
  \hfill
  \begin{minipage}{0.45\textwidth}
  \centering
    \includegraphics[width= \textwidth]
        {path_Image/pngs/Aufgabe_4/externe2d_basis_th_04.png}
  	\caption{Threshold: 04, gray external 2D Basis.} 
  	\label{fig_externe_basis_04}
  \end{minipage}\\
  
  \vspace{0.5cm}
  \begin{minipage}{0.45\textwidth}
  \centering
    \includegraphics[width= \textwidth]
    {path_Image/pngs/Aufgabe_4/externe3d_basis_th_02.png}
  	\caption{Threshold: 02, gray external 3D Basis..} 
  	\label{fig_externe_basis3d_02}
  \end{minipage}
  \hfill
  \begin{minipage}{0.45\textwidth}
  \centering
    \includegraphics[width= \textwidth]
        {path_Image/pngs/Aufgabe_4/externe3d_basis_th_04.png}
  	\caption{Threshold: 04, gray external 3D Basis.} 
  	\label{fig_externe_basis3d_04}
  \end{minipage}
    \end{figure}
  
      \begin{figure}[!h]
    \begin{minipage}{0.35\textwidth}
  \centering
    \includegraphics[width= \textwidth]
    {path_Image/pngs/Aufgabe_4/externe3d_basis_th_02_2.png}
  	\caption{Threshold: 02, gray external 3D Basis..} 
  	\label{fig_externe_basis3d_02_2}
  \end{minipage}
  \hfill
  \begin{minipage}{0.35\textwidth}
  \centering
    \includegraphics[width= \textwidth]
        {path_Image/pngs/Aufgabe_4/externe3d_basis_th_04_2.png}
  	\caption{Threshold: 04, gray external 3D Basis..} 
  	\label{fig_externe_basis3d_04_2}
  \end{minipage}
  \end{figure}~\\