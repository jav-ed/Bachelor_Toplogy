 \documentclass[a4paper 11 pt] {scrreprt}
 \usepackage[left= 2.5cm,right = 2cm, bottom = 4 cm]{geometry}
 \usepackage{fancyhdr}

\usepackage[english]{babel} 
\usepackage{amsmath} 
\usepackage{amssymb}

% for tables which needs more than one page
\usepackage{longtable}

%%Innerhalb von includegrpahics
%\usepackage[export]{adjustbox}

% Table colering
\usepackage[table]{xcolor}
\definecolor{lightgray}{gray}{0.9}
\definecolor{lightblue}{rgb}{0.93,0.95,1.0}

%\usepackage{tabu}
\usepackage[table]{xcolor}

\definecolor{tableHeader}{RGB}{211, 47, 47}
\definecolor{tableLineOne}{RGB}{245, 245, 245}
\definecolor{tableLineTwo}{RGB}{224, 224, 224}

%________________

%If you want to generate a caption without beeing inside of a flot object
\usepackage{caption}
\usepackage{subcaption}

% float forces the graphics to be on the Postisiton [♦H]
\usepackage{float}

%Clickable Ref
\usepackage{hyperref}

%PDF.TEXT -Graphics
\usepackage{pst-plot} 
\usepackage{graphicx}

% Code-implementation matlab
\usepackage[numbered,framed]{matlab-prettifier}

%Um Tikz einzuladen
\usepackage{pgfplots}

% Abkürzungsverzeichnis
\usepackage{acronym}

% Um den Namen des Autos aus Bibtex anzuzeigen
%\usepackage[round, authoryear]{natbib}
%\usepackage{natbib}
%\usepackage[numbers]{natbib}
%\usepackage[round, sort]{natbib}}
\usepackage{apacite}

%Zitationsstil
%\bibliographystyle{unsrt}
%\bibliographystyle{unsrtnat}
%\bibliographystyle{plainnat}
\bibliographystyle{apacite}

% Tables coleriert
\usepackage{xcolor,colortbl}

\newcommand{\mc}[2]{\multicolumn{#1}{c}{#2}}

\definecolor{Gray1}{gray}{0.85}
\definecolor{Gray2}{gray}{0.3}
\definecolor{Green}{HTML}{82FA58}
\definecolor{LightCyan}{rgb}{0.88,1,1}

\newcolumntype{L}[1]{>{\raggedright\arraybackslash}p{#1}}
\newcolumntype{C}[1]{>{\centering\arraybackslash}p{#1}}
\newcolumntype{R}[1]{>{\raggedleft\arraybackslash}p{#1}}

\newcolumntype{a}{>{\columncolor{Gray1}}c}
\newcolumntype{b}{>{\columncolor{Gray2}}c}

%_______________Kopfzeile____________
\pagestyle{fancy}
\fancyhf{}
\fancyhead[L]{\rightmark}  % links kommt mit \rightmark die section
\fancyhead[R]{\thepage}  % rechts kommt die Pagenr
\renewcommand{\headrulewidth}{0pt}
%:::::ENDE KOPFZEILE____________________

%Für pdf_tex
%_____Muss am Ende sein ______
\usepackage{color}
\usepackage{transparent}
\graphicspath{{path_Image/}, {path_Image/pngs/Aufgabe_1/Parameter/},{path_Image/pngs/Aufgabe_1/3D_Ergebnisse},{path_Image/pngs/Aufgabe_1/Parameter/orig}, {path_Image/pngs/Aufgabe_2},
{path_Image/pngs/Aufgabe_3}, {path_Image/pngs/Aufgabe_4},
{path_Image/pngs/Meet_Adapt}, {path_Image/pngs/Paraview}}

% Definiere hier einen eigenen Befehl, um die Größe der TIkz auszuwaehlen
\newcommand{\inputTikZ}[2]{%  
     \scalebox{#1}{\input{#2}}  
}

\newcommand{\repeatcaption}[2]{%
  \renewcommand{\thefigure}{\ref{#1}}%
  \captionsetup{list=no}%
  \caption{#2 (repeated from page \pageref{#1})}%
}



\begin{document}

%% Deckblatt
\begin{center}
\begin{tabular}{p{\textwidth}}

\begin{minipage}{0.5\textwidth}
\centering
\includegraphics[scale=0.12]{path_Image/pngs/Aufgabe_3/bmw.png}
\end{minipage}
\begin{minipage}{0.5\textwidth}
\centering
\includegraphics[scale=0.5]{path_Image/pngs/Aufgabe_3/uni_wupp_1.png}
\end{minipage}



\\

\begin{center}
\LARGE{\textsc{
Extension of a topology optimisation method for the adaptive reinforcement of mechanical structures from 2D to 3D
 \\
}}
\end{center}

\\


\begin{center}
\large{Bergische Universität Wuppertal \\
Department of Mechanical Engineering and Safety Engineering \\
Chair of Optimization of Mechanical Structures
}
\end{center}

\\

\begin{center}
\textbf{\Large{Thesis}}
\end{center}


\begin{center}
to obtain the academic degree\\
Bachelor of Science
\end{center}


\begin{center}
written by
\end{center}

\begin{center}
\large{\textbf{Javed Butt}} \\

\large{1532570} \\
\end{center}

\begin{center}
\large{born on 20.05.1996 in Gujrat}
\end{center}
\\

\begin{center}
\begin{tabular}{lll}
\textbf{Submission date:} & & 19.02.2019\\
\textbf{Initial examiner :} & & Prof. Dr.-Ing. Axel Schumacher\\
\textbf{Second examiner :} & & Fabian Plate (M.Sc.)\\
\textbf{Supervisor :} & & Saad Hafsa (M.Sc.)\\


\end{tabular}
\end{center}

\end{tabular}
\end{center}
%Damit die erste Seite = Deckblatt nicht nummeriert wird.
\thispagestyle{empty}

%%Sperrvermerk
% \input{2_Latex_Files/Restriction}

\chapter*{Abstract}

The aim of this thesis is to extend a Matlab code, which has been extensively investigated for academic 2D load cases, to the third dimension. For this purpose, Matlab will be continued. The proven 2D code is an adaptation topology optimization method, with which it is possible to mechanically reinforce an existing structure in retrospect by adding new structures.
In this regard, a method is presented with which corner and edge connections
can be prevented.
In order to have a reasonable post processor for both 2D and 3D results,
 Paraview is introduced.
Furthermore, FEA (Finite Element Analysis)
with Hyperworks Optistruct, are 
performed to see if the overall stiffness is increased through the
Adapt.\\

Afterwards features are presented with 
which the Adapt-load can be slightly different or a new load
compared to the Basis-load.
These features include, controlling the tangential thickness
of a connection, defining a minimum distance between
multiple connections and preventing local reinforcements.
Finally a method is presented in order to 
load a .stl file into Matlab. This serves to 
load a external Basis in both 2D and 3D for the Adapt.
The appendix offers brief explanations of some 
useful paraview-functions for 3D topology optimisation. 

 
%
%%%Danksagung
\chapter*{Acknowledgement}

Note: This section was modified from the version that was submitted in the original bachelor's PDF.\\

All praise and thanks to the \textbf{ONE}, Who does neither need my praise nor my thanks. 
To the \textbf{ONE}, Who is independent of everything and everyone, but on Whom everything and everyone depends.

\vspace{1cm}
Mama, even though, I have some troubles to understand you, I am very grateful to you, thank
your love, commitment and effort. And by the way, I never really belived, that you could raise
your 4 children properly, however, with two you already managed it halfway.\\

Api, dear duck, thank you for all your efforts, despite all the stress you always put on yourself, you still played the role of the older sister perfectly. Amna and Jasmin, funny people - thank you for being so unique, in your own lovely way and Pa Waseem, you are a man from whom I can learn a lot. \\


Mister best Aburajab  thank you for letting me always rely you and dear Nikolina Migli\^{c},
I am happy to have you back in my life again. \\

Saad Hafsa, I would also like to thank you, thank you, if you would not have spend
your nights working
with me on this project, I would have spent 3 more months with the Bachelor thesis - thank you for your friendship and thank you Mariam for spreading good mood at BMW.\\

Finally, I would like to
thank Prof. Dr.-Ing. Axel Schumacher,  Jens Trilling and  Miram Kick, who
made the bachelor thesis at BMW possible for me. 



%%Inhaltsverzeichnis
\tableofcontents
%%Füge List of abbreviations zum INhaltsverzeichnis hinzu, ohne
%% eine Zählung davor.
\addcontentsline{toc}{chapter}{List of abbreviations}

%%Abkurzungsverzeichnis
\chapter*{List of abbreviations}

\label{sec:abkuerzungsverzeichnis}

\begin{acronym}[slmtA]
 \acro{AM}{Adapt Member}
  \acro{BC}{Boundary Condition}
 \acro{BZ}{Boundary Zone}
 \acro{CAD}{Computer Aided Design}
  \acro{CZ}{Connection Zone}
    \acro{CZG}{Connection Zone Group}
    \acro{DE}{Design Element}
	\acro{DOF}{Degree Of Freedom}
 \acro{FE}{Finite Element}
 \acro{FEA}{Finite Element Analysis}
 \acro{FEM}{Finite Element Method}
  \acro{GPU}{Graphic Processing Unit}
    \acro{MMA}{Method of Moving Asymptotes}
    \acro{OC}{Optimality Criterion}
    \acro{RE}{Remaining Element}
    \acro{SA}{Sensitivity Analysis}
 \acro{SIMP}{Solid Isotropic Material with Penalization of intermediate
densities}

\end{acronym}


%G = shear modulus
%E = young's modulus
%c = compliance
%x = design varibale
%$\rho$ = density
%$\nu$ = poisson ratio
%$\sigma$ = stress
%$\varepsilon$ = strain (Dehnung)



%%ABbildungs und Tabellen verzeichnis
%%Verzeichnis aller Bilder
\listoffigures

%Verzeichnis aller Tabellen
\listoftables



%%
\chapter{Introduction}

The construction of a mechanical component comes along with mechanical requirements, such as
stiffness, solidity, mass-restrictions, costs and
whether the desired component offers manufacturability.
In order to meet these challenges, optimization algorithms are useful tools. 
Most algorithms are deployed iteratively and affect the
geometrical properties.\\

The so called \emph{topology optimisation} is a well established method
used by engineers,
for example in the field of aeronautical, civil, materials and mechanical structural optimisation.
One example for
topology optimisation can be
stated as, for a given mass-restriction and a 
design domain, topology optimisation distributes
 material in order to minimize an objective function.
 The design domain can be an area  or a volume.
The optimisation algorithms discretize
the geometry and decide for each
element whether it needs to be void (holes) or filled with material.\\

The Adapt is a topology optimiser, written in Matlab and up to now,
it can be used to increase the stiffness of a 2D structure.
Therefore the optimiser obtains a
structure, which is a matrix from Matlab.
Each entry of the matrix represents a discretized element 
containing values between 0 and 1.
These values stand for the the density of each element.
After having loaded a structure into
the Adapt, the following tasks
are performed, locating non design- and design-space,
structural analysis, with a defined load-case, sensitivity analyse,
sensitivity filtering and update of
density distribution using the Optimality Criterion (OC). The Adapt serves
to generate connection-zones on the Basis and connect
multiple connection-zones with
struts instead of thicken the loaded structure or Basis. \\

The Adapt can be used in every discipline for which it is desired
to increase the overall stiffness of a structure, e.g.
the racing department at the company BMW would like to use the bodywork of an existing production
car for motor sports. In order to make this possible
the higher loads that occur during racing must
be  absorbed by the body of the series car.
To ensure that the existing body series does not fail,
its stiffness needs to be increased.
It would be conceivable to create a completely new
body series, however, this would be inefficient for economic reasons.\\

In order to be able to use the Adapt on a complete vehicle,
the Adapt needs various modifications. First it needs
to work in 3D, it should not only be able to load
structures generated with Matlab, but any given 2D or
3D structure from a CAD-software. It would also be
conceivable to control the distance of the individual
attached bindings, as well as the thickness of the individual bindings.\\

The main aim of this work is to extend the Adapt for 3D load cases,
verify whether the 3D Adapt ensures an increase
of stiffness, load any 2D or 3D structure into the Adapt and enable
the optimization of an Adapt load case, which is slightly different from
the Basis load case.




\chapter{Principles of structural optimisation}
 
%% Grundlagen der Struktur-Optimierung
 \section{Term Definitions}
 \label{subsection_term_princip}
 The basic principle of optimisation is to find the best possible or at least improved solution under given conditions. In structure optimisation, these conditions are represented by parameters in mathematical functions. An example for optimisation, outside of the structure-optimisation-topic, can be, finding the fastest way to a location, with
 the restriction to only use bicycle roads.
 In the case of structural optimisation, optimisation means to fulfil the 
 objective function with given restrictions, e.g. in case of
 minimizing the compliance, the optimiser distributes material and
 the restriction can be the volume fraction.\\
 
 The objective of the optimisation problem is often maximization or minimization, this could be for example minimization of time consuming or maximizing stiffness. Note that maximization problems can easily be transformed to minimization problems by maximization the negative objective function. Mathematically the optimisation problem is most often formulated as minimisation of the objective functions. In order to make some modification, e.g. to the structure, among others sensitivities or heuristics can be used.\\
 
The following table shall help the reader to understand the meaning of common topology optimisation technical terms.

%_____________________TABELLE 1 ________________________________________
\begingroup
\renewcommand{\arraystretch}{2} % Default value: 1
 \begin{longtable}{L{0.2\textwidth} L{0.8\textwidth}}
 \hline 
 \rowcolor{Green}
 \multicolumn{1}{c}{Term} & \multicolumn{1}{c}{Explanation} \\ 
 \hline
 \rowcolor{Gray1} 
 Structure & The word structure is always referred to a mechanical structures. This can be anything, which consist of
 a solid material, e.g. a desk, a chair or a steal beam. \\ 
\hline
 Compliance & The compliance is equivalent to the strain energy and also it is the inverse is of stiffness, which basically means a high stiffness result in a small compliance. The compliance is abbreviated with the letter \textit{c}.\\
 \hline 
 \rowcolor{Gray1} 
 Material distribution & Material distribution means, that the optimiser has to define at which places, inside of the available construction space, mass needs to be set.\\

 \hline
Mechanical property & A mathematical description of a property, which is going to be optimised, e.g. mass, displacement or compliance (strain energy).\\
 \hline 
 \rowcolor{Gray1} 
Discretization & Any mechanical structure, for example a desk, fills an
area or volume. This area or volume can be disassembled into smaller
sub-areas or sub-volumes. Each sub-area or sub-volume is considered as
an element.
A high number of elements results in a fine mesh and analogously a small number results in a coarse mesh. See also figures \ref{fig_2d_discr} \ref{fig_3d_discr}. \\

 \hline
 Grids & Grids are mostly understand to be elements for a simulation, which are discretized in a squared or rectangular order. Because of this order, the structure is well defined. \\
 \hline
  \rowcolor{Gray1} 
Mesh & The term mostly refers to a more general expression. In this case, the discretization does not have to be squared, it can be done by various shapes of elements and there is also the possibility to mix elements of different types in the same mesh.

Note, there is no official distinction
between the terms \textit{Grid} and
 \textit{Mesh}. However, they might be used in practical in slightly different ways. "The following definitions are more guidelines of common usage than actual rules and you may hear people use them interchangeably in many cases"\footnotemark \\
\hline
Optimiser & The optimiser is the short term of optimisation algorithm or optimisation code.\\

\hline
 \rowcolor{Gray1} 
Design variable & The optimiser follows the rules or instructions which were given through mathematical functions. These functions require parameters to perform their task. If a parameter contributes to obtain a structure, then the parameter is called design variable.
A design variable is parameter, which is modified by the optimiser e.g. the density 
is one of most common design variable in topology optimisation. \\
 \hline
 \footnotetext{https://scicomp.stackexchange.com/questions/17606/whats-the-difference-between-grid-based-and-mesh-based-methods-for-pdes, 29.11.2018 - 19:36}
% 
%\centering \def\arraystretch{1.5} \small
%
%%________________ZWEIITE TABELLE____________________________________________
%
% \begin{tabular}{|p {3cm}|p {12cm}|}
Volume fraction & In topology optimisation milieu volume fraction is better known by its abbreviation \textit{volfrac}. In the case of this work, it defines how much volume of the initial volume is allowed to be used to generate a topology optimised structure.
 \textit{Volfrac} is a value between 0 and 1, where 0 means, no material at all and 1 means, the optimiser must not act at all, because it has not reduce the initial volume.
 
$$ volfrac = \frac{current \; Volume}{total \; Volume}$$\\

\hline
 \rowcolor{Gray1} 

 %Hier eine Refernez einbauen zu einem Bild
Design domain & The design domain is the construction space containing design variables. \\
 \hline 

Non design domain & The non design is the
domain of the construction space containing structural elements,
but no design variables. The non design can be
e.g. the domain, which is restricted by initial 
conditions like boundary conditions. \\
\hline
  \rowcolor{Gray1} 

Available construction space & The available construction space is the complete domain, which can contain mass in order to generate a mechanical structure.

\textit{available construction space = design domain + non design domain} \\

\hline

Objective function: & The objective function will be minimized
by the optimiser. The definition of the objective
function subjects to a mathematical function, e.g. f(x,u). 
For structural optimisation issues the objective function is going to be a structure property or a mathematical expression of multiple structure properties. \\
\hline
 \rowcolor{Gray1} 

Constraint & The constraint can be defined as a value of an output that has to be matched by the optimizer, e.g. \textit{volfrac = 0.3}.\\
\hline

Optimisation problem & Collection of the objective functions, constraints and the design variables as mathematical functions. An example for a complete optimisation problem could be for instance minimizing the mass of a structure by following the stiffness-constraints and also following the maximum wall thickness of a structure.\\
\hline
 \rowcolor{Gray1} 
Sensitivity & The sensitivity is the derivative of a structure property with respect to a design variable, e.g. the derivation of compliance with respect to design
variable (density).\\
\hline
\label{tabel_basics}
\end{longtable}
\endgroup

 
% __________Types of structual optimisation______________________
\section{Types of structual optimisation} 
Figure \ref{fig_struc_arten} shows a classification of structural optimisation tasks based on the type of the design variable. \\

{\large{\textbf{Construction:}} }
The choice of the construction is the most general way of optimising. Therefore this method of optimising, mostly leads to the best results. Since
generally the optimiser has no implementation for
choosing a design type, the optimiser is not restricted to any kind of design type.
The decision of the design type defines whether the loaded structure needs to be a metal profile,a truss or a composite.\\

 {\large{\textbf{Topology optimisation:}}}
 The design variables defines the position and order of the structure elements
 \cite{Schumacher.2013}.
 A doughnut and a ring have the same topology,
 the same applies to a ring and a cup of tea. 
 All the mentioned examples exhibits a hole. With topology optimisation the resulting shape or topology is not known, the number of holes and components like beams or
 rods are not decided yet. From a given available construction space the purpose is to find the optimum distribution of material and voids. In
 the case of this work, the solution of the problem can
 be obtained, by discretization using the finite element method (FEM) (for more information see section \ref{section_FEA})
 and dividing the available construction space into discrete elements (mesh or grid).
 The upcoming task is then accomplished
 by means of optimisation methods, which decide whether an
 element is material
 respectively mass, or not. This result is a 0 - 1 problem, the elements either exists or not,
 which is basically an integer problem with two different possible outcomes for each element, a so called ISE topology (Isotropic Solid or Empty elements)
 \cite{Olason.2010}.\\
 
The number of different combinations is $2^N$, where N is devoted as the number of elements. For example, a model with $10 000$ elements
 would result in $2^{10000}$ numbers of combinations.
 With the
 assumption an evaluation of
 one million combinations per second,
 testing all of the combinations would take about $3.17 \times 10^{2996}$ years 
\cite{Olason.2010}.\\

The two mainly used methods for solving optimisation
 problems with respect to ISE topology (Isotropic Solid or Empty elements) 
 are the homogenization and density method. 
 Other methods, which will not be further explained, can make use of genetic algorithms or heuristic methods such 
 as ESO (Evolutionary Structural Optimization) or BESO (Bi-directional Evolutionary Structural Optimisation). For a detailed compare of some optimisation
  method see \cite{Sigmund.2013}.\\
  
   {\large{\textbf{Shape optimisation:}}}
The design variables describe the shape of the components border.
The geometry of the component can be changed, however an insertion of new
structural elements such as cavities and struts are not allowed \cite{Schumacher.2013}.
Shape optimisation does not affect or does not change the topology and
thus the number of holes, beams and rods of the structure will not be changed
\cite{Olason.2010}.\\

{\large{\textbf{Sizing optimisation:}}}
Sizing optimization is known for being the simplest form of structural optimization. The shape of the strcuture is
known already and the objective is to optimize the strucutre. This optimisation is performed by adjusting
sizes of the components. The design variable can be stated as the size of the structural elements, e.g.
the diameter of a rod or the thickness of  a beam \cite{Olason.2010}. \\

\begin{large}
\textbf{Material:}
\end{large}
The choice of the material has a big impact on almost every mechanical property of the
whole structure.\\

\begin{figure} [!h]
 \centering
 \def\svgwidth{0.85\textwidth}
 \input{path_Image/arten_struktur.pdf_tex}
 \caption{Different types of structural optimisation. } % Bildunterschrift 
 \label{fig_struc_arten}   % Label für Verweise 
\end{figure} 


%________________TOPOLOGY________________________________

\subsection{Homogenization method}
The desired result of topology optimization is
to find  that the domain either contains material or exhibits void regions. Since it is difficult to work mathematically with integer variables, thus relaxation is usually applied, 
like in the density method. Homogenisation is
 to be applied on holey or 	porous material. The concept 
 of homogenization is it to homogenize porous or 
 holey material. In order to obtain informations about
 material properties microscopic cells are evaluated. 
 These microscopic cells are constituted of massive 
 material and a domain without any material. 
 This formulation serves to simulate porous material behaviour.
 Some common types of micro-structure are solids with square or rectangular holes. 
 Since the macroscopic properties of the microscopic cells are not isotropic an orientation angle is also introduced. 
 The elasticity mostly needs to be calculated numerically, e.g. by means of finite elements method and then interpolating between these values. 
 In order to obtain a structure, which is manufacturable, it is desired not to have any intermediate desensitise. This can be achieved by penalizing the intermediate densities. The homogenization methods brings by themselves a penalization along. Because the 'intern' penalization of the homogenisation method often does not suit the requirement, some addition penalization needs to be applied. For more information see \cite{Bendse.1988}\\

The optimising is performed similarly to the density method then
the problem is discretized with FEM (Finite element Method) with the design variables (hole sizes and rotation) assumed to be constant over each element
\cite{Olason.2010}. The homogenisation method demands more than one design variable per element, which results in a disadvantage because it requires more optimising effort than the density method.
 
\subsection{Density method}
As mentioned topology optimisation can be defined, e.g.
 as a binary (0-1) programming problem that has the goal to find the optimal material layout (solid and void) for minimizing an objective function. The density based method relaxes the integer-based topology optimisation problem on artificial continuous material densities. By applying relaxation, intermediate densities between 0 and 1 is allowed and as a consequence the objective and the constraint functions become continuous. Furthermore they become differentiable, this permits the usage of gradient solvers to find a minimum of the objective function. The density method is basically a simplification of the homogenisation method and in contrast to the homogenization method the density method only has one design variable, which can be, e.g. the density or elasticity. 
 
\section{Material interpolation}
Material interpolation schemes allow intermediate densities, but has the purpose to penalize the intermediate densities
 at the same time in order to obtain the original binary requirement. The so called SIMP (Solid Isotropic Material with Penalization, \cite{Bendse.1988}) and RAMP (Rational Approximation of Material Properties) are two well known examples for material interpolations schemes, both methods supply a relation between (relative) elements density \textbf{$\rho_e$} and the stiffness
 $E_e$.
  For a detailed compare of these two material interpolation schemes see \cite{Hvejsel.2011}.\\

\subsection{Solid Isotropic Material with Penalization (SIMP)}
\label{subsection_SIMP}
SIMP is a material interpolation scheme, which serves to penalize intermediate densities. Penalization of intermediate densities is necessary due to desired binary results (intermediate densities can not be manufactured). Since 2007 there are two available SIMP approaches, the classical SIMP by \cite{Bendse.1988} and the modified SIMP by \cite{Sigmund.2007}. The only mathematically difference between those two variants is the additional term $E_{min}$, see \eqref{eq_simp_1}.
 SIMP attempts to eliminate intermediate densities, but it is not able to eliminate all of them. However, it still affects the convergence behaviour of the design variable $\rho$ in a beneficial way, the convergence becomes faster,
 $x_e \approx 0 $ or $x_e = 1$.\\
 
According to \cite{Bendse.1988} SIMP can be expressed as: 
\begin{align}
E_e(x_e) = x_{e}^p E_{0}, \quad x_e \in [0,1]
\end{align}
and according to \cite{Sigmund.2007} SIMP can be expressed (figure \ref{fig_SIMP_sceme_modified}) as:

\begin{equation}
\label{eq_simp_1}
 E_e(x_e) = E_{min}+ x_e^p (E_0-E_{min}), \quad x_e \in [0,1]
 \end{equation}\\
 
Notation: The index \emph{e} shall define the current considered element, \textbf{$x$}
 is the design variable, in the case of this work
  it is the density \textbf{$\rho $}, \textbf{$p$} stands for 
  the penalization exponent, \textbf{$E_{0}$} is the stiffness of the material, \textbf{$E_{min}$ }is a very small stiffness assigned
   to void regions in order to prevent the stiffness matrix from becoming singular 
  and \textbf{$E_e$} is the new calculated Young's modulus with the usage of SIMP. The penalty exponent \textbf{$p$}
   is mostly chosen to be \textbf{$1<p \leq 3$}. The bigger \textbf{$p$} gets, the more intermediate densities are avoided or penalized. By defining \textbf{$p>>3$} the number of locally minima may increases that much, that the optimisation results getting worse \cite{Dienemann.2018}.\\

The classical SIMP approach avoids the stiffness to be zero, thus the the densities has to be \textbf{$ 0 < x_{min} \leq x_{e} \leq 1 $}, where \textbf{$x_{min}> 0 $}. As mentioned\textbf{ $E_{e} = 0$} would cause a singular stiffness matrix. The modified SIMP allows a straight forward implementation of filters, e.g. the heaviside projection filter
\cite{Andreassen.2011}. For more information about
 different kinds of filter inclusive the mentioned heaviside filters see
 \cite{Andreassen.2011} and to obtain a wider compare between the origin SIMP and the modified SIMP see \cite{Sigmund.2007}. \\

%______________SIMP_TIKZ________________________
\begin{figure}[!h]
\centering
\inputTikZ{1}{path_Image/SIMP_modified.tikz}
\caption{Modified SIMP material-interpolation-scheme, with $E_{min} = 1 \times 10^{-4}$. }
\label{fig_SIMP_sceme_modified}
\end{figure}

\subsection{RAMP- Rational Approximation of Material Properties}
RAMP (RAMP- Rational Approximation of Material Properties, Stolpe und Svanberg 2001) also interpolates the Young's modulus and is defined as follow: 

\begin{equation}
E_e = \dfrac{x_e}{1+p \;(1-x_e)} \; E_{0e} \quad ,
\label{math_equation}
\end{equation}

where\textbf{ $x$} is the design variable, in the case of this work it is the density $\rho $, \textbf{$p$} stands for the penalization exponent, \textbf{$E_{0e}$} is the given Young's modulus for the considered element\textsubscript{e} and\textbf{ $E_e$} is the new calculated Young's modulus with the usage of RAMP (see figure \ref{fig_RAMP_sceme}). In general RAMP is explicit restricted to convex material interpolation, for a more detail introduction please be referred to \cite{Stolpe.2001}.\\


%TikZ Grafik
\begin{figure}[!h]
\centering
\inputTikZ{1}{path_Image/RAMP.tikz}
\caption{RAMP material-interpolation-scheme.}
\label{fig_RAMP_sceme}
\end{figure}
%

\section{Objective Function by using SIMP}
\label{subsection_obj_SIMP}

Each element is assigned a density \textbf{$\rho_{e}$ } that determines its Young's modulus. Since there are two known approaches of the SIMP, the objective function, where the objective is to minimize compliance, can be written in two ways.\\

\cite{Sigmund.2001} defines the objective function by using the SIMP defined by \cite{Bendse.1988} as:

\begin{align}
\underset{x}{min} : c(x) = U^TKU =& \sum_{e = 1}^N (x_e)^pu_e^Tk_0u_e
\label{equ_objective_func_99}
\\ \nonumber \\
\nonumber
\text{subject to} \quad & \frac{V(x)}{V_0} = f \\
\nonumber
& KU = F \\
\nonumber
&0<x_{min} \leq x \leq1
\end{align}
\cite{Andreassen.2011} defines the objective function by using the modified SIMP created by \cite{Sigmund.2007} as:

\begin{align}
\underset{x}{min} = c (x) = U^TKU = \displaystyle &\sum_{e = 1}^N E_e(x_e)u_e^Tk_0u_e 
\label{equ_objective_function_88} \\
\nonumber \\ \nonumber 
 \text{subject to}: \quad &\frac{V(x)}{V_0} = f \\\nonumber
& KU = F\\\nonumber
& 0\leq x_{min} \leq1, \nonumber
\end{align}

%\begin{align*}
%\underset{x}{min} = c (x) = U^TKU = & \sum_{e = 1}^N E_e(x_e)u_e^Tk_0u_e \\\\
% \text{subject to}: \quad &\frac{V(x)}{V_0} = f \\
%& KU = F\\
%& 0\leq x_{min} \leq1
%\end{align*}

% $$min_x: c({x}) = U^TKU = \sum_{e = 1}^N E_e(x_e)u_e^Tk_0u_e$$
%subject to: $$ \frac{V(x)}{V_0} = f$$
%$$:KU = F$$
%$$:0\leq x_{min} \leq1$$

where \textbf{$c$} is the compliance, \textbf{$U$} 
and \textbf{$F$} are the global displacement- and force-vector, \textbf{K} is the global stiffness matrix, \textbf{$u_e$} 
is the displacement vector for the considered element,
 \textbf{$k_0$} is the stiffness matrix for the considered
  element with unit Young's modulus,
   \textbf{$x$} is the vector of the design variables, in case of this work
   the design variable is the density, \textbf{$N (nelx*nely)$} is the number of elements used to discretize the available construction space, where \textbf{$nelx$} and \textbf{$nely$} are the number of elements in horizontal \textbf{(x)} and in vertical \textbf{(y)} direction. \textbf{$V(x)$} and \textbf{$V_0$} is the 
   actual material volume and available construction space volume, respectively and \textbf{$f(volfrac)$} is the predefined volume fraction.
%unit bedeutet Einheit - kg and m
Furthermore, \textbf{$p$} is the penalization exponent and \textbf{$E_e(x_e)$} is given as in the equation \ref{eq_simp_1}


 
\section{FEA - Finite Element Analysis}
\label{section_FEA}
Over the time the Finite Element Method (FEM) has proved to be a reliable method for obtaining informations about mechanical structures. And because it not always possible to collect information about mechanical structures by means of analytical methods, it is suggested using the numerically FEA (Finite Element Analysis). The basic progress of a FEA can be described as follow: 

\begin{enumerate}
\item{ Considering a (complex) problem},
\item{ Breaking it into small pieces (a \textbf{finite} number of \textbf{elements})},
\item{ Simplifying  each piece (simple relationships)}, 
\item {Re-assembling the pieces (matrix equations)},
\item {Solving the problem (matrix manipulation).}\\
\end{enumerate}

By talking about a (complex) problem, it is meant to choose any mechanical structure, then create a model (1D, 2D, 3D) of the desired structure, define material properties like (\textbf{$\rho$}) the densitiy, mass, the Young's modulus \textbf{$E$} and shear modulus \textbf{$G$}.\\
The second step from the list is, to break the complex structure
 into small pieces, which can be expressed as \textbf{discretization}
 (see figures \ref{fig_2d_discr} \ref{fig_3d_discr}). The pieces are called \textbf{elements}, which can have any geometrical shape.
 In most cases in 2D, triangles or squares, respectively in 3D tetrahedra or hexaeder, are used.
 In figure \ref{fig_2d_discr} a object is discretized sqared with green nodes is presented. The pieces respectively the elements can be find in the figure \ref{fig_2d_discr} as integer numbers. \\

\begin{figure} [!h]
 \centering
 \def\svgwidth{\textwidth}
 \input{path_Image/2d_discrezisierung.pdf_tex}
 \caption{2D squared discretized object with green nodes} % Bildunterschrift 
 \label{fig_2d_discr}   % Label für Verweise 
\end{figure} 

 However, in this thesis the elements are discretized squared, or in 3D
 with hexaeder (voxels). By increasing the number of elements, the results
 converges to the reality, but as a drawback the computations-time and the storage place increases. Note, the number of the \textbf{elements} is \textbf{finite}. Discretization does not only come along with elements, but also with \textbf{nodes}, see figure \ref{fig_2d_discr} and figure \ref{fig_3d_discr}. 
 
 \begin{figure} [!h]
 \centering
 \def\svgwidth{\textwidth}
 \input{path_Image/3d_discrezisierung.pdf_tex}
 \caption{3D squared discretized (voxels) object with green nodes} % Bildunterschrift 
 \label{fig_3d_discr}   % Label für Verweise 
\end{figure} 

These nodes are points where elements attach or meet each other. Furthermore the nodes hold loads and constraints. Again, in FEM the loads and constraints are not applied on elements, rather they are applied on nodes. In 2D by means of squared discretziation each discretized element has 4 nodes, respectively in 3D by means of voxel discretied elements each element has 8 Nodes. In 2D there are two possible
main translatory axes to apply a load on it. To choose a load
axes, the so called 'Degree Of Freedom' (DOF) is introduced. \\

This work uses the FEA for stress analysis, therefore the DOF can be interpreted as a displacement of a node that is resisted by the
attached element(s). More general, a DOF is some property of a node, e.g. temperature or charge that relates to the attached element stiffness matrix.\\

Each node has 2 translatory DOFs, one 
for the horizontally (x) axis and one for the vertically (y) axis.
 In 3D there are 3 available translatory 
 axes and therefore each Node has 3 DOFs, in x,y, and z direction. In the figure \ref{fig_2d_one_ele} a graphical explanation is appended. The figure \ref{fig_2d_one_ele} only shows one discretized element, the green dots are the nodes, the turquoise arrows stand for the displacement in the horizontally (x) direction and the yellow arrows are representative for the displacement in the vertically (y) direction. Each Node has a x- and y-displacement. Anlaogusly to 2D, in 3D each Node has 3 displacement axes (x,y,z).

\begin{figure} [!h]
\begin{minipage}{0.5\textwidth}
 \centering
 \def\svgwidth{\textwidth}
 \input{path_Image/Element_2D.pdf_tex}
 \caption{One 2D squared discretized element with 4 Nodes and DOFs.} % Bildunterschrift 
 \label{fig_2d_one_ele}   % Label für Verweise 
\end{minipage}
\hfill
\begin{minipage}{0.4\textwidth}
In the third step, each element is considered and described. In this work a
 a linear relationship between stress and strain or between force and displacement can be seen. The relation between all the elements need to be defined by e.g. using compatibility or continuity equations. With that having said, the relations are going to be linked together and this will result in a large system of equations, which are assembled then into a matrix form. In the case of this thesis, the matrix has the form of: 
$$\mathbf{F = KU}$$
Where $\textbf{F}$ is the force-vector, $\textbf{K}$ is the stiffness-matrix and $\textbf{U}$ is the displacement vector.\\
\end{minipage}
\end{figure} 

The force vector $\textbf{F}$ is also known as the nodal load vector, it is a vector containing all the component loads at the nodes. The displacement vector $\textbf{U}$, or the nodal displacement vector is the vector which contains all the DOFs of a model.
The last step is to solve the matrix system. In most cases the stiffness is known, because it is a material property and since the external forces are applied by the user they are also known.
The only unknown is the displacement vector. In order to solve the problem, the inverse of the stiffness matrix \textbf{$K^{-1}$} is required. The stiffness matrix is always square ($n \times n$).\\

Regarding general knowledge about stiffness matrices, a explanation of a singular matrix may be helpful.
A matrix $A$, which cannot be inverted
(\textbf{$A^{-1} $ is not possible}), because
the determinant is $0$ and therefore \textbf{$A$} has no unique solution to the system of equations. This usually happens, when a the system is violated though incorrect or incomplete constraints.

\subsection{What kind of problems can FEA solve}
FEA can solve any kind of Boundary Value Problems (BVP). BVPs are mathematical problems in which the quantity of interest is defined by a differential equation throughout a region, with initial values only known in specific areas (the region's boundaries). Some common BVP's in Engineering:
\begin{itemize}
\item{Stress Analysis},
\item{ Heat Transfer},
\item{ Fluid Flow}, 
\item {Electric or Magnetic Potential.}
\end{itemize}
 
 \subsection{3D finite element analysis}
 In this section, an explanation of the 3D-FEA-task for solving the objective function (minimizing the the compliance), which is mostly based on \cite{Liu.2014}, is provided. Following \\
 the modified SIMP (equation \ref{eq_simp_1}) and Hooke's law for (equation \ref{equ_hook_1} ) the three-dimensional constitutive matrix for an \textbf{isotropic} (equal material properties for every direction) element \textbf{$e$} is interpolated from void ($\rho = 0$) to solid ($\rho = 1$) as:
 
 \begin{align} 
 C_e(x_e) = C _e(x_e)C_e^0 ,\quad x_e \in [0,1]
\end{align} 
where \textbf{$C_e^0$} is the constitutive matrix with unit Young's modulus.
The constitutive matrix can be found in the equation \eqref{equ_constit}.
%unit = Einheit (kg)
 \begin{align} 
 E_e(x_e) = E_{min}+ x_e^p (E_0-E_{min}), \quad x_e \in [0,1]
 \tag{\ref{eq_simp_1}} 
\end{align}
\begin{align} 
\sigma = E \varepsilon
\label{equ_hook_1}
\end{align}
where \textbf{$\sigma$} is denoted as stress, \textbf{$E$} is the Young's modulus, \textbf{$\varepsilon$} is the strain.\\

\begin{align}
C_e^0 = \frac{1}{(1+\nu)(1-\nu)}
\begin{split}
\begin{bmatrix}
1-\nu & \nu & \nu & 0 & 0 & 0 \\ 
\nu & 1-\nu & \nu & 0 & 0 & 0 \\ 
\nu & \nu & 1-\nu & 0 & 0 & 0 \\ 
0 & 0 & 0 & (1-2\nu)/2 & 0 & 0 \\ 
0 & 0 & 0 & 0 & (1-2\nu)/2 & 0 \\ 
0 & 0 & 0 & 0 & 0 & (1-2\nu)/2\\
\end{bmatrix} 
\end{split},
\label{equ_constit}
\end{align}
where \textbf{$\nu$} is the Poisson's ratio of the \textbf{isotropic} material.\\

\begin{figure}[!h]
\begin{minipage}{0.45\textwidth}
By applying the finite element method, the elastic solid element stiffness matrix, can be obtained as the volume integral of the elements constitutive matrix \textbf{$C_e(x_e)$} and of the strain-displacement matrix \textbf{$B$} (see equation \ref{equ_strain_displace_B_matrix}) in the form of:\\

\begin{align*}
k(x_e) = \int_{-1}^{+1} \int_{-1}^{+1}\int_{-1}^{+1} B^T C_e(x_e) B d  \xi_1 d \xi_2 d \xi_3 
\end{align*}

\begin{align*}
k(x_e) = \int_{-1}^{+1} \int_{-1}^{+1}\int_{-1}^{+1} B^TC_e \quad , 
\end{align*}
where $\xi_i$ (i = 1,2,3) are the natural coordinates as shown in Fig. \ref{fig_3d_integral}, and the cube coordinates of the corners are
shown in Table
\ref{tab_cube}. The strain-displacement
 matrix \textbf{$B$} related the strain \textbf{$\varepsilon$} and the nodal displacement \textbf{$u$}, \textbf{$\varepsilon = Bu$}. Using the SIMP method, the element stiffness matrix is interpolated as:\\

\begin{align}
k_e(x_e) = E_e(x_e)k_e^0 \quad , 
\end{align} 
where
\begin{align}
k_e^0 = \int_{-1}^{+1}\int_{-1}^{+1}\int_{-1}^{+1}B^TC^0Bd\xi_1 d \xi_2 d \xi_3
\label{equ_stiffnes_matrix_2}
\end{align}
\end{minipage}
\hfill
\begin{minipage}{0.4\textwidth}
 \centering
 \def\svgwidth{\textwidth}
 \input{path_Image/Element_3D_integral.pdf_tex}
 \caption{Green dots as nodes and the natrual coordinates $\xi_1,\xi_2,\xi_3$.} 
 \label{fig_3d_integral}   % Label für Verweise 
 \vspace{0.2cm}
\captionof{table}[]{The eight-node hexahedral (cubic) with node numbering convention \cite{Andreassen.2011}}
\begin{center}
\begin{tabular}{|c|c|c|c|}
\hline 
Node & $\xi_1$ & $\xi_2$ & $\xi_3$ \\ 
\hline 
1 & -1 & -1 & -1 \\ 
\hline 
2 & +1 & -1 & -1 \\ 
\hline 
3 & +1 & +1 & -1 \\ 
\hline 
4 & -1 & +1 & -1 \\ 
\hline 
5 & -1 & -1 & +1 \\ 
\hline 
6 & +1 & -1 & +1 \\ 
\hline 
7 & +1 & +1 & +1 \\ 
\hline 
8 & -1 & +1 & +1 \\ 
\hline 
\end{tabular} 
\label{tab_cube}
\end{center}

\end{minipage}
\end{figure}

Replacing values in \eqref{equ_stiffnes_matrix_2}, the $24 \times 24$ element stiffness matrix \textbf{$k_e^0$} for an eight-node hexahedral respectively a cubic element is:
\begingroup
\renewcommand*{\arraystretch}{1.6}
\begin{align}
k_e^0 = \frac{1}{(\nu +1)(1-\nu)}
\begin{bmatrix}
k_1 & k_2 & k_3 & k_4 \\ 
k_2^T & k_5 & k_6 & k_4^T \\ 
k_3^T & k_6^T & k_5^T & k_2^T \\ 
k_4 & k_3 & k_2 & k_1^T
\end{bmatrix} ,
\end{align}
\endgroup
where $k_m$ (m = 1,...6) are $6 \times 6$ symmetric matrices and it can be noted that \textbf{$k_e^0$} is a positive definite matrix.\\


\begingroup
\renewcommand*{\arraystretch}{1.2}
\begin{align*}
k_1 =& \begin{bmatrix}
k_1 & k_2 & k_2 & k_3 & k_5 & k_5 \\ 
k_2 & k_1 & k_2 & k_4 & k_6 & k_7 \\ 
k_2 & k_2 & k_1 & k_4 & k_7 & k_6 \\ 
k_3 & k_3 & k_4 & k_1 & k_8 & k_8 \\ 
k_5 & k_6 & k_7 & k_8 & k_1 & k_2 \\ 
k_5 & k_7 & k_6 & k_8 & k_2 & k_1
\end{bmatrix}
&k_ 2 =& \begin{bmatrix}
k_9 & k_8 & k_{12} & k_6 & k_4 & k_7 \\ 
k_8 & k_9 & k_{12} & k_5 & k_3 & k_5 \\ 
k_{10} & k_{10} & k_{13} & k_7 & k_4 & k_6 \\ 
k_6 & k_5 & k_{11} & k_9 & k_2 & k_{10} \\ 
k_4 & k_3 & k_5 & k_2 & k_9 & k_{12} \\ 
k_{11} & k_4 & k_6 & k_{12} & k_{10} & k_{13}
\end{bmatrix}\\ \nonumber \\
k_3 =& \begin{bmatrix}
k_6 & k_7 & k_4 & k_9 & k_{12} & k_8 \\ 
k_7 & k_6 & k_4 & k_{10} & k_{13} & k_{10} \\ 
k_5 & k_5 & k_3 & k_8 & k_{12} & k_9 \\ 
k_9 & k_{10} & k_2 & k_6 & k_{11} & k_5 \\ 
k_{12} & k_{13} & k_{10} & k_{11} & k_6 & k_4 \\ 
k_2 & k_{12} & k_9 & k_4 & k_5 & k_3
\end{bmatrix}
&k_4 =& \begin{bmatrix}
k_{14} &k_{ 11} & k_{11} & k_{13} & k_{10} & k_{10} \\ 
k_{11} & k_{14} & k_{11} & k_{12} & k_9 & k_8 \\ 
k_{11}& k_{11} & k_{14} & k_{12} & k_8 & k_9 \\ 
k_{13} & k_{12} & k_{12} & k_{14} & k_7 & k_7 \\ 
k_{10} & k_9 & k_8 & k_7 & k_ {14} & k_{11} \\ 
k_{10}& k_8 & k_9 & k_7 & k_{11} & k_{14}
\end{bmatrix}\\ \nonumber \\
k_5 =& \begin{bmatrix}
k_1 & k_2 & k_8 & k_3 & k_5 & k_4 \\ 
k_2 & k_1 & k_8 & k_4 & k_6 & k_{11} \\ 
k_8 & k_8 & k_1 & k_5 & k_{11} & k_6 \\ 
k_3 & k_4 & k_5 & k_1 & k_8 & k_2 \\ 
k_5 & k_6 & k_{11} & k_8 & k_1 & k_8 \\ 
k_4 & k_{11} & k_6 & k_2 & k_8 & k_1
\end{bmatrix}
&k_6 =& \begin{bmatrix}
k_{14} & k_{11} & k_7 & k_{13} & k_{10} & k_{12} \\ 
k_{11} & k_{14} & k_7 & k_{12} & k_9 & k_2 \\ 
k_7 & k_7 & k_{14} & k_{10} & k_2 & k_9 \\ 
k_{13} & k_{12} & k_{10} & k_{14} & k_7 & k_{11} \\ 
k_{10} & k_9 & k_2 & k_7 & k_{14} & k_7 \\ 
k_{12} & k_2 & k_9 & k_{11} & k_7 & k_{14}
\end{bmatrix} 
\end{align*}
\endgroup
 
and
\begin{align*}
&k_1 = -(6\nu-4)/9 \quad && k_2 = 1/12\\
&k_3 = -1/9 \quad && k_4 = -(4\nu-1)/12\\
&k_5 = (4\nu-1)/12 \quad && k_6 = 1/18\\
&k_7 = 1/24 \quad && k_8 = -1/12\\
&k_9 = (6\nu-5)/36 \quad &&k_{10} = -(4\nu - 1)/24\\
&k_{11} = -1/24 \quad &&k_{12} = (4\nu -1)/24\\
&k_{13} = (3\nu -1)/18 \quad &&k_{14} = (3\nu -2)/18\\
\end{align*}

The global stiffness matrix \textbf{$K$} is obtained by the assembly of element-level counterparts $k_e$ \cite{Liu.2014}.
$$K(x) = A_{e=1}^nk_e(x_e) = A_{e=1}^nE_e(x_e)k_e^0 \quad,$$
where n is the total number of elements. Using the global versions of the element stiffness matrices \textbf{$K_e$} and \textbf{$K_e^0$}, is expressed as:

$$\sum_{e=1}^n \; K_e(x_e) = \sum_{e=1}^nE_e(x_e)K_e^0 \quad,$$
where \textbf{$K_e^0$} is a constant matrix. Using the interpolation function in \eqref{eq_simp_1}, it can be observed that
$$K(x) = \sum_{e=1}^n  \; [E_{min}+x_e^p(E_0 -E_{min}] \; K_e^0$$
Finally, the nodal displacement vector \textbf{$U(x)$} is the solution of the equilibrium equation \cite{Liu.2014}\\
$$K(x)\;U(x) = F \quad,$$

where \textbf{$F$} is the vector of the nodal forces.\\

%strain ist die Dehung
The strain-displacement matrix \textbf{$B$} is defined for an eight-node hexahedal element by:

\begingroup
\renewcommand*{\arraystretch}{1.6}
\begin{align}
B =\begin{bmatrix}
\dfrac{\partial n_1(\xi_i)}{\partial\xi_1} & 0 & 0 & ... & \dfrac{\partial n_q(\xi_i)}{\partial\xi_1} & 0 & 0 \\ 
0 & \dfrac{\partial n_1(\xi_i)}{\partial\xi_2} & 0 & ... & 0 & \dfrac{\partial n_q(\xi_i)}{\partial\xi_2} & 0 \\ 
0 & 0 & \dfrac{\partial n_1(\xi_i)}{\partial\xi_3} & ... & 0 & 0 & \dfrac{\partial n_1(\xi_i)}{\partial\xi_3} \\ 
\dfrac{\partial n_1(\xi_i)}{\partial\xi_2} & \dfrac{\partial n_1(\xi_i)}{\partial\xi_1} & 0 & ... & \dfrac{\partial n_q(\xi_i)}{\partial\xi_2} & \dfrac{\partial n_q(\xi_i)}{\partial\xi_1} & 0 \\ 
0 & \dfrac{\partial n_1(\xi_i)}{\partial\xi_3} & \dfrac{\partial n_1(\xi_i)}{\partial\xi_2} & ... & 0 & \dfrac{\partial n_q(\xi_i)}{\partial\xi_3} & \dfrac{\partial n_q(\xi_i)}{\partial\xi_2} \\ 
\dfrac{\partial n_1(\xi_i)}{\partial\xi_3} & 0 & \dfrac{\partial n_1(\xi_i)}{\partial\xi_1} & ... & \dfrac{\partial n_q(\xi_i)}{\partial\xi_3} & 0 & \dfrac{\partial n_q(\xi_i)}{\partial\xi_1} 
\end{bmatrix} 
\label{equ_strain_displace_B_matrix}
\end{align}
\endgroup

for \textbf{$i = $}1,2,3 and q = 1,...8. The corresponding shape functions \textbf{$n_q$} in a natural coordinate systems $\xi_i$ are defined by
\begin{align*}
n_q(\xi_i) = \dfrac{1}{8}
\begin{bmatrix}
(1-\xi_1)(1-\xi_2)(1-\xi_3) \\ 
(1+\xi_1)(1-\xi_2)(1-\xi_3) \\ 
(1+\xi_1)(1+\xi_2)(1-\xi_3) \\ 
(1-\xi_1)(1+\xi_2)(1-\xi_3) \\ 
(1-\xi_1)(1-\xi_2)(1+\xi_3) \\ 
(1+\xi_1)(1-\xi_2)(1+\xi_3) \\ 
(1+\xi_1)(1+\xi_2)(1+\xi_3) \\ 
(1-\xi_1)(1+\xi_2)(1+\xi_3)
\end{bmatrix} 
\end{align*}

\section{Optimality Criteria}
\label{section_OC}
In order to solve the optimisation problem:
\begin{align}
\underset{x}{min} = c (x) = U^TKU = \displaystyle &\sum_{e = 1}^N E_e(x_e)u_e^Tk_0u_e 
\tag{\ref{equ_objective_function_88}}\\
\nonumber \\ \nonumber 
 \text{subject to}: \quad &\frac{V(x)}{V_0} = f \\\nonumber
& KU = F\\\nonumber
& 0\leq x_{min} \leq1, \nonumber
\end{align}
some approaches, e.g. Optimality Criteria (OC), Sequential Linear Programming (SLP) (Wilson 1963) or the Method of Moving Asymptotes(MMA) (Svanberg 1987) and other could be employed.
 In this bachelor thesis not all mentioned methods are
  going to be explained, but some further readings will be offered. An
   explanation as well a numerically implementation can be found in \cite{Liu.2014} for all the
   mentioned solving-approaches.

Due to its simplicity and its numerical efficiency, this work 
uses the OC \cite{Bendse.2004} as the solver. 
The updating scheme for the density \textbf{$\rho$} in OC is described as:

\[ \rho_e^{new} =
 \begin{cases}
 max (0, \rho_e - m) & \quad \text{if } \rho_e B_e^{\eta} \leq max(0, \rho_e - m), \\
 min (0, \rho_e \geq m) & \quad \text{if } \rho_e B_e^\eta \leq max(0, \rho_e - m), \\
 \rho_e B_e^{\eta}&\quad \text{otherwise,}
 \end{cases}
\]
where \textbf{$m$ }is a positive move-limit, and \textbf{$\eta$} is a numerical damping coefficient. The choice of $m = 0.2$ and $\eta = 0.5$ is a recommendation for minimum compliance problems \cite{Bendse.2004} in order not to face numerically instabilities. \textbf{$B_e$} is found from the optimality condition as :

\begin{align}
B_e = \dfrac{\dfrac{-\partial c}{\partial \rho_e}}{\displaystyle\lambda \dfrac{\partial V}{\partial\rho_e}} \quad,
\label{equ_opti_condition}
\end{align}

where \textbf{$\lambda$} is the Lagrangian multiplier that can be found by a b-sectioning algorithm. \\
The sensitivies of the objective function \textbf{$c$} and the material volume \textbf{$V$} with respect to the element densities $\rho_e$, which is also the design variable \textbf{$x_e$} and are given by:
\begin{align}
\dfrac{\partial c}{\partial \rho_e} =& -p x_e^{p-1} (E_0-E_{min})u_e^Tk_0u\\ \nonumber\\
\dfrac{\partial V}{\partial \rho_e} =& 1
\end{align}

\subsection{Filtering}
In order to ensure the existence of solutions to the topology optimisation problem and furthermore
to avoid the appearance of \textbf{checkerboard patters
 (elements are connected at their edges)} (see figure \ref{fig_checkerboard}), some restrictions on the design must
  be imposed. A common approach is the application of 
  a filter on the sensitivity or on the densities. In \cite{Sigmund.2007} and in \cite{Andreassen.2011} some filters were presented
   with explanations and also results. Very popular filters
    are, e.g. the standard convolution-filter, which is implemented in the 99-lines of code \cite{Sigmund.2001}, the Heaviside projection
     filter, numerically implemented in \cite{Andreassen.2011}, the PDE (Partial Differentail Equation)-filter by\cite{Lazarov.2011}, 
which is worth to be mentioned because of its 
parallelize competence . The element by element filter (ebef) is also also very suitable for parallelization. 
It operates matrix free, which means the filter does not require additionally memory space, and a Pseudo-implementation
 code for the GPU can be found in \cite{MartinezFrutos.2017}.\\
 	
 \begin{figure}[!h]
 \begin{minipage}{0.6 \textwidth}
   However, the main reasons to use a filter is, to avoid mesh 
 dependence, which can declared as obtaining varying optimal solutions by varying mesh resolutions. The second main reason to deploy a filter is to avoid the checkerboard effect (see figure \ref{fig_checkerboard}). 
 Furthermore, by means of the common filter from 99-lines of code \cite{Sigmund.2001} it is possible to 
 control the minimum 
 thickness of topology structure components, 
 which is also known as \textbf{minimum member size control}. 
 The components will consist of 
 at least \textbf{$r_{min}$} connected elements, where \textbf{$r_{min}$} is the filter radius, otherwise these elements will become void. The 
 figure \ref{fig_rmin} shows the filter with a chosen $r_{min}$ and
 and a considered element called \textit{Center\textsubscript{e}}.
 All the green elements are going to be considered as neighbour elements of the $center_e$ element. In order to be considered as a neighbour element
 the distance between the \textbf{center} of $center_e$
 and the \textbf{center} other elements need to 
 be $dist(center_e, neigbourelements) \leq r_{min}$. The red marked elements are 
 not treated as neighbour elements. \\
 \end{minipage}
 \hfill
 \begin{minipage}{0.35 \textwidth}
 	\centering
 \def\svgwidth{\textwidth}
 \input{path_Image/filter_radius_rmin_1.pdf_tex} 
 \caption{ Convolution filter as a sensitivity filter.} % Bildunterschrift 
 \label{fig_rmin}   % Label für Verweise 
 \end{minipage}
  \end{figure}

 \begin{figure}[!h]
	\centering
 \includegraphics[width=\textwidth, height=0.45\textheight]{path_Image/pngs/checkerboard.png}
	\caption{An example for a checkerboard pattern caused by an inactive filter ($r_{min} = 1)$} or by the lack of a filter.
	\label{fig_checkerboard}
	\end{figure}
	\vspace{0.7 cm}

 
 In this work the modified filter from \cite{Andreassen.2011}, which is based on \cite{Sigmund.2001} and is modified by one single term.
\subsection{99 lines of code filter} 
\label{subsection_99_lines_of_code}
 The unmodified sensitivity filter from \cite{Sigmund.2001} can be expressed as:
 
 \begin{align}
 \dfrac{\partial c}{\rho_e} = \dfrac{1}{\rho_e \displaystyle\sum_{f=1}^{N}H_f}\sum_{f=1}^{N}H_f \rho_f\dfrac{\partial c}{\partial \rho_f}
 \label{equ_sigmund_99_SAfilter}
 \end{align}
 
The convolution operator (weight factor ) \textbf{$H_f$} is defined as:
\begin{align}
&H_f = r_{min} - dist(e,f) 
\label{equ_convultion_99} \\ \nonumber
&f \in \mathbb{N} \vert dist(e,f) \leq r_{min},\quad r= 1,...\mathbb{N} \quad ,
\end{align}
where the operator dist(e,f) represents the distance between the center of the considered element \textbf{$e$} and center of the surrounding element \textbf{$f$}. The convolution operator \textbf{$H_f$} decays linearly with he distance from the distance from element \textbf{$f$}. Instead of the original sensitivities, 
\begin{align}
\dfrac{\partial c}{\partial \rho_e} =& -p \;(x_e)  ^{p-1} \;u_e^Tk_0u_e
\label{equ_sa_99_lines}
\end{align}
the modified sensitivies \eqref{equ_sigmund_99_SAfilter} are used in the Optimality Criteria update:
\begin{align}
B_e = \dfrac{\dfrac{-\partial c}{\partial \rho_e}}{\displaystyle\lambda \dfrac{\partial V}{\partial\rho_e}}
\tag{\ref{equ_opti_condition}}
\end{align}

The equation \eqref{equ_sa_99_lines} is commonly known as the Sensitivity Analysis (SA) and represents only the SA of the 99 lines of code \cite{Sigmund.2001}. Note that the SA in \cite{Sigmund.2001}\eqref{equ_sa_99_lines} and the sensitivity of the objective function \cite{Andreassen.2011} \eqref{equ_sa_88} are \textbf{not} the same. \cite{Andreassen.2011} uses the modified SIMP and \cite{Sigmund.2001} uses the unmodified SIMP (for more Information see \ref{subsection_SIMP}).

 \subsection{88 lines of code filter} 
 As mentioned, the 88 lines of code sensitivity-filter is almost identical to the filter of \cite{Sigmund.2001}. The 88 lines of code filter by \cite{Andreassen.2011} is stated as:

 \begin{align}
 \dfrac{\partial c} {\rho_e} = \dfrac{1}{\displaystyle max(\rho_e,\gamma)\displaystyle\sum_{f=1}^{N}H_f}\sum_{f=1}^{N} H_f \rho_f \dfrac{\partial c}{\partial \rho_f}
 \label{equ_Ander_88_SAfilter}
 \end{align}
 The convolution operator (weight factor) \textbf{$H_f$} is defined as (same as in equation \ref{equ_convultion_99})
However, the new term \textbf{$\gamma = 10^{-3}$} is a small positive number, introduced in order not dividing by zero. This difference comes along though the deployment of the modified SIMP version \cite{Sigmund.2007}.
And analogously to subsection \ref{subsection_99_lines_of_code}, the convolution operator \textbf{$H_f$} decays linearly with he distance from the distance from element \textbf{$f$}. Instead of the original sensitivities,

\begin{align}
\dfrac{\partial c}{\partial \rho_e} =& -p x_e^{p-1} (E_0-E_{min})u_e^Tk_0u
\label{equ_sa_88}
\end{align}

the modified sensitivities \eqref{equ_Ander_88_SAfilter} are used in the Optimality Criteria update
\begin{align}
B_e = \dfrac{\dfrac{-\partial c}{\partial \rho_e}}{\displaystyle\lambda \dfrac{\partial V}{\partial\rho_e}}
\tag{\ref{equ_opti_condition}}
\end{align}
 
 Analogously to equation \eqref{equ_sa_99_lines}, the equation \ref{equ_sa_88}
  acts as the Sensitivity Analysis (SA) regarding the objective function \eqref{equ_objective_function_88}.
 
	
% \begin{figure} [!h]
% \centering
% \def\svgwidth{\textwidth}
% \input{path_Image/88_lines_visio.pdf_tex}
% \caption{Diffent types of structual optimisation} % Bildunterschrift 
% \label{fig_struc_arten}   % Label für Verweise 
%\end{figure} 
 

\chapter{State of art}
Many researches
in the area of optimization with compliance
as objective function has been done. 
 The first worthy contribution, which nowadays can bee seen as a pioneer contribution is \cite{Sigmund.2001}. Sigmunds 99 lines of code is often the basis of a more advanced topology optimisation code.\cite{Andreassen.2011} used Sigmund's 99 lines of code and extended it with some new filters with the modified SIMP by \cite{Sigmund.2007}, like the PDE-filter (Partial differential equation filter) by\cite{Lazarov.2011}. The greatest benefit of the 88 lines of code is that it is able to provide, the remarkable raise of computational time, for a
 comparison see \cite{Andreassen.2011}. \cite{Amir.2014} presents a 2D-matlab-code, which offers a iterative FE solver (conjugated gradient) and a improved  precondition, in order to solve the equation-system fast and 
 with low residuals. \cite{Duarte.2015} implemented 
 the MGCG (MultiGrid Preconditioned Conjugate Gradient)-FE solver 
 and also the\cite{Lazarov.2011} PDE-Filter (Partial differential equation filter) and with the usage of PETSC (Portable, Extensible Toolkit Scientific Computation) they provide a fully parallel topology optimiser. Polytop++ in 
 \cite{Duarte.2015} also has been improved further in \cite{Aage.2015}. Because the research of the topology optimisation has the possibly to advance from parallelization, there also some optimiser introduced, which run on the GPU (Graphic Processing Unit). The best known
 contribution for GPU-topology optimisation might be the 2589 lines
 of code in\cite{Schmidt.2011}.\\

However, this work will take the 88 lines of code in \cite{Andreassen.2011} as its basis. In the following a brief introduction about the 99- and 88-lines of code as well as into the 169 lines of code by \cite{Liu.2014} is provided.

\section{99 and 88 lines of code overview}
The 88 line of code \cite{Andreassen.2011} is based on the 99 lines of code \cite{Sigmund.2001}. 
The goal of the 88 lines of code is to solve the objective function:
\begin{align}
\underset{x}{min} = c \;(x) =  U^TKU = & \sum_{e = 1}^N E_e(x_e)u_e^Tk_0u_e 
\tag{\ref{equ_objective_function_88}}
\\ \nonumber \\ \nonumber
 \text{subject to}: \quad &\frac{V(x)}{V_0} = f \\ \nonumber
& KU = F\\ \nonumber
& 0\leq x_{min} \leq 1
\end{align}
and the equation of the objective function of the 99 lines of code can be found in \eqref{equ_objective_func_99}. The difference between both versions has been introduced in the subsection \ref{subsection_obj_SIMP} and also in the section
\ref{section_OC}.
The 99 lines of code, the 88 lines of code and also the 3D version of the 88 lines by \cite{Liu.2014} obey the progress shown in the flowchart \ref{fig_flowchart_88}.

 \begin{figure} [!h]
 \centering
 \def\svgwidth{\textwidth}
 \input{path_Image/88_lines_visio.pdf_tex}
 \caption{Flowchart of topology optimisation using 88, 99 and the 169 (3D) lines of code. Note, that the 169 lines of code also requires \textbf{nelz}.}  
 \label{fig_flowchart_88}          % Label für Verweise 
\end{figure} 
 
% 
\newpage
\section{Preconditions}
The predefinitions have the purpose to initialise the optimizer, they are constituted of the following parameters: \textbf{nelx, nely, penal, r\textsubscript{min}, load-case}, \textbf{volfrac} and in case of the 3D code the additional parameter \textbf{nelz}, where \textbf{nelx} is denoted as the number of horizontally (\textbf{x}) discritized number of elements, respectively nely is the number of the vertically (\textbf{y}) discritized number of elements, \textbf{nelz} is the number of the elements, which are discrezied in the depth (\textbf{z}), \textbf{penal}
is the the penalization exponent, \textbf{r\textsubscript{min}} determines the radius of the sensitivity filter (see figure \ref{fig_rmin}) and \textbf{volfrac} is the volume fraction.

\subsection{Numbering elements, nodes and DOFs in 2D}
Numerous simplifications are introduced in order to make the 2D Matlab code simple.
First, the considered design domain is assumed to be square. 
Therefore the domain can be discretized by square finite elements and the numbering of elements and nodes becomes easier. The numbering starts from the the upper left corner, it increases from up to down and afterwards from left to right.
The upper left corner is located in the first row and the first column. The upper right corner is located in first the row
and in the last column, the lower left corner can be found at the last  row and the first column and finally the lower corner can be find at the last row and the last column.\\
%______________________________________________________________

\begin{minipage}{0.5\textwidth}
\captionof{table}[]{Element numbering in 2D, also compare with figure \ref{fig_2d_matlab}}
\begin{center}
\begin{tabular}{|l|l|}
\hline 
Left upper corner& [ r = 1, c = 1 ];\\
\hline 
 Right upper corner& [ r = 1, c = last ];\\
\hline 
Left lower corner& [ r = last, c = 1 ];\\
\hline 
 Right lower corner& [ r = last, c = last ];\\
\hline 
\end{tabular} 
\end{center}
\end{minipage}
\begin{minipage}[t]{0.4\textwidth}
 After reaching the lower left, the counting continues by jumping to the next column respectively moving along the x-axis.
 The number of elements in x-direction (horizontal) is denoted as nelx, which stands for "number elements x" and the number of elements
 in y direction (vertical) are denoted as nely, which analogously stands for "number elements y".\\
\end{minipage}
This work uses Matlab as the the coding language and its Editor. Matlab offers two ways for indexing a matrix respectively a vector. The linear indexing is identical to the element indexing in figure \ref{fig_element_nr_2D}, e.g. with the expression \textbf{A(5)}, where\textbf{ A} is matrix, Matlab is going to extract the 5 element inside the matrix. However, working with big meshes results is making the use of linear indexing unhandy, therefore this work is mostly going to use the subscript-numbering. The subscript takes two input parameter, the first is the number of the desired row and the second parameter is the number of the column, e.g. A(2,3). Compare the subscript numbering with figure \ref{fig_2d_matlab}
and for more information visit \cite{.matlab_index_ref}.  \newpage

\begin{figure} [!h]
\begin{minipage}{0.4 \textwidth}
\centering
 \def\svgwidth{0.9\textwidth}
 \input{path_Image/numerierung_2d.pdf_tex}
 \caption{Element numbering in 2D.}    % Bildunterschrift 
 \label{fig_element_nr_2D}          % Label für Verweise 
\end{minipage}
\hfill
\begin{minipage}{0.4 \textwidth}
\centering
 \def\svgwidth{\textwidth}
 \input{path_Image/Element_2D.pdf_tex}
\repeatcaption{fig_2d_one_ele}{One 2D squared discretized element with 4 Nodes andDOFs}
\end{minipage}\\

\vspace{1.2cm}
\begin{minipage}{\textwidth}
   \centering
% \def\svgwidth{0.6\textwidth}
 \input{path_Image/numerierung_2d_2.pdf_tex}
 \caption{Numbering in 2D by means of Matlab's array-index-declaration.}    % Bildunterschrift 
 \label{fig_2d_matlab}   
 \end{minipage}
\end{figure}
\newpage


\begin{figure}[!h]
\begin{minipage}{0.6\textwidth}
   \centering
% \def\svgwidth{0.6\textwidth}
 \input{path_Image/numerierung_2d_nodes.pdf_tex}
 \caption{Numbering Nodes of a 2D mesh.}    % Bildunterschrift 
 \label{fig_2d_nr_nodes}    
 \vspace{1cm}
   % \def\svgwidth{\textwidth}
 \input{path_Image/numerierung_2d_dofs.pdf_tex}
 \caption{Numbering DOFs of a 2D mesh. Even Numbers are DOFs of the \textbf{y }axis and all the odd numbers represents the DOFs of the \textbf{x} axis}    % Bildunterschrift 
 \label{fig_2d_nr_dofs}      
   \end{minipage}
   \hfill
   \begin{minipage}[h]{0.35\textwidth}
   Once the user has assigned all the parameters \textbf{nelx, nely, penal, r\textsubscript{min}} and \textbf{volfrac}, the next step, which also belongs to the preconditions (see flowchart \ref{fig_flowchart_88}), is to define a load-case. In order to be able to define a load-case, a mechanical structure needs to be discretized (obtaining a mesh).
   The mesh consists of elements (figure \ref{fig_element_nr_2D}), nodes (figure \ref{fig_2d_nr_nodes}), which have DOFs (figure \ref{fig_2d_nr_dofs}). \\
      
 With FEA, the force is not applied on elements, but rather on nodes, and in order to select a node, it is necessary to be familiar with the elements numbering convention (figure\ref{fig_element_nr_2D}). Now to select a node from an
  chosen element see figures \ref{fig_2d_one_ele} and \ref{fig_2d_nr_nodes}. 
  After having selected the desired node or nodes the user may want to fix some
  of the nodes DOF in the horizontally (\textbf{x}) direction or vertically \textbf{y} direction.
  As figure \ref{fig_2d_nr_dofs} shows, the numbering of the DOFs starts with the horizontally DOF with the number \textbf{$1$} and all the DOFs in the horizontally direction \textbf{x} are defined with odd numbers, respectively all the vertically (\textbf{y}) DOFs have an even number.\\
 
 Now the user is able to set Boundary Conditions (BC) as well as define load-cases. In the 88 lines of code the forces can be applied in the lines \textit{19-20} and the  Boundary Condition (BC) can be modified in the lines \textit{21-23}.
   \end{minipage}
\end{figure} 
\newpage


 \begin{figure} [!h]
\begin{minipage}{0.6 \textwidth}
 \centering
 \def\svgwidth{0.6\textwidth}
 \input{path_Image/Element_3D_xyz.pdf_tex}
 \caption{Local node-numbering for one element.}  
 \label{fig_3d_xyzl}          % Label für Verweise 
 \vspace{1cm}
  \def\svgwidth{0.8\textwidth}
 \input{path_Image/3d_Elements_nr.pdf_tex}
 \caption{Element numbering in 3D}  
 \label{fig_3d_element_numberin}          % Label für Verweise 
\end{minipage}
\begin{minipage}{0.35 \textwidth}
   \subsection{Numbering elements, nodes and DOFs in 3D}
\cite{Liu.2014} uses voxels to discretize 
(see figure \ref{fig_3d_discr})
 the available construction space 
 and therefore each element 
 does not have 4 nodes any more 
 like in 2D, but 8 nodes per element (figure \ref{fig_3d_xyzl}).\\

The numbering of the elements can be described as follows: The first element is located in \textbf{$z=0$}. To understand where 
these z-coordinates must be, see the coordinate system in figure \ref{fig_3d_element_numberin}. The 
numbering starts by defining the \textbf{z}-axis, then the 
numbering continues like in 2D, from up to bottom,
then one element to the right, again up to bottom,
 one element to the right, up to bottom until reaching
  the last element in the current \textbf{z}-axis. In 2D it would be the location \textbf{x(nelx, nely)}( figure \ref{fig_2d_matlab}), where \textbf{x} is Matrix consisting of the discretized elements. Then the element numbering in 3D continues with \textbf{$z+1$}, the depth changes from
  \textbf{$z = 0$} to \textbf{$z=1$} and the numbering
  continues again with up to down and then one element to the right.
\end{minipage}
\end{figure}
\newpage


\begin{figure}[!htb]
\begin{minipage}[!h]{0.65 \textwidth}
%\centering
  \def\svgwidth{\textwidth}
 \input{path_Image/3d_nodes_nr.pdf_tex}
 \caption{Global node IDs and node coordinates in 3D} 
 \label{fig_3d_node_numberin} 
 \vspace{1cm}
\end{minipage}
\hfill
\begin{minipage}{0.3 \textwidth}
In the FE method, to define a load-case, locating
a node is required. The 169 lines of code \cite{Liu.2014} employs node-coordinates in order to target a node. \\
The first step in order to locate a Node, is choosing an Element (see figure \ref{fig_3d_element_numberin}) then choosing a local node 
number (see figure \ref{fig_3d_xyzl}). With the local 
node number, a translation using
the table \ref{table_nr_3d} is needed in order to obtain the
 global Node-ID. Note that the force or the constraint is applied on 
 the global Node-ID, the local Node-ID is only there to help the reader to navigate
  through the mesh. In 3D there are 3 translatory
   axes and the reader can define
  the DOF with the table \ref{table_nr_3d}.\\   
  \end{minipage}   
     \end{figure} 
\begin{table}[!h]
    \begin{minipage}{0.65\textwidth}
   For example, the reader wants to apply a force on the fourth element (figure \ref{fig_3d_element_numberin}). The current knowledge about
   the element is, the element-coordinates are \textbf {x = 2, y = 3 } and \textbf{z = 1}.    
   Since in FEM the forces are applied on nodes, the number
   of the node needs to be found out. By taking the sixth node 
   (figure \ref{fig_3d_node_numberin})
   the first step is to
    think locally (see figure \ref{fig_3d_xyzl}).
     Locally the node-number of the fourth element is \emph{N1}
     (globally Node-ID equals 6).
     Because the local node-number is \emph{N1} the node coordinates can be
     obtained with:
     $x_{ni} = x_{e}-1 = 2-1 = 1$, $y_{ni} = y_{e}-1 = 3-1 = 2$,
     $z_{ni} = z_{e}-1 = 1-1 = 0$, where the index \emph{n} stands for
     \emph{node coordinates} (see figure \ref{fig_3d_node_numberin}), respectively
     the index \emph{e} stands for the \emph{element coordinates}
     (see figure \ref{fig_3d_element_numberin}) in x,y and z
     direction.
   \end{minipage}
   \hfill
   \begin{minipage}{0.3\textwidth}
   \centering
   \begin{tabular}{|c|c|}
 \hline 
 NID\textsubscript{1} &6 \\ 
 \hline 
 NID\textsubscript{2} & 10 \\ 
 \hline 
 NID\textsubscript{3} & 9 \\ 
 \hline 
 NID\textsubscript{4} & 5 \\ 
 \hline 
 NID\textsubscript{5}& 22 \\ 
 \hline 
 NID\textsubscript{6} & 26 \\ 
 \hline 
 NID\textsubscript{7} & 25 \\ 
 \hline 
 NID\textsubscript{8} & 21 \\ 
 \hline 
 \end{tabular} 
 \caption{Example for global Node IDs.} 
        \label{table_nID_example}
   \end{minipage}
            \end{table}
               	 The subtraction with \emph{-1} is required, when the
     chosen local node-number is on the left side in x-direction, at the bottom
     in y-direction and away from the sheet level in z-direction, otherwise the node
     coordinates are equal to the element coordinates.
     In the case of the chosen local node-number \emph{6}, with a mesh size of
     $3 \times 3 \times 1$, the node coordinates can be stated as:
     $x_{ni} = x_{n1}$, $y_{ni} = y_{n2}$,
     $z_{ni} = z_{n0}$, see figure \ref{fig_3d_node_numberin}.
     Note that, even if it is obvious that the global
   node ID of the desired node is 6, however,
    in general, this information is not given and needs to be found out. 
    The goal of this example was to demonstrate, how the reader can find the global node number of the desired node.\\  
    
 With all the informations obtained, it is possible to 
 consider the table \ref{table_nr_3d}. The local node ID is N1, with
 the section node coordinates in the table,
  the node coordinated can be expressed as:
 node coordinates = $(1,2,0)$ and the solution for NID\textsubscript{1} = 6, respectively all the other NID\textsubscript{i} can be obtained from table \ref{table_nID_example}.
 
\begin{table}[!h]
\begin{tabular}{|l|l|l|l|l|l|}
\hline 
 LNN & Node coordinates & Global node ID &
  \multicolumn{3}{c|}{Node DOF } \\ 
  &  & & \cline{1-3}

  &  & & x &y &z \\
\hline 
N\textsubscript{1} & (x\textsubscript{n1}, y\textsubscript{n1}, z\textsubscript{n1}) & NID\textsubscript{1} & 3*NID\textsubscript{1}-2 & 3*NID\textsubscript{1}-1 & 3*NID\textsubscript{1} \\ 
\hline 
N\textsubscript{2} & (x\textsubscript{n1}+1, y\textsubscript{n1}, z\textsubscript{n1}) &NID\textsubscript{2} = NID\textsubscript{1}+(nely +1) & 3*NID\textsubscript{2}-2 & 3*NID\textsubscript{2}-1 & 3*NID\textsubscript{2} \\ 
\hline 
N\textsubscript{3} & (x\textsubscript{n1}+1, y\textsubscript{n1}+1, z\textsubscript{n1}) & NID\textsubscript{3} = NID\textsubscript{1}+nely & 3*NID\textsubscript{3}-2 & 3*NID\textsubscript{3}-1 & 3*NID\textsubscript{3}\\ 
\hline 
N\textsubscript{4} & (x\textsubscript{n1}, y\textsubscript{n1}+1, z\textsubscript{n1}) & NID\textsubscript{4} = NID\textsubscript{1}-1 & 3*NID\textsubscript{4}-2 & 3*NID\textsubscript{4}-1 & 3*NID\textsubscript{4} \\ 
\hline 
N\textsubscript{5} & (x\textsubscript{n1}, y\textsubscript{n1}, z\textsubscript{n1}+1) & NID\textsubscript{5} = NID\textsubscript{1}+NID\textsubscript{z} & 3*NID\textsubscript{5}-2 & 3*NID\textsubscript{5}-1 & 3*NID\textsubscript{5} \\ 
\hline 
N\textsubscript{6} & (x\textsubscript{n1}+1, y\textsubscript{n1}, z\textsubscript{n1}+1) & NID\textsubscript{6} = NID\textsubscript{2}+NID\textsubscript{z} & 3*NID\textsubscript{6}-2 & 3*NID\textsubscript{6}-1 & 3*NID\textsubscript{6} \\ 
\hline 
N\textsubscript{7} & (x\textsubscript{n1}+1, y\textsubscript{n1}+1, z\textsubscript{n1}+1) & NID\textsubscript{7} = NID\textsubscript{3}+NID\textsubscript{z}& 3*NID\textsubscript{7}-2 & 3*NID\textsubscript{7}-1 & 3*NID\textsubscript{7} \\ 
\hline 
N\textsubscript{8} & (x\textsubscript{n1}, y\textsubscript{n1}+1, z\textsubscript{n1}+1) & NID\textsubscript{8} = NID\textsubscript{4} +NID\textsubscript{z}& 3*NID\textsubscript{8}-2 & 3*NID\textsubscript{8}-1 & 3*NID\textsubscript{8} \\ 
\hline 
\end{tabular} 
\caption{
Illustrates the the relationships between node number, node coordinates,node ID and node DOFs.\\
LNN = Local Node Numbering,\\
NID\textsubscript{1} = z\textsubscript{n1}*(nelx+1)*(nely+1)+x\textsubscript{n1}*(nely+1)+(nely+1-y\textsubscript{n1}),\\
NID\textsubscript{z} = (nelx+1)*(nely+1).
\label{table_nr_3d}
}
\end{table}

 \chapter{Adapt in 2D}
%%Welche Parameter gibt es und was machen sie - I am using Paraview as a Postprocessor
%% later you will get more inos about Paraview
 The Adapt is an optimisation code written in Matlab. The idea of the Adapt is to use topology optimisation algorithms to increase any given structures stiffness.\\
 
 \begin{figure}[!h]
 \begin{minipage}{0.45\textwidth}
 The practical use of the adapt is shown in picture
  \ref{fig_karos}. The purple car body is the structure, 
  which the Adapt receives and should generate
   the yellow struts. Through these newly added struts 
   the stiffness of the
  Basis structure, represented by the car body, is
  increased. This is especially important, if motor sport
  vehicles are to be obtained with a series car body. 
  
 \end{minipage}
\hfill
 \begin{minipage}{0.45\textwidth}
  \centering
    \includegraphics[ width =\textwidth]
    {path_Image/pngs/Aufgabe_1/karos.png}
  	\caption{Idea of the Adapt.\protect\footnotemark}  
  	\label{fig_karos}
 \end{minipage}
    \end{figure}  
\footnotetext{This picture is obtained by Saad Hafsa from BMW.}
Since topology optimisation dedicates itself to minimize
 functions, the original code attempts to minimize 
 the compliance of the structure. Compliance is 
 the inverse of stiffness, a small compliance \emph{c} results in a great stiffness.
Before explaining some details about the work-flow it is essential to have a reasonable understanding about some basics, which are introduced in the upcoming section (see \ref{sec_basic_terminology_2d_adapt}).

%Note that all the figures, which are provided in this chapter are generated with Paraview, which will be discussed in section \ref{section_paraview}.


\section{Terminology of the 2D Adapt}
\label{sec_basic_terminology_2d_adapt}
The aim of the following section is to provide the reader with
a brief explanation of some frequently used technical words with regard to the Adapt.\\

{\large Basis:} 
\vspace{0.18cm}
\hrule 
\vspace{0.18cm}

 The Basis
 (see figure \ref{fig_basis_exampl}) 
 is the mechanical structure, which is given to the Adapt and can be seen as one
 of the inputs to the optimiser. The Basis can have
 any structure and is a matrix consisting of 0 and 1, where 0 means void and 1 stands for material. 
 Up to now the basis was generated in Matlab with the 88 lines of code. 
The adapt obtains the basis and treats it as the initial structure. Because 
the Adapt obeys some restrictions,
 it will not change the origin basis structure at all. Again,
the Adapt will not change the Basis-structure, it will not change the basis-matrix and so the basis matrix value remains unchanged. \\

\begin{figure}[!h]
\centering
 \includegraphics[width= 0.45 \textwidth]
 {path_Image/pngs/Meet_Adapt/basis_example.png}
	\caption{The green structure is representative for the Basis.} 
	\label{fig_basis_exampl}
\end{figure}

\newpage
{\large Boundary-Zone (BZ)}:
\vspace{0.18cm}
\hrule 
\vspace{0.18cm}
The Adapts needs to append the new generated material or
  structure to the Basis in some way. 
 The solution is to find the area, which surrounds the Basis.
 This area is denoted as the Boundary-Zone, 
(see figures \ref{fig_bz_alone} and
 \ref{fig_bz_basis}). It is composed of elements, which are 
 stored in a matrix. This BZ-matrix is a boolean-matrix, which only
 contains the values \emph{0} or \emph{1}, where
 \emph{1} means considered element
 belongs to the BZ
 and \emph{0} means the current element 
 does not belong to the BZ. In order to 
 get some  control over the BZ, the parameter "ep" is introduced.\\

\begin{figure} [!h]
\begin{minipage}{0.45\textwidth}
 \includegraphics[width= \textwidth]
 {path_Image/pngs/Meet_Adapt/BZ_alone_example.png}
	\caption{Boundary Zone (BZ) without any other zones or structures.} 
	\label{fig_bz_alone}
\end{minipage}
\hfill
\begin{minipage}{0.45\textwidth}
 \includegraphics[width= \textwidth]
 {path_Image/pngs/Meet_Adapt/BZ_example.png}
	\caption{Basis (green) and BZ (yellow).} 
	\label{fig_bz_basis}
\end{minipage}

\end{figure}

{\large Parameter \emph{ep}: }
\vspace{0.18cm}
\hrule 
\vspace{0.18cm}
\emph{ep} shall define the normal length or the thickness of the BZ. 
The ability to define the normal length of the BZ will be described in chapter \ref{chapter_prohibit_edge}.
An example is given in figures \ref{fig_bz_alone} and
 \ref{fig_bz_basis}, which have an \emph{ep} of 2 (\textbf{$ep = 2$}).\\

\newpage
{\large Remaining Elements (RE)}:
\vspace{0.18cm}
\hrule 
\vspace{0.18cm}
\textit{Available construction space - Basis - BZ}, see table \ref{tabel_basics} and figures \ref{fig_re_example_alone} and \ref{fig_re_bas}. 

\begin{figure} [!h]
\begin{minipage}{0.45\textwidth}
 \includegraphics[width= \textwidth]
 {path_Image/pngs/Meet_Adapt/re_only.png}
	\caption{Remaining Elements (REs) without any other zones or structures.} 
	\label{fig_re_example_alone}
\end{minipage}
\hfill
\begin{minipage}{0.45\textwidth}
 \includegraphics[width= \textwidth]
 {path_Image/pngs/Meet_Adapt/basis_re.png}
	\caption{Basis (green) and REs (blue).} 
	\label{fig_re_bas}
\end{minipage}
\end{figure}
%%__________________________________
%%DESIGN ELEMNTS
%%_____________________________________
{\large Design Elements (DE)}:
\vspace{0.18cm}
\hrule 
\vspace{0.18cm}
\textit{Available construction space - Basis} or \textit{RE+BZ}, see figures \ref{fig_de_example_alone} and \ref{fig_de_bas}.\\

\begin{figure} [!h]
\begin{minipage}{0.45\textwidth}
 \includegraphics[width= \textwidth]
 {path_Image/pngs/Meet_Adapt/de_only.png}
	\caption{Design Elements (DE) without any other zones or structures.} 
	\label{fig_de_example_alone}
\end{minipage}
\hfill
\begin{minipage}{0.45\textwidth}
 \includegraphics[width= \textwidth]
 {path_Image/pngs/Meet_Adapt/basis_de.png}
	\caption{Basis (green) and DEs (turquoise).} 
	\label{fig_de_bas}
\end{minipage}
\end{figure}

{\large Adapt Member (AM)}:
\vspace{0.18cm}
\hrule 
\vspace{0.18cm}
The goal of the optimiser Adapt is to add material to the Basis.
The added structures or members are called Adapt Members (AM) 
(see figures \ref{fig_am_example_alone} and \ref{fig_am_bas}).
The reason why the AMs are not generated around the Basis,
which only would lead to a thicker Basis, is that the deployment
of two different penalization exponents. The BZ is 
penalized with the penalization exponent of
$p = 3$ and the REs are penalized with $p = 2$.\\

\begin{figure} [!h]
\begin{minipage}{0.45\textwidth}
 \includegraphics[width= \textwidth]
 {path_Image/pngs/Meet_Adapt/only_adapt.png}
	\caption{Adapt Members (AM) without any other zones or structures.} 
	\label{fig_am_example_alone}
\end{minipage}
\hfill
\begin{minipage}{0.45\textwidth}
 \includegraphics[width= \textwidth]
 {path_Image/pngs/Meet_Adapt/adapt_basis.png}
	\caption{Basis (green) and AMs (red).} 
	\label{fig_am_bas}
\end{minipage}
\end{figure}

\newpage
{\large Parameter \emph{r\textsubscript{min}} and 
\emph{r\textsubscript{b}}:}
\vspace{0.18cm}
\hrule 
\vspace{0.18cm}
 r\textsubscript{min} is sensitivity filter
 radius  for the remaining 
 elements.
 It prohibits the occurrence of the checkerboard effect and also defines the minimum member size (number of
 directly connected elements) of the REs. r\textsubscript{b} serves the same purpose with respect to the BZ. The difference between r\textsubscript{min} and r\textsubscript{b} is that r\textsubscript{min} is  going to filter the BZ and 
 the REs and r\textsubscript{b} will filter only inside of the BZ (see figure
\ref{fig_re_bz_filter}). 
The figure \ref{adapt_basis_bz} shows a result with r\textsubscript{min} = 1.5 and r\textsubscript{b} = 1.5.\\


%__________AM+BZ+Basis_________-
\begin{figure}
\centering
 \includegraphics[width= \textwidth]
 {path_Image/pngs/Meet_Adapt/all_ohne_wire.png}
	\caption{Green Basis, red AMs and  yellow BZ,
	 r\textsubscript{min} = 1.5 and r\textsubscript{b} = 1.5.} 
	\label{adapt_basis_bz}
\end{figure}


%%%1) _______________Erweiterung des Software-Codes für 3-dimensionale Problemstellungen____________________
\chapter{Expand a existing 2D-adapt code on 3D}
\section{Why expanding the Adapt on 3D?}
In order to be able to make use of a topology optimiser in practice, 
it is necessary to be able to optimise in three dimensions. A \textit{2.5 D}alternative solution would be, to extrude the 2D Adapt- results into the third axis. However, this work 
will present a optimiser, which is going to consider all the three axes at the beginning of the topology optimisation as an 3D optimiser.

\section{Expanding the Adapt on 3D}
\cite{Liu.2014} already provide a Matlab code, which is used as a 3D topology optimiser. This 3D topology optimiser is based on the 88 lines of code by \cite{Andreassen.2011}, which
 means the 169 lines of code \cite{Liu.2014} can be taken as the basis code.
This signifies that only the parts
that make the differences between the Adapt 
and the 88 lines of code need to be extended to 3D, which reduces the 
effort of 3D extension.
 The flowchart \ref{fig_flow_88_adal}  gives an overview about the difference in work flow between the Adapt and the 88 lines of code.
 The deviations from the 88 lines respectivley the 169 lines of code will
 be explained in this section. \\

The first task in order to extend the Adapt on 3D would be discretizing
 the mechanical structures in 3D, therefore the parameter \textbf{nelz} is introduced, which stands for the number of element in the depth-direction (\textbf{z}). Since the 169 lines \cite{Liu.2014} is taken as the
 basis-code, the first task is only named, without any more informations.  \\

\begin{figure}[!h]
 \def\svgwidth{\textwidth}
 \input{path_Image/88_lines_adapt_visio.pdf_tex}
 \caption{The flowchart shows the difference between the Adapt and 88 lines of Code}  
 \label{fig_flow_88_adal}          % Label für Verweise 
 \end{figure}
 
\subsection{ Adapt specific parameter }
Since \textbf{r\textsubscript{b}, e\textsubscript{p} and  r\textsubscript{min}} (see subsection \ref{sec_basic_terminology_2d_adapt}) are  scalar
values they do not need to be modified and furthermore 
these parameters are independent of the number of dimensions. \\

\subsection{Finding the \textbf{Basis}} 
\label{subsection_finding_basis}

In 2D, the optimiser, looped though the whole Basis-matrix (see figure \ref{fig_basis_exampl}) and verified whether the matrix-entry, which is a density value between 0 and 1, is greater than a predefined Basis-recognise-threshold. If the considered Basis-matrix-entry is greater than
the Basis-recognise-threshold, then this entry
 will be treated as an element, which actually belongs to the Basis.
 Otherwise the element is not going to be treated as part of the Basis.
 The Basis-recognise-threshold is required up to now, because, the Basis is generated with the 169 line of code in 3D or with the 88 lines of Code in 2D and both
  matrix-results contain intermediate densities.
  It is possible to generate any matrix by the Matlab user
   and provide it to the Adapt as the Basis. 
   Later on, in section \ref{chapter_external_basis} 
   the ability of inclusion of any external Basis is provided.\\
   
In 3D the optimiser has to loop through the 3D-Basis-matrix and has to do the same verification, as explained for the 2D-Adapt (see Listing \ref{lst_finding_BZ} line 5 and \ref{lst_finding_BZ_3D} line 6).
 The only difference is that the Basis-Matrix has a size of \textbf{$n \times m \times j$}, where  $(n , m,j) \in \mathbb{N} $ .\\


\subsection{Finding the \textbf{Design Elements}}
\label{subsection_finding_DEs}
All the DEs are stored inside the DE-matrix (see figures  \ref{fig_de_example_alone} 
and \ref{fig_de_bas}). 
The difference between 
the 2D- and 3D-DE-matrix is the same as explained
in subsection \ref{subsection_finding_basis}. 
After finding by means of 
the Basis-recognise-threshold, which element
 in the mesh belongs to the Basis and which not, 
 the following task can be performed: all the elements,
  which do not belong to the Basis must assigned 
  to the DE-matrix. Therefore with respect to 
  expending the Adapt to 3D, the main adjustment 
  is not only to apply the Basis-recognise-threshold 
  in two axes, but also in the third axis.\\

\subsection{Finding the \textbf{Boundary Zone}}
\label{subsectin_find_BZ}
The BZ is a matrix, which
is required to penalize the surrounding elements of the Basis with a higher
penalization exponent. This is essential to prevent all AMs to be 
applied around the Basis, which would only lead to a thickening of
the Base. 
In order to find the BZ in 2D with the assumption,
 already knowing which elements belongs to the Basis and which not, 
 two main steps need to be accomplished 
 (see figures \ref{fig_bz_alone} and \ref{fig_bz_basis}). 
The first step is, to loop over the Basis 
elements and store all the neighbour 
elements of the Basis in the BZ-matrix. 
The neighbour elements are  composed of 
the next and previous element in the horizontal 
direction, and the next and previous elements in the vertically
 direction. Since the parameter
  \emph{ep} defined the thickness of the BZ,
  the Adapt loops over Basis and stores the \emph{ep-th} neighbour
   element inside the BZ-matrix. In other words,
    if the current element belongs to the basis, then
     the optimiser is going to the next \emph{ep}  
     and to the previous \emph{ep} neighbour element 
     in the horizontally and vertically direction and will store their density-values inside the BZ-matrix, see figure \ref{fig_find_BZ_2D}. \\

\begin{figure}[!h]
 \centering
 \def\svgwidth{0.45\textwidth}
 \input{path_Image/BZ_1_2d.pdf_tex}
 \caption{The green element represents a Basis-element and the yellow elements with a distance of \emph{ep} are the neighbour elements. }  
 \label{fig_find_BZ_2D}          % Label für Verweise 
\end{figure}

The next step is to overlap the BZ and the Basis and in case of an overlapping, 
the overlapped elements inside the BZ are 
getting a value of 0. The reason for that is, that 
 only the identification of the Basis surrounding elements is desired. 
 Therefore all
 the elements which belongs to the Basis are not 
 allowed to be inside the BZ-matrix (see Matlab code \ref{lst_finding_BZ} for a better understanding).\\

In order to find the BZ in the third dimension, 
a loop over the three dimensional Basis and the extension of the BZ search
is required. Regarding extension of the BZ-search, the
 neighbour elements by \emph{ep} next and \emph{ep}  previous 
 elements in the depth-direction needs to be considered.
(\textbf{z}). Note that, before verifying the
 Basis has a neighbour, it is necessary to check, 
 whether the running index does not exceed the 
 matrix dimension. For example, if the matrix 
 Basis have the dimensions $100 \times 50 \times 20$, a
  neighbour at the position $A(100,51,21)$ is not going 
  to exist (see the if condition in lines 6,9,12 and 
  14 in the the Listing \ref{lst_finding_BZ}).
  The 3D method for finding the BZ is presented in
   Listing \ref{lst_finding_BZ_3D}. \\

\subsection{Finding the \textbf{Remaining Elements}}
\label{subsectin_find_REs}
Since the BZ-matrix is already obtained,
 the BZ-matrix and the DE-matrix, the RE-matrix can be obtained by 
 overlapping the DE-matrix and the BZ-matrix.
  The only difference between the REs and DEs is that the RE do not have any BZ-elements. This means each BZ-element needs to be removed from the RE-matrix, or, its density needs to be set to 0. The 3D  extension only requires to take the third axis into account. The 2D  code and the 3D code for obtaining the RE-matrix can be obtained in Listening \ref{lst_2d_RE} and \ref{lst_3d_RE}.
  The set of elements can be seen on figures \ref{fig_re_example_alone} and \ref{fig_re_bas}.\\

\subsection{Sensitivity filter for remaining elements and boundary zone }
\label{subsection_SF_RE_BZ}

In the 169 lines of code, the sensitivity filter is applied on all the element. Because the Adapt distinguish between the RE and the BZ, it is 
required to have two different filter, which filters in their respective zones. The
RE-filter considers all the REs and the BZ, whereas the BZ-filter only filters
elements inside the BZ.
The adjustment for the REs-filter can be achieved by a
if-statement, which examines, whether the elements are part of the RE-matrix and the same apllies for the BZ-filter.
By means of a if-condition, it can be examined, whether a element belongs to the BZ. In case the current considered element belongs to the BZ-matrix, it is going to be considered by the BZ-filter otherwise not (see figure \ref{fig_re_bz_filter}).

\begin{figure} [!h]
\centering
 \def\svgwidth{0.85\textwidth}
 \input{path_Image/re_bz_filter.pdf_tex} 
 \caption{The procedure of the REs-filter and the BZ-filter in 2D, where the green squared elements are the Basis and the yellow elements represents the surrounding BZ.}    % Bildunterschrift 
 \label{fig_re_bz_filter}  
 \end{figure}
 
\subsection{Sensitivity penalization of the BZ and the REs }
This task can be performed with the explanations in the subsections \ref{subsectin_find_REs}
respectively \ref{subsection_finding_DEs}. The penalization 
exponent does not change by extending to 3D. 
The REs are still penalized with an exponent, or 
a value  of \textit{2} and the BZ is penalized with an exponent
 of \textit{3}. According to the figure \ref{fig_SIMP_sceme_modified} factor \textit{3} is more severe in penalization.










\chapter{3D Adapt Results}
In the field of 3D topology optimization,
visualization of the optimization progress and results is very important
in order to understand how a set of parameter influences
the evolution of the optimization and to evaluate the result in a qualitative way. Matlab as a post processor is not suitable for this, reasons for this
statement and advantages which are offered by the chosen post processor ParaView are described along with handling instructions in the section
 \ref{section_paraview}. 
All shown results were generated with Maltab and displayed with ParaView. 
First of all, it will be shown how the Adapt
was extended from 2D to 3D. For this purpose, a 2D load case has been
defined in 3D with on element in the depth in order to be
similar to each other. To show the results,
 one 2D result, one 3D result, one overlap
 and the inverse of an overlap are shown with 
 the usage of two different thresholds . Afterwards results with more than
 one element in the depth will be presented in section
  \ref{section_3d_loadcases}.

\section{Compare 2D and 3D results }
In order to find, whether the Adapt is correctly extended to 3D some investigations were made. A reasonable assumption is, that a 3D mesh with only one element in the depth-direction (\textbf{z}) should lead
to similar results to those obtained with
a 2D mesh.
A comparison between a 2D mesh and a
 3D mesh with only one depth- element was tested 
 with three different load-cases.

\subsection{2D and 3D load-cases}

The first load-case is known as the Cantilever,
where the left
 lower edge and the left upper edge is clamped and a force is applied vertically. In figure \ref{fig_loadcase_canti_2d} a force is applied at 80\% of the horizontally length at the bottom in vertical direction. In order to be able to get similar results, the 3D load-case is defined as follow:
 the length-details remains unchanged, however, two forces with the half magnitude of the origin force magnitude are applied, see figure \ref{fig_loadcase_canti_3d}.

\begin{figure}[!h]
\begin{minipage}{0.45\textwidth}
\centering
 \def\svgwidth{\textwidth}
\input{path_Image/load_case_2d_canti.pdf_tex} 
\caption{Load-case Cantilever in 2D.}  
\label{fig_loadcase_canti_2d}
\end{minipage}
\hfill
\begin{minipage}{0.45\textwidth}
\centering
 \def\svgwidth{\textwidth}
\input{path_Image/load_case_3d_canti.pdf_tex} 
\caption{Adjusted load-case Cantilever for 3D.}  
\label{fig_loadcase_canti_3d}
\end{minipage}
\end{figure}

The second for this purpose used load-case is similar to the the first one, therefore this load-case will be defined as Cantilever\textsubscript{2}. The force is applied vertically at 50\% of the whole vertically length and at 100\% of the horizontally length, see figure \ref{fig_loadcase_canti2_2d}. In figure \ref{fig_loadcase_canti2_3d} is shown which load-case was used to approximate the 2D load-case for 3D.\\

\begin{figure}[!h]
\begin{minipage}{0.45\textwidth}
\centering
 \def\svgwidth{\textwidth}
\input{path_Image/load_case_2d_canti2.pdf_tex} 
\caption{Load-case Cantilever\textsubscript{2} in 2D.}  
\label{fig_loadcase_canti2_2d}
\end{minipage}
\hfill
\begin{minipage}{0.45\textwidth}
\centering
 \def\svgwidth{\textwidth}
\input{path_Image/load_case_3d_canti2.pdf_tex} 
\caption{Adjusted load-case Cantilever\textsubscript{2} for 3D.}  
\label{fig_loadcase_canti2_3d}
\end{minipage}
\end{figure}

The last load-case is called MBB (Messerschmitt-Bölkow-Blohm),
where a force is applied at the top center. See figure \ref{fig_loadcase_mbb_2d} for the 2D load-case and the used 3D load case in figure \ref{fig_loadcase_mbb_3d}.
\begin{figure}[!h]
\begin{minipage}{0.45\textwidth}
\centering
 \def\svgwidth{\textwidth}
\input{path_Image/load_case_2d_mbb.pdf_tex} 
\caption{Load-case MBB in 2D.}  
\label{fig_loadcase_mbb_2d}
\end{minipage}
\hfill
\begin{minipage}{0.45\textwidth}
\centering
 \def\svgwidth{\textwidth}
\input{path_Image/load_case_3d_mbb.pdf_tex} 
\caption{Adjusted load-case MBB for 3D.}  
\label{fig_loadcase_mbb_3d}
\end{minipage}
\end{figure}

\newpage
\subsection{2D and 3D comparison results}
The upcoming figures, which are presented from figure \ref{fig_2dbasis_03} to
 figure \ref{fig_3dbasis_re_03}, show the 2D Basis, the 3D Basis, an overlapping of
 the 2D and 3D Basis and also one figure with highlighted elements which
 are not common in both models.
 The 2D Basis (figure \ref{fig_2dbasis_03}) is generated with the 88
  lines of code by \cite{Andreassen.2011} and the 3D Basis (figure \ref{fig_3dbasis_03}) is
  obtained with the 169 lines of code by \cite{Liu.2014}. 
  Since it is hardly possible 
   to see the difference between the both Bases without any support of a tool, an
 overlapping figure \ref{fig_3dbasis_overl_03} and a figure, which highlights
 especially the not matching elements (figure \ref{fig_3dbasis_re_03}), which
 are coloured yellow.  
  Both Bases
  are obtained with a $volfrac = 0.2$, a mesh-resolution of $100 \times 50$
  respectively $100 \times 50 \times 1$, $r_{min} = 1.5,\; r_{b} = 1.5,\; ep = 2$, with a
   penal-exponent of $p = 3$ and the load-case Cantilever, 
   2D-(see figure
   \ref{fig_loadcase_canti_2d}) respectively 3D-load case (see figure
 \ref{fig_loadcase_canti_3d}).
  All the shown figures starting from figure \ref{fig_2dbasis_03} to figure \ref{fig_3dbasis_re_03}
  are presented with the restriction of a threshold of 0.3
 (see subsection \ref{subsection_paraview_threshold} for more details about thresholds).
It can be observed that, there is a difference between the 2D Basis
(figure \ref{fig_2dbasis_03}) and the 3D Basis (figure \ref{fig_3dbasis_03}).
The reason for that lies in the
optimization procedure or optimization history of each element
and furthermore the simulations accuracy is limited to the computers numerically precision. \\

With the same working procedure the extension from the 2D Adapt to 3D is examined. Figure \ref{fig_2dadapt_03} shows a 2D Adapt, generated with the 2D optimiser Adapt, figure \ref{fig_3dbadapt_03} shows a 3D Adapt,
with only
 one element in the third axis (\textbf{z}), the figure \ref{fig_3dadapt_overl_03}
 shows an overlapping and figure \ref{fig_3dadapt_re_03} shows the difference
  between the 2D and 3D Adapt. It can be observed, that the difference
  between the 2D Adapt and 3D Adapt is similar to the difference between the 2D Basis and 3D Basis.
  Note that, the figures \ref{fig_2dbasis_03} - 
  \ref{fig_3dadapt_re_03} are presented with a threshold of 0.3 and the figures \ref{fig_2dbasis_05} - \ref{fig_3dadapt_re_05} are
presented 
 with a threshold of 0.5.\\

The reason, why
  some results
  with a threshold of \emph{0.5} show elements, which 
  are completely alone in the space without being connected with
   other elements is the threshold. The purpose of a threshold is to prohibit the display
   of elements which do not satisfy the value of the threshold.
   A threshold of 0.5 means that every element in the matrix, which do not fulfil the
    criteria of exhibiting a density value, which is $\rho_e \geq 0.5$ is not going to
    be displayed. \\
       
     The comparison of the 2D and 3D Adapts with the other two load-cases Cantilever\textsubscript{2} and MBB exhibits a smaller difference. The least difference, with only 5 elements was observed with Cantilever\textsubscript{2} as the load-case.
   
%:::::::::::::::::2D and 3D comparison :::::::::::::::
\begin{figure}[!h]
\begin{minipage}{0.45\textwidth}
%_____2D BASIS_________
  \includegraphics[width = \textwidth]{path_Image/pngs/Aufgabe_1/Uberlappungen/2d_basi_3D_basis/2d_basis_03.png}
	\caption{Threshold: 0.3, 2D green Basis generated with 88 lines of code \protect\cite{Andreassen.2011}} 
	\label{fig_2dbasis_03}
\end{minipage}
\hfill
\begin{minipage}{0.45\textwidth}
%K____3D_Basis__________________
  \includegraphics[width = \textwidth]{path_Image/pngs/Aufgabe_1/Uberlappungen/2d_basi_3D_basis/3d_basis_03.png}
	\caption{Threshold: 0.3, 3D red Basis generated with 169 lines of code \protect\cite{Liu.2014}.} 
	\label{fig_3dbasis_03}
\end{minipage}
\end{figure}
\begin{figure}[!h]
\begin{minipage}{0.45\textwidth}
%______2D und 3D Basis Uberlappung______
  \includegraphics[width = \textwidth]{path_Image/pngs/Aufgabe_1/Uberlappungen/2d_basi_3D_basis/2d_3d_basi_ov_03.png}
    \vspace*{-10mm}
	\caption{Threshold: 0.3, overlapped green 2D and yellow 3D Basis.} 
	\label{fig_3dbasis_overl_03}
\end{minipage}
\hfill
\begin{minipage}{0.45\textwidth}
%______2D und 3D Basis RE______
  \includegraphics[width = \textwidth]{path_Image/pngs/Aufgabe_1/Uberlappungen/2d_basi_3D_basis/2d_3d_basi_re_03.png}
    \vspace*{-10mm}
	\caption{Threshold: 0.3, overlapping 2D and 3D Basis.} 
	\label{fig_3dbasis_re_03}
	\end{minipage}
\end{figure}

%
%__________________ADAPT --- 0.3________________________________________
%
%
\begin{figure}[!h]
\begin{minipage}{0.45\textwidth}
%_____2D BASIS_________
  \includegraphics[width = \textwidth]{path_Image/pngs/Aufgabe_1/Uberlappungen/2D_Adapt_2DBasis_Adapt/2d_adapt_03.png}
  \vspace*{-10mm}
	\caption{Threshold: 0.3, 2D green Adapt generated with the 2D Adapt.} 
	\label{fig_2dadapt_03}
\end{minipage}
\hfill
\begin{minipage}{0.45\textwidth}
%K____3D_Basis__________________
  \includegraphics[width = \textwidth]{path_Image/pngs/Aufgabe_1/Uberlappungen/2D_Adapt_2DBasis_Adapt/3d_adapt_03.png}
    \vspace*{-10mm}
	\caption{Threshold: 0.3, current 3D Adapt.}
	\label{fig_3dbadapt_03}
\end{minipage}
\end{figure}
\begin{figure} [H]
\begin{minipage}{0.45\textwidth}
%______2D und 3D Basis Uberlappung______
  \includegraphics[width = \textwidth]{path_Image/pngs/Aufgabe_1/Uberlappungen/2D_Adapt_2DBasis_Adapt/2d_3d_adapt_th_03_ov.png}
    \vspace*{-10mm}
	\caption{Threshold: 0.3, overlapped green 2D and yellow 3D Adapt.} 
	\label{fig_3dadapt_overl_03}
\end{minipage}
\hfill
\begin{minipage}{0.45\textwidth}
%______2D und 3D Basis RE______
  \includegraphics[width = \textwidth]{path_Image/pngs/Aufgabe_1/Uberlappungen/2D_Adapt_2DBasis_Adapt/2d_3d_adapt_th_03_re.png}
    \vspace*{-10mm}
	\caption{Threshold: 0.3, overlapping 2D and 3D Adapt.} 
	\label{fig_3dadapt_re_03}
	\end{minipage}
\end{figure}

%_______________________________BASIS 2D 3D 0.5___________________________
\begin{figure}[!h]
\begin{minipage}{0.45\textwidth}
%_____2D BASIS_________
  \includegraphics[width = \textwidth]{path_Image/pngs/Aufgabe_1/Uberlappungen/2d_basi_3D_basis/2d_basis_05.png}
      \vspace*{-10mm}
	\caption{Threshold: 0.5, 2D green Basis generated with 88 lines of code \protect\cite{Andreassen.2011}} 
	\label{fig_2dbasis_05}
\end{minipage}
\hfill
\begin{minipage}{0.45\textwidth}
%K____3D_Basis__________________
  \includegraphics[width = \textwidth]{path_Image/pngs/Aufgabe_1/Uberlappungen/2d_basi_3D_basis/3d_basis_05.png}
      \vspace*{-10mm}
	\caption{Threshold: 0.5, 3D red Basis generated with 169 lines of code \protect\cite{Liu.2014}.} 
	\label{fig_3dbasis_05}
\end{minipage}
\end{figure}
\begin{figure}[!h]
\begin{minipage}{0.45\textwidth}
%______2D und 3D Basis Uberlappung______
  \includegraphics[width = \textwidth]{path_Image/pngs/Aufgabe_1/Uberlappungen/2d_basi_3D_basis/2d_3d_basi_ov_05.png}
      \vspace*{-10mm}
	\caption{Threshold: 0.5, overlapped green 2D and yellow 3D Basis.} 
	\label{fig_3dbasis_overl_05}
\end{minipage}
\hfill
\begin{minipage}{0.45\textwidth}
%______2D und 3D Basis RE______
  \includegraphics[width = \textwidth]{path_Image/pngs/Aufgabe_1/Uberlappungen/2d_basi_3D_basis/2d_3d_basi_re_05.png}
      \vspace*{-10mm}
	\caption{Threshold: 0.5, overlapping 2D and 3D Basis.} 
	\label{fig_3dbasis_re_05}
	\end{minipage}
\end{figure}

%
%__________________ADAPT --- 0.5________________________________________
%
%
\begin{figure}[H]
\begin{minipage}{0.45\textwidth}
%_____2D BASIS_________
  \includegraphics[width = \textwidth]{path_Image/pngs/Aufgabe_1/Uberlappungen/2D_Adapt_2DBasis_Adapt/2d_adapt_05.png}
      \vspace*{-10mm}
	\caption{Threshold: 0.5, 2D green Adapt generated with the 2D Adapt.} 
	\label{fig_2dadapt_05}
\end{minipage}
\hfill
\begin{minipage}{0.45\textwidth}
%K____3D_Basis__________________
  \includegraphics[width = \textwidth]{path_Image/pngs/Aufgabe_1/Uberlappungen/2D_Adapt_2DBasis_Adapt/3d_adapt_05.png}
	\caption{Threshold: 0.5, current 3D Adapt.}
	\label{fig_3dbadapt_05}
\end{minipage}
\end{figure}
\begin{figure} [!h]
\begin{minipage}{0.45\textwidth}
%______2D und 3D Basis Uberlappung______
  \includegraphics[width = \textwidth]{path_Image/pngs/Aufgabe_1/Uberlappungen/2D_Adapt_2DBasis_Adapt/2d_3d_adapt_th_05_ov.png}
      \vspace*{-10mm}
	\caption{Threshold: 0.5, overlapped green 2D and yellow 3D Adapt.} 
	\label{fig_3dadapt_overl_05}
\end{minipage}
\hfill
\begin{minipage}{0.45\textwidth}
%______2D und 3D Basis RE______
  \includegraphics[width = \textwidth]{path_Image/pngs/Aufgabe_1/Uberlappungen/2D_Adapt_2DBasis_Adapt/2d_3d_adapt_th_05_re.png}
      \vspace*{-10mm}
	\caption{Threshold: 0.5, overlapping 2D and 3D Adapt.} 
	\label{fig_3dadapt_re_05}
	\end{minipage}
\end{figure}

%____________3D results with more than one depth-elemt________________-

\section{3D Adapt results with more than one elements in depth}
\label{section_3d_loadcases}

Before presenting the main 3D Adapt results,
some new 3D load cases are introduced.
 Figure \ref{fig_load_case_clamped_cant} 
 shows the load-case Cantilever\textsubscript{3}. The difference between the Cantilever and
 Cantilever\textsubscript{3} is in the boundary 
conditions. The Cantilever\textsubscript{3}'s left side is 
 clamped fully, whereas the Cantilever has only 4 corner  
 nodes, which are clamped. The second difference 
 is that, the applied force is not multiplied by the 
 factor $\frac{1}{2}$, each node in the depth (\textbf{z}),
  is stressed with the same force \textbf{$F = 1$}. The 
 topology will not change by increasing the 
 magnitude of $F$, however, the compliance will increase.
 The second 3D load-case is a modified version 
 (see figure \ref{fig_load_case_MBB_modified}) of the 
 MBB load-case (figure \ref{fig_loadcase_mbb_3d}).
 Each node in the depth (z) is clamped, which means
 that the DOFs are blocked in all three axes. Furthermore
 the force equals $F = 1$, instead of $F = \frac{1}{2}$ and is applied on
 each node in the depth. The 
 load-case Cantilever\textsubscript{2} 
 (figure \ref{fig_loadcase_canti2_3d}) is also 
 deployed with some modifications for this section. 
 The boundary conditions for this lode-cases remains 
 unchanged, however the applied magnitude of the 
 forces are not going to be reduced by half and also 
 the force is applied on each 
 node in the depth (\textbf{z}). Therefore the 
 modified MBB, Cantilever\textsubscript{2} and
  Cantilever\textsubscript{3}  will be 
 defined as depth load-cases.\\
 
 The 3D Adapt results are shown 
in figure \ref{fig_load_canti_02} - \ref{fig_load_mitchell_04_t} 
 with two different thresholds,
 where the left figure is presented with a displaying threshold of
 \emph{0.2} and the right figure has a threshold of
 \emph{0.4}. The Basis is gray coloured and the
 added AMs are coloured with the colormap 
 \emph{Blue to Red Rainbow} with
 blue representing the lowest densitiy respectively
 red the highest. The table \ref{tabe_3d_load_cases} 
 offers more informations about
 the figures \ref{fig_load_canti_02} - \ref{fig_load_mitchell_04_t}.
 The compliance \emph{c} of the Basis and of
 the Adapt from table \ref{tabe_3d_load_cases}
 are not completely correct for their presented threshold.
 The reason for this is, if a threshold is
  defined, then all elements that do not fulfil this threshold
   are considered as non-existent elements (no material) and 
   all elements with a density above this threshold are set to \emph{1}.
    This means that several elements are completely lost, but 
    also that the density of many elements is artificially 
    increased to \emph{1}. 
    Increasing the density to 1 is necessary because
     in practice there are no intermediate densities. 
     There is either material or there is none. In order 
     to be able to give an exact c-value, it would be
      essential to load each Adapt result with
       the desired threshold into e.g. Optistruct and 
       perform a recalculation. However, in order to provide a comparison value, 2 
       Adapt results with different thresholds were 
       loaded and the variation to the \emph{c}-values given in 
       the table
       \ref{tabe_3d_load_cases}
        is about 2\%. To emphasize it again,
        the values in the table
        \ref{tabe_3d_load_cases} 
        are not entirely 
        correct for their thresholds and therefore no guarantee is given on 
        the accuracy of the \emph{c}-values.

 
 
% _______________Load-Case______________________
 \begin{figure} [!h]
\begin{minipage}{0.45\textwidth}
 \centering
 \def\svgwidth{\textwidth}
 \input{path_Image/load_case_3d_canti_voll.pdf_tex}
 \caption{3D load-case: Cantilever\textsubscript{3}.}    % Bildunterschrift 
 \label{fig_load_case_clamped_cant}          % Label für Verweise f
\end{minipage}
\hfill
\begin{minipage}{0.45\textwidth}
 \centering
 \def\svgwidth{\textwidth}
 \input{path_Image/load_case_3d_mbb_2.pdf_tex}
 \vspace*{6mm}
 \caption{3D load-case: MBB modified.}    % Bildunterschrift 
 \label{fig_load_case_MBB_modified}          % Label für Verweise f
\end{minipage}
\end{figure} ~\\

% __________________::::::::::::::::::::::The 3D Results_________________________-
\begin{figure}[!h]
\begin{minipage}{0.45\textwidth}
%________________CANTIlever______________________
\centering
  \includegraphics[width=  \textwidth]{path_Image/pngs/Aufgabe_1/3D_Ergebnisse/canti.png}
	\caption{Cantilever\textsubscript{3}, threshold 0.2, $100\times 50 \times 8$.}  
	\label{fig_load_canti_02}
\end{minipage}
\hfill
\begin{minipage}{0.45\textwidth}
\centering
  \includegraphics[width= \textwidth]{path_Image/pngs/Aufgabe_1/3D_Ergebnisse/canti_04.png}
	\caption{Cantilever\textsubscript{3}, threshold 0.4, $100\times 50 \times 8$.} 
	\label{fig_load_canti_04}
\end{minipage}\\

\vspace{0.75cm}
%____________________Cantilever_2________________
\begin{minipage}{0.45\textwidth}
\centering
  \includegraphics[width=  \textwidth]{path_Image/pngs/Aufgabe_1/3D_Ergebnisse/mitchell.png}
	\caption{Cantilever\textsubscript{2}, threshold 0.2, $100\times 50 \times 8$.} 
	\label{fig_load_mitchell_02}
\end{minipage}
\hfill
\begin{minipage}{0.45\textwidth}
\centering
  \includegraphics[width= \textwidth]{path_Image/pngs/Aufgabe_1/3D_Ergebnisse/mitchell_04.png}
	\caption{Cantilever\textsubscript{2}, threshold 0.4, $100\times 50 \times 8$.} 
	\label{fig_load_mitchell_04}
\end{minipage}\\

\vspace{0.75cm}
%______________________MBB_________________
\begin{minipage}{0.45\textwidth}
\centering
  \includegraphics[width=  \textwidth]{path_Image/pngs/Aufgabe_1/3D_Ergebnisse/mbb.png}
	\caption{MBB modified, threshold 0.2, $100\times 50 \times 8$.} 
	\label{fig_load_mbb_02}
\end{minipage}
\hfill
\begin{minipage}{0.45\textwidth}
\centering
  \includegraphics[width= \textwidth]{path_Image/pngs/Aufgabe_1/3D_Ergebnisse/mbb_04.png}
	\caption{MBB modified, threshold 0.4, $100\times 50 \times 8$.} 
	\label{fig_load_mbb_04}
\end{minipage}
\end{figure}

\begin{figure}[!h]
%_______________________Canti1 -Tiefen________________
\begin{minipage}{0.45\textwidth}
\centering
  \includegraphics[width=  \textwidth]{path_Image/pngs/Aufgabe_1/3D_Ergebnisse/canti_t.png}
	\caption{Cantilever\textsubscript{3}, threshold 0.2, $100\times 50 \times 14$.} 
	\label{fig_load_canti_02_t}
\end{minipage}
\hfill
\begin{minipage}{0.45\textwidth}
\centering
  \includegraphics[width= \textwidth]{path_Image/pngs/Aufgabe_1/3D_Ergebnisse/canti_t_04.png}
	\caption{Cantilever\textsubscript{3}, threshold 0.4, $100\times 50 \times 14$.} 
	\label{fig_load_canti_04_t}
\end{minipage}\\

\vspace{0.75cm}
%%______________________CANTI 2_Tiefen__________________--
\begin{minipage}{0.45\textwidth}
\centering
  \includegraphics[width=  \textwidth]{path_Image/pngs/Aufgabe_1/3D_Ergebnisse/mitchell_t.png}
	\caption{Cantilever\textsubscript{2}, threshold 0.2, $100\times 50 \times 14$.} 
	\label{fig_load_mitchell_02_t}
\end{minipage}
\hfill
\begin{minipage}{0.45\textwidth}
\centering
  \includegraphics[width= \textwidth]{path_Image/pngs/Aufgabe_1/3D_Ergebnisse/mitchell_t_04.png}
	\caption{Cantilever\textsubscript{2}, threshold 0.4, $100\times 50 \times 14$.}
\label{fig_load_mitchell_04_t}
\end{minipage}
\end{figure}
%%::::::::::::::::::::.3D REsults-Table::::::::::::::::::::::::::::::::::
With the table from the table \ref{tabe_3d_load_cases} it
 can be observed that each \emph{c}-Basis is higher than 
 \emph{c}-Adapt. This proves that with the usage of the
Adapt the compliance is reduced, or the stiffness
is increased. Furthermore it can be seen that the
improvement regarding stiffness is highly depended on the
load-case, e.g. in case of the load-case Cantilever\textsubscript{3} the
stiffness is increased by the factor \emph{1.57}, whereas the
the load-case Cantilever\textsubscript{2} exhibits an improvement
of the factor \emph{1.19}.\\

\begin{table}[!h]
\begin{tabular}{|c|c|c|c|c|c|c|c|}
\hline 
\rowcolor{Green}
Figure & Mesh resolution & TA & \emph{c} Basis &
\emph{c} Adapt& NI & sec/iteration \\ 
\hline 
\ref{fig_load_canti_02} &$100\times 50 \times 8$&0.2& 448.9 &286.34 &  167& 34 \\ 
\hline 
\ref{fig_load_canti_04} &$100\times 50 \times 8$&0.4& 448.9&  286.34 & 167 & 34 \\ 
\hline 
\ref{fig_load_mitchell_02}  &$100\times 50 \times 8$& 0.2&874.28 &  737.775 & 250 & 42 \\ 
\hline 
\ref{fig_load_mitchell_04} &$100\times 50 \times 8$& 0.4 &874.28 &  737.775  & 250 & 42 \\ 
\hline 
\ref{fig_load_mbb_02} &$100\times 50 \times 8$& 0.2&72.15& 59.5 & 146 & 32 \\ 
\hline 
\ref{fig_load_mbb_04} &$100\times 50 \times 8$&0.4&72.15 &  59.5 & 146 & 32 \\ 
\hline 
\ref{fig_load_canti_02_t}  &$100\times 50 \times 14$&0.2& 2699.62&1750.9 & 250 & 36 \\ 
\hline 
\ref{fig_load_canti_04_t} &$100\times 50 \times 14$&0.4&2699.62& 1750.9 & 250 & 36 \\ 
\hline 
\ref{fig_load_mitchell_02_t} &$100\times 50 \times 14$& 0.2 & 7930.95 & 7084.74 & 193 & 38 \\ 
\hline 
\ref{fig_load_mitchell_04_t} &$100\times 50 \times 14$& 0.4&7930.95& 7084.74 & 193 & 38 \\ 
\hline 
\end{tabular} 
\caption{Shows the compliance \emph{c} with the usage of the optimiser Adapt and without (Basis).\\
TA = Threshold Adapt
NI = Number of iterations.\\
\emph{c} = compliance\\
Parameters of the figures \protect\ref{fig_load_canti_02} to \protect\ref{fig_load_mitchell_04_t}\\
Basis-volfrac = 0.2, Adapt-volfrac = 0.3, ep = 2, r\textsubscript{min} = r\textsubscript{b} = 1.5.
}
\label{tabe_3d_load_cases}
\end{table}

%______________________________________________________________________________________________________
%::::::::::::::::::::::3D Results -- Parameter::::::::::::::::::::::::::::::::::::::::::::::::
\section{Parameter investigation}

This section will investigate how changing Adapt specific parameters
 affects the compliance and display the Adapt results.
  For this,
the figure \ref{fig_comp_orig} is defined as the
origin Adapt result, which serves for comparison.
The load case is named as the Cantilever\textsubscript{3} (see figure
\ref{fig_load_case_clamped_cant}), the mesh resolution is 
defined as $100 \times 50 \times 8$, $ep = 2$, $r_{min} = r_b = 1.5$, the \textit{volfrac}
 of the Basis is \emph{0.2} and the \textit{volfrac} of the Adapt is
 \emph{0.3}. 
 However, the
 Re-filter radius r\textsubscript{min} and the BZ-filter 
 radius r\textsubscript{b} have always the same value. First 
 the influence of the parameter \emph{ep} is shown. Afterwards with
  constant \emph{ep}-value the filter radii are changed. The results
  are shown in figures \ref{fig_comp_ep_3} - \ref{fig_comp_re_3},
  all with a display threshold of \emph{0.2}.
  A summary of these results can be obtained from the
  table \ref{tabel_compare_parameters}. \\
  
  The figures \ref{fig_comp_orig} - \ref{fig_comp_ep_5} exhibit an increasing \emph{ep}-value.
  It can be observed that with the 
  increase of \emph{ep} the distance of the \emph{AMs}
  and the Basis is increased as expected. This effect can be seen in
  the left upper area of the figures 
  \ref{fig_comp_orig} - \ref{fig_comp_ep_5}.
  The reason for this occurrence is that, \emph{ep}
  defines the thickness of the BZ, in which the elements
  are penalized with the higher exponent of \emph{3}. This
  means only the elements with a high sensitivity can remain in the
  BZ. In case of the actual Basis and the Adapt, the Basis load-case
  and the Adapt load-case are equal, which
  signifies that the sensitivity around the Basis is 
  higher than elsewhere. With increasing distance to the
  Basis the sensitivity decreases. However, in conclusion it
  can be stated that, because
  of the penalization exponent of 
  \emph{3} only few elements with a high
  sensitivity can remain and 
most of the elements are penalized to void. Furthermore, in case of 
  a too high 
 penal exponent, it may be that, the AMs cannot
  connect to the Basis. Therefore the penalization exponent
  of \emph{3} is recommended for the BZ.\\
  
   The figures
  \ref{fig_comp_re_2} - \ref{fig_comp_re_3} exhibits
  a variation of the sensitivity filter radii wit a common vlue of
  \emph{ep}. The filter defines the
  minimal member size of the AMs. With an increasing 
  filter radius the minimal member size is also increased, which can
  be seen well, by comparing the figures \ref{fig_comp_re_2} and
  \ref{fig_comp_re_3}. Narrow bars in the lower middle area
  are removed and the already existing bars are thickened.
  Another comprehension that can be obtained by
comparing the figures \ref{fig_comp_re_2} - \ref{fig_comp_re_3}
  is that many small struts have a positive impact 
  for high stiffness then few wide struts. This can be stated, since
  the compliance (see table \ref{tabel_compare_parameters}) 
  increases with an increase of the filter radii.\\
 
%_____________Parameter-INvestigation-Table__________________----  
\begin{table}[!h]
\centering
\begin{tabular}{|c|c|c|c|c|c|c|c|}
\hline 
Figure & ep & r\textsubscript{min} = r\textsubscript{b} &
\emph{c} &NI & sec/iteration  \\ 
\hline 
\ref{fig_comp_orig} & 2 & 1.5  & 286.34 & 168 & 33 \\ 
\hline 
\ref{fig_comp_ep_3} & 3 & 1.5 & 284.6 & 198 & 40 \\ 
\hline 
\ref{fig_comp_ep_4} & 4 & 1.5  & 291.78 & 195 & 45 \\ 
\hline 
\ref{fig_comp_ep_5} & 5 & 1.5  & 293.62 & 215 & 48 \\ 
\hline 
\ref{fig_comp_re_2} & 2 & 2  & 237.95 & 102 & 35 \\ 
\hline 
\ref{fig_comp_re_2_5} & 2 & 2.5  & 298.57 & 128 & 41 \\ 
\hline 
\ref{fig_comp_re_3} & 2 & 3  & 313.8 & 74 & 36 \\ 
\hline 
\end{tabular} 
\caption{Shows the impact of the parameter \emph{ep, r\textsubscript{min}
and r\textsubscript{b}}
on the compliance.\\
NI = Number of Iterations.}
\label{tabel_compare_parameters}
\end{table}

\begin{figure}[!h]
\centering
  \includegraphics[width = \textwidth]{path_Image/pngs/Aufgabe_1/Parameter/orig.png}
	\caption{$ep = 2\; r_{min} = r_{b}=1.5$.} 
	\label{fig_comp_orig}
\end{figure}

%________________________________________________________________
\begin{figure}[!h]
\begin{minipage}{0.45\textwidth}
%EPS
\centering
  \includegraphics[width = \textwidth]{path_Image/pngs/Aufgabe_1/Parameter/re15_ep3.png}
	\caption{$ep = 3\; r_{min} = r_{b}=1.5$.} 
	\label{fig_comp_ep_3}
\end{minipage}
\hfill
\begin{minipage}{0.45\textwidth}
\centering
  \includegraphics[width = \textwidth]{path_Image/pngs/Aufgabe_1/Parameter/re15_ep4.png}
	\caption{$ep = 4\; r_{min} = r_{b}=1.5$.} 
	\label{fig_comp_ep_4}
	\end{minipage}
	\vspace{1cm}
	\begin{minipage}{0.45\textwidth}
	\centering
  \includegraphics[width = \textwidth]{path_Image/pngs/Aufgabe_1/Parameter/re15_ep5.png}
	\caption{$ep = 5\; r_{min} = r_{b}=1.5$.} 
	\label{fig_comp_ep_5}
\end{minipage}
\hfill
%_______________________RES__________________________________
\begin{minipage}{0.45\textwidth}
\centering
  \includegraphics[width = \textwidth]{path_Image/pngs/Aufgabe_1/Parameter/re2_ep2.png}
	\caption{$ep = 2\; r_{min} = r_{b}=2$.} 
	\label{fig_comp_re_2}
	\end{minipage}\\
	
	\begin{minipage}{0.45\textwidth}
	\centering
  \includegraphics[width = \textwidth]{path_Image/pngs/Aufgabe_1/Parameter/re25_ep2.png}
	\caption{$ep = 2\; r_{min} = r_{b}=2.5$.} 
	\label{fig_comp_re_2_5}
	\end{minipage}
	\hfill
	\begin{minipage}{0.45\textwidth}
	\centering
  \includegraphics[width = \textwidth]{path_Image/pngs/Aufgabe_1/Parameter/re3_ep2.png}
	\caption{$ep = 2\; r_{min} = r_{b}=3$.} 
	\label{fig_comp_re_3}
\end{minipage}\\
\end{figure}~\\





\chapter{Prohibit connections at edges and corners}
\label{chapter_prohibit_edge}

The 3D Adapt exhibits results with corner- and edge-connections in the BZ (see figure \ref{fig_edge_edge_connec}
and \ref{fig_edge_corner_connec}), which
can be considered as a \emph{3D checker-board}.
Since such connections are not possible to manufacture in reality, they need to be prohibited. 

\begin{figure}[!h]
\begin{minipage}{0.45\textwidth}
\centering
 \includegraphics[width = \textwidth]{path_Image/pngs/Aufgabe_1/edges/edges.png}
	\caption{Edge connection in 3D, gray Basis.} 
	\label{fig_edge_edge_connec}
\end{minipage}
\hfill
\begin{minipage}{0.45\textwidth}
\centering
 \includegraphics[width = \textwidth]{path_Image/pngs/Aufgabe_1/edges/corner.png}
	\caption{Corner connections in 3D, gray Basis.} 
	\label{fig_edge_corner_connec}
\end{minipage}
\end{figure}
 
 To prevent the appearance of edge- and corner-connected elements,
 the search of the BZ needs to be modified. The 2D discretized
 element exhibits 4 Nodes and 4 Edges, this results in 4 possible 
 edge-connections 
 and 4 possible node-connections. Since the Adapt is 
 not programmed for letting the Basis directly
 connected to REs, the BZ needs a reasonable definition.
 In case $ep = 1$, the distance between the border
 elements of the Basis and the REs is only one element, which causes corner-connections (see figure \ref{fig_2dcorner_conections}).
 The Basis is directly connected with the REs at the corners. Therefore the Adapt is not obligated to fill the BZ with material and
 the BZ-elements can remain void without
 impacting the overall stiffness.
 In case $ep = 2$, the distance between the border elements of the Basis and the REs is 2 elements, which forces the Adapt to supply the BZ-elements with material. 
 The reason for this behaviour can be explained by closer examination of the load-case. Since the Basis was generated with the same load-case as the Adapt, the highest Sensitivities are next to the Basis. Furthermore not only the Adapt, but also the 88 lines of code 
are written in a way that elements are formed along
a load path and are connected to each other (requirement of topology
optimization). No elements
should generated, which are not bound to other elements.
The reason, why elements need to be connected to each other
along a load path is, that this increases the stiffness (see figure \ref{fig_2dcorner_conections}).\\

Up to now, the knowledge about the procedure of \textit{ep} was sufficient in
 order to get acceptable results and therefore modifications of the BZ-search were
 not required in 2D. However with the 3D-Adapt two problems, edge-and corner-connection, were encountered, as shown in figure
 \ref{fig_edge_edge_connec} and \ref{fig_edge_corner_connec}.\\


\begin{figure}[!h]
 \centering
 \def\svgwidth{\textwidth}
 \input{path_Image/2d_connections.pdf_tex} 
 \caption{2D connections by varying  \emph{ep}.} % Label für Verweise 
 \label{fig_2dcorner_conections}
 \end{figure}
 
In 3D there are three possible ways to create a connection between two
 elements: edge-connections, node-connections and surface-connections. The
 discretized voxel has 8 nodes, 12 edges and 6 surfaces,which can lead to 8 node-connections, 12 edge-connections and 6 surface-connections
 (figure \ref{fig_edge_3d_search}). This results in 26 possible 
 connections which can be made by a single element.
To prohibit the 26 possible connections between all border elements of the Basis
 and the REs, the 3D BZ-search methods needs to be extended. Instead of
 only considering right, left, upper, lower, elements in $z + 1$ and $z - 1$ elements
 as neighbour elements, which represent
 the 6 potential surface connections
 (see figure \ref{fig_find_BZ_2D}), the diagonal elements
 in $z-1, z$ and $z+1$ need also to be counted as neighbour elements. The figure
 \ref{fig_edge_3d_search} shows the modified 3D BZ-search with $ep =1$, where
 $\rho_e$ represents the current element and all the green, turquoise and
 yellow elements defines the search directions in order to find the 3D BZ.\\
 
  By taking the diagonal elements into account, it is possible to perform the Adapt
 with $ep = 1$ without
 getting a checkerboard type of connection.
 A comparison of the modified 3D BZ-search with $ep = 2$
 and the previous 3D BZ-search method is provided in figure \ref{fig_edges_comp_old_new_ep_1_2}, where the green BZ represents the modified BZ-search with $ep = 1$ and the blue BZ represents the prevoius BZ-search with $ep = 2$.
 Figure \ref{fig_edges_comp_new_ep_1}
 shows the modfied 3D BZ-search with $ep = 1$, where 
 the Basis is gray and the BZ is yellow.
  However, even if $ep = 2$ in 2D works fine, it is not guaranteed that 
 it is going to avoid checkerboard like connections for every problem.
 With a little effort the 
 2D BZ-search was then also modified as shown in 
 figure \ref{fig_edge_2d_search}, where $\rho_e$ represents 
 the current element and all the yellow elements are
 the search area in order to find the BZ.
 With the explained modification, creating a connection
 at corners, even with $ep = 1$ is not
 possible any more. This means border elements 
 of the Basis must be connected to the REs with at least one BZ element in between.\\
 
A comparison between the previous 3D BZ-search and the modified 3D BZ-search can
 be obtained with the figures \ref{fig_edge_3d_BZsearch_modi} to
 \ref{fig_edge_3d_2compare_mod_old}.
 Their results are generated with a mesh resolution 
 of $100 \times 50 \times 8, r_{min} = r_{b} = 1.5, ep = 2$ and $ volfrac = 0.3$.
 

%___________________3D BZ-Search-Modified_________________
\begin{figure}[!h]
\centering
\begin{minipage}{0.8\textwidth}
 \def\svgwidth{\textwidth}
 \input{path_Image/3d_diag.pdf_tex} 
 \caption{Modified BZ-search method for 3D,  \emph{ep} = 1.} % Label für Verweise 
 \label{fig_edge_3d_search}
\end{minipage}\\
\vspace{0.5cm}
%\end{figure}
% \begin{figure}[!h]
 \begin{minipage}{0.45\textwidth}
 \centering
 \includegraphics[width= \textwidth]{path_Image/pngs/Aufgabe_1/edges/compare_old_new_ep_1_2.png}
 	\caption{ Green modified BZ-search with $ep = 1$ and blue previous BZ-search with $ep = 2$ .} 
 	\label{fig_edges_comp_old_new_ep_1_2}
 \end{minipage}
 \hfill
 \begin{minipage}{0.45\textwidth}
 \centering
 \includegraphics[width= \textwidth]{path_Image/pngs/Aufgabe_1/edges/basis_new_ep_1.png}
 	\caption{ Yellow modified BZ-search with $ep = 1$ and gray Basis.}
 	\label{fig_edges_comp_new_ep_1}
 \end{minipage}
 \end{figure}   
 It can be observed that the modified 3D BZ-search exhibits a 
 bigger BZ. Therefore the 2D and 
 3D modified search-methods
 penalize more elements in the BZ than the previous search methods.\\
 %____________________________BOUNDARY-3D_search_______________________________
%________________________________________________________________

\begin{figure}[!h]
\vspace{0.75cm}
\begin{minipage}{0.45\textwidth}
\centering
 \includegraphics[width = \textwidth]{path_Image/pngs/Aufgabe_1/edges/new_BZ_ep_2.png}
	\caption{Modified 3D BZ-search.} 
	\label{fig_edge_3d_BZsearch_modi}
\end{minipage}
\hfill
\begin{minipage}{0.45\textwidth}
\centering
 \includegraphics[width = \textwidth]{path_Image/pngs/Aufgabe_1/edges/old_BZ_ep_2.png}
	\caption{Previous 3D BZ-search.} 
	\label{fig_edge_3d_BZsearch_ohne}
\end{minipage}\\

\vspace{0.75 cm}
\begin{minipage}{0.45\textwidth}
\centering
 \includegraphics[width = \textwidth]{path_Image/pngs/Aufgabe_1/edges/new_BZ_Basis_ep_2.png}
	\caption{Modified 3D BZ-search, gray Basis and yellow BZ.} 
	\label{fig_edge_3d_BZBasissearch_modi}
\end{minipage}
\hfill
\begin{minipage}{0.45\textwidth}
\centering
 \includegraphics[width = \textwidth]{path_Image/pngs/Aufgabe_1/edges/old_BZ_Basis_ep_2.png}
	\caption{Previous 3D BZ-search, gray Basis and yellow BZ.} 
	\label{fig_edge_3d_BZBasissearch_ohne}
\end{minipage}
\end{figure}

\begin{figure}[!h]
\vspace{0.75 cm}
\begin{minipage}{0.45\textwidth}
\centering
 \includegraphics[width = \textwidth]{path_Image/pngs/Aufgabe_1/edges/compare_old_new_ep_2.png}
	\caption{Yellow modified 3D BZ-search and blue previous search.} 
	\label{fig_edge_3d_compare_mod_old}
\end{minipage}
\hfill
\begin{minipage}{0.45\textwidth}
\centering
 \includegraphics[width = \textwidth]{path_Image/pngs/Aufgabe_1/edges/2_compare_old_new_ep_2.png}
	\caption{Difference between modfied and previous 3D BZ-search.} 
	\label{fig_edge_3d_2compare_mod_old}
\end{minipage}
\end{figure}
\newpage
 
 \begin{figure} [!h]
 \vspace{0.75cm}
\centering
\begin{minipage}{0.5\textwidth}
 \def\svgwidth{\textwidth}
 \input{path_Image/BZ_2_2d.pdf_tex} 
 \caption{Modified BZ search method for 2D, $ep = 1$.} % Label für Verweise 
 \label{fig_edge_2d_search}
\end{minipage}
\hfill
\begin{minipage}{0.45\textwidth}
Owing to the different penalization factors which are used for the REs and the BZ,
 the elements are not imposed with equal penalization. The penalization factor for
 the BZ is set to 3, whereas the penalization factor for the REs is set to 2. To
 understand the arising difference by using different penalization factors,
 respectively to see the functions behaviour by means of different penalization
 factor, see figure \ref{fig_SIMP_sceme_modified}. In 
 conclusion, due to the more severe penalization in the BZ 
 and deploying the modified BZ-search-method, not only 
 more elements are found to be penalized, but also the 
 penalization is heavier. Therefore even if the modified BZ-search
 finds a greater BZ than the previous BZ-search, however
 due to the more servere penalization,
 the BZ exhibits less elements with a high density.
 With this combination it is possible 
 to prohibit corner- and edge-connection to occur. 
 The described effect can be observed with a comparison, 
 the figure \ref{fig_edge_3d_micth2_modi} and
 \ref{fig_edge_3d_micth2_ohne} are obtained 
 with the same predefining and the modified load-case
 Cantilever\textsubscript{2} from
 section \ref{section_3d_loadcases}, which stresses each node in depth.
\end{minipage}
\end{figure}


%_____Adapt Mitchell-3D_with|edge without_______________________________
%________________________________________________________________

\begin{figure}[!h]
\begin{minipage}{0.45\textwidth}
\centering
 \includegraphics[width = \textwidth]{path_Image/pngs/Aufgabe_1/edges/adapt_mitchel_02_new.png}
	\caption{Modified 3D BZ-search, gray Basis.} 
	\label{fig_edge_3d_micth_modi}
\end{minipage}
\hfill
\begin{minipage}{0.45\textwidth}
\centering
 \includegraphics[width = \textwidth]{path_Image/pngs/Aufgabe_1/edges/adapt_mitchel_02_old.png}
	\caption{Previous 3D BZ-search, gray Basis.} 
	\label{fig_edge_3d_micth_ohne}
\end{minipage}\\

\vspace{0.75cm}
\begin{minipage}{0.45\textwidth}
\centering
 \includegraphics[width = 0.75\textwidth]{path_Image/pngs/Aufgabe_1/edges/adapt_mitchel_02_new_2.png}
	\caption{Modified 3D BZ-search, gray Basis.} 
	\label{fig_edge_3d_micth2_modi}
\end{minipage}
\hfill
\begin{minipage}{0.45\textwidth}
\centering
 \includegraphics[width = 0.75\textwidth]{path_Image/pngs/Aufgabe_1/edges/adapt_mitchel_02_old_2.png}
	\caption{Previous 3D BZ-search, gray Basis.} 
	\label{fig_edge_3d_micth2_ohne}
\end{minipage}
\end{figure}~\\


\section{Results with modfied 3D BZ-search method}
\label{section_results_modified_3D_BZ_search}
In this section two results with the modified 3D BZ-search method are given. The
 load-cases are Cantilever\textsubscript{2} and the MBB from section
 \ref{section_3d_loadcases}. The figures
 \ref{fig_edge_modi_canti_03} - \ref{fig_edge_modi_mbb_05} are all
 generated with a mesh resolution of $100 \times 50 \times 8, 
 \; r_{min} = r_b = 1.5, 
 \; ep = 2$ and $volfrac = 0.3$. 
 The figures \ref{fig_edge_modi_canti_03} and \ref{fig_edge_modi_canti_05} are
	generated with the loadcase Cantilever\textsubscript{2} and are 
 shown with a display threshold of \emph{0.3} and \emph{0.5}.
 The figures \ref{fig_edge_modi_mbb_03} and 
 \ref{fig_edge_modi_mbb_05} are obtained with the load-case 
 MBB modified and are also shown 
 with a threshold of \emph{0.3} and \emph{0.5}. The
 Basis is gray coloured and the AMs have the colormap
 \emph{Blue to Red Rainbow}.

\begin{figure}[!h]
\vspace{0.75cm}
\begin{minipage}{0.45\textwidth}
\centering
 \includegraphics[width = \textwidth]{path_Image/pngs/Aufgabe_1/edges/mitchell_03.png}
	\caption{Gray Basis, threshold 0.3.} 
	\label{fig_edge_modi_canti_03}
\end{minipage}
\hfill
\begin{minipage}{0.45\textwidth}
\centering
 \includegraphics[width = \textwidth]{path_Image/pngs/Aufgabe_1/edges/mitchell_05.png}
	\caption{Gray Basis, threshold 0.5.} 
	\label{fig_edge_modi_canti_05}
\end{minipage}\\

\vspace{0.75 cm}
\begin{minipage}{0.45\textwidth}
\centering
 \includegraphics[width = \textwidth]{path_Image/pngs/Aufgabe_1/edges/mbb_03.png}
	\caption{Gray Basis, threshold 0.3.} 
	\label{fig_edge_modi_mbb_03}
\end{minipage}
\hfill
\begin{minipage}{0.45\textwidth}
\centering
 \includegraphics[width = \textwidth]{path_Image/pngs/Aufgabe_1/edges/mbb_05.png}
	\caption{Gray Basis, threshold 0.5.} 
	\label{fig_edge_modi_mbb_05}
\end{minipage}
\end{figure}



%
%%%
%%%2) Überprüfung der Leistungsfähigkeit anhand praktischer Beispiele mit vorgegebenen Basis-Strukturen bei erhöhten Lasten 
\chapter{Finite element reanalysis}
\label{chapter_fea_reanalysis}
In order to find out whether the 3D Adapt delivers
 meaningful reinforcements, a FE calculation is carried out
 in this section with 
a commercial software in order to obtain
access to further output values,
such as stress distribution. The selected commercial  
software is \emph{Hyperworks OptiStruct}.
However, the structures, which are added by the Adapt, the Adapt Members (AMs),
are to be understood as meaningful extension, if the AMs
lead to an increase of the stiffness of the entire structure. 
 
\section{Finite element analysis with OptiStruct}
For applying a FE reanalysis with OptiStruct 
 a function called \emph{OSSmooth} is used. 
 It takes two input parameter,
 one is a 
 \emph{.fem}-file and the other is a \emph{.sh}-file.
These files are required in order to load a topology
optimized structure from Matlab into \emph{Hyperworks} by
using \emph{OSSmooth}.
 The \emph{.fem}-file stores information 
 about the discretization e.g. mesh size and type of element to discretize.
 The \emph{.sh}-file is composed of
 the obtained density
 values from the optimization,
 which are between \emph{0} and \emph{1}.
 Each finite element from the \emph{.fem}-file obtains its belonging  density from the \emph{.sh}-file. 
  Furthermore the OSSmooth function offers a selectable OSSmooth threshold
  in order to select the elements which will be considered
  for the interpretation and smoothing og the geometry.\\
 
 Through some investigations it was
 determines that not all value of OSSmooth threshold lead to reasonable results.
 In case of a 
 high OSSmooth threshold, e.g. 0.9, most of the elements 
 are not included in the smoothing process. 
  This leads to elements, which have no connection with other elements, an example can be
  obtained by the figure \ref{fig_2dadapt_05}. 
  These isolated elements lead to an increase in compliance and also distort the grid.
  A OSSmooth threshold $\leq 0.25$ was found to be
  a value, which delivered interpreted geometries
  close to the results obtained from the optimization.\\
  
    As mentioned, the\emph{ .sh}- and\emph{ .fem}-files are required in
     order to make use of the OSSmooth and
  to generate a \emph{.fem}-file. It is necessary to be familiar with
  \emph{OptiStructs} node numbering convention,
  which can be obtained through the figure
   \ref{fig_optistruct_node_1}.
   To make sure that each element in the \emph{.fem} file is assigned the correct density, the command from Listing \ref{lst_sh_data} for Matlab
   is used, where \textit{density\_matrix} can be a 2D or 3D Matlab default matrix which contains the density values. 
 
 \begin{figure} [!h]
 \centering
 \def\svgwidth{0.70\textwidth}
 \input{path_Image/3d_nodes_opti.pdf_tex}
 \caption{OptiStruct node numbering convention.}    % Bildunterschrift 
 \label{fig_optistruct_node_1}          % Label für Verweise 
\end{figure} 
%______________matlab 3d BZ_____________
\begin{lstlisting}[
style=Matlab-editor,
basicstyle=\mlttfamily,
escapechar=`,
label=lst_sh_data,
caption={Helpful command for generating a .sh-file.}
]
sh_densities = (density_matrix(:))';
\end{lstlisting} 

\section{Reanalysis of MBB}
In this section an example for a reanalysis is given. The chosen Adapt for this purpose is generated with modified MBB version, see figure
\ref{fig_load_case_MBB_modified}.
This \textit{ MBB Adapt} is loaded into OptiStruct using OSSmooth and a 
OSSmooth threshold of \emph{0.5} is used.
 However note that a threshold of 0.5
 for most other Adapt results, is going to be too high. A OSSmooth 
 threshold greater than \emph{0.25} will most likely cause a distorted grid.\\

 The element size was selected to 0.5, the feature angle to 30,
  the force equals 1000, the discretization was 
  selected to \textit{mixed}, and the options \textit{connection detect} and \textit{iso surface}
  were chosen. The results can be obtained
  from the figures
 \ref{fig_mbb_re_1} and \ref{fig_mbb_re_2}.
One requirement that the Adapt has to meet is that it should not 
show any local stress peaks. Continuous stress curves, which should be in the middle range of the stress-range, are
 desired. In principle, the
 higher the stress in the Adapt Members (AMs), the greater their
 influence on increasing stiffness, or deceasing the compliance. A high
 stress in the AMs would ensure that
 placement of the AMs increases the stiffness of the whole
 structure. It can be observed that the Basis 
 (gray coloured in figure \ref{fig_edge_modi_mbb_05}) 
is exposed to a lower stress in most places than the AMs. The reason for
this occurrence is that, the added AMs take over a part of the total load. Therefore the Basis has to take up a lower load. Furthermore, it can be 
seen that the stress distribution 
does not show any erratic behaviour, the transitions
 from low to high stress are clear and the the AMs 
 are mostly not stressed to the red
 maximum stress level (see figures \ref{fig_mbb_re_1} and \ref{fig_mbb_re_2}).
  Because the stress in the AMs is clearly visible, the
  Adapt performed a useful material distribution in order to support
  the Basis.

\begin{figure}[!h]
 \centering
 \begin{minipage}{\textwidth}
  \centering
    \includegraphics[width= 0.8 \textwidth]{path_Image/pngs/Aufgabe_2/alles.png}
 	\caption{FEA renalysis of MBB Adapt.} 
 	\label{fig_mbb_re_1}
 \end{minipage}\\
 
 \vspace{0.75cm}
  \begin{minipage}{\textwidth}
 \centering
   \includegraphics[width= 0.6 \textwidth]{path_Image/pngs/Aufgabe_2/Schnitt.png}
 	\caption{FEA renalysis of MBB Adapt, cut view.} 
 	\label{fig_mbb_re_2}
 \end{minipage}

 \end{figure}


%
%%%%3) Erweiterung der Adaptierungs-Methode für leicht unterschiedliche Belastungen zwischen Basis und adaptierte Struktur.   4) Überprüfung der Leistungsfähigkeit der Erweiterung in Punkt 3 
%Explain, why the current 2d adpt does not work for our purpouses
%Explain what we want to have
\chapter{Extension of the 2D Adapt for slightly different loads between Basis and Adapt}
\label{chapter_features_adapt}
 The Adapt was created to strengthen already existing structures, which are called Basis. 
 The Basis must not be changed, only new structures are allowed to be added to ensure  that the Basis gains stiffness.
 The connections of the added structures may 
 only be created at the border of the Basis (in the BZ).  
 Desired are small connections between Basis and the AMs.
  Up to now, the Basis and the AMs are obtained with the same load case.
  This means that the Basis was obtained with a load-case, as shown in
  figure \ref{fig_equal_load} and the Adapt was then
  obtained with the same load case, which only differs in the higher
  requirement as shown in figure
  \ref{fig_equal_load_adapt}.
  If the load case for the Adapt is slightly changed,
  few and large-area connections are created between Basis
   and AMs, which are unwanted and can be
   seen in figure 
   \ref{fig_adapt_large_Connec}.
   The following section introduces and describes further restrictions and parameters 
   that provide more control over the Adapt in
    order to obtain small connections between the Basis and the AMs.
    These restrictions are going to be called \emph{features} and are going to
    be presented in the upcoming sections. The first feature
    (section \ref{section_feature_1})
    enables the control of the tangential thickness of
    a connection between Basis and the AMs.
    The second feature enables to define
    the minimal distance between each found
    connection
    (see section \ref{section_dealing_with_overlap})
    and the third feature enables it
    to remove local reinforcements
    (see section \ref{section_feature_3}).
    Why these features are
    needed and how they work is explained in their
    respective section. After having discussed the features
    a table with the brief explanations of
    the new Adapt specific terms is provided in
    the section
    \ref{section_fundamentals}.
    Finally in
    section \ref{section_fea_features} one
     Adapt result with these features is
    used for a FE reanalysis.\\
    
     In order to explain, what slightly
     different load-case means,
    the figure \ref{fig_slighlty_different_load} 
    gives an overview about the definitions of slightly different load-case,
    where the origin Basis load-case
    can be seen in figure \ref{fig_equal_load}.
    A slightly different Adapt load-case results in an additional force or
    slightly shift of the original load and a new Adapt load-case can
    be obtained from figure \ref{fig_new_load}.

 
\begin{figure}[!h]
\begin{minipage}{0.45\textwidth}
\centering
  \includegraphics[width= \textwidth]{path_Image/pngs/Aufgabe_3/grbegri_1.png}
	\caption{Resulting from optimization.} 
	\label{fig_equal_load}
\end{minipage}
\hfill
\begin{minipage}{0.45\textwidth}
\centering
  \includegraphics[width= \textwidth]
  {path_Image/pngs/Aufgabe_3/grbegri_2.png}
	\caption{Same loadcase, higher requirements.} 
	\label{fig_equal_load_adapt}
\end{minipage}\\

\vspace{0.75cm}
\begin{minipage}{0.45\textwidth}
\centering
  \includegraphics[width= \textwidth]{path_Image/pngs/Aufgabe_3/grbegri_3.png}
	\caption{Slightly different load.} 
	\label{fig_slighlty_different_load}
\end{minipage}
\hfill
\begin{minipage}{0.45\textwidth}
\centering
  \includegraphics[width= \textwidth]{path_Image/pngs/Aufgabe_3/grbegri_4.png}
	\caption{New load.} 
	\label{fig_new_load}
\end{minipage}\\

\vspace{0.75cm}
\centering
\begin{minipage}{0.7\textwidth}
   \includegraphics[width= \textwidth]
   {path_Image/pngs/Aufgabe_3/last_pos_79_ohne.png}
 	\caption {Adapt with slightly different load-case.} 
 	\label{fig_adapt_large_Connec}
 	\end{minipage}
\end{figure}

    
\section{Feature 1: control of tangential length connection}
\label{section_feature_1}

    The goal of the Adapt is to generate small connections
    between the Basis and the Adapt while increasing
    the stiffness of the whole structure. To create thin
    connections, it is necessary to control both the normal
     and tangential width of the
    connection. The figure \ref{fig_ep_cz_tangen} shows a gray coloured Basis
    and the yellow zone represents elements of a generated
    connection. Furthermore, it  
    can be observed that the parameter \textit{ep} defines the 
     normal length and the parameter \textit{CZ\textsubscript{mind}}
      controls the tangential length of the connection.
      \newpage
 
 \begin{figure} [!h]
 \begin{minipage}{0.45 \textwidth}
The idea to define a connections tangential thickness
is shown in the figure 
\ref{fig_penal_domain}.
 Within  \textit{CZ\textsubscript{mind}}, which
 is the parameter that defines the
 maximal thickness of a connection,
 the sensitivity of the elements
 will not be reduced or the elements will not be penalized.
  However, every element between \textit{CZ\textsubscript{mind}} and
    \textit{CZ\textsubscript{max}} will be punished. Note that only
    elements
     whose centre lies within \textit{CZ\textsubscript{mind}} will be penalized (compare with figure \ref{fig_rmin}).
     This ensures that no connection is larger than \textit{CZ\textsubscript{mind}}.
     In order to be able to apply this process, it is first necessary to find out where the individual connections arise. Each connection consist of elements and
      because the Adapt consist of multiple connections, the 
      term \emph{Connection Zone Group} is introduced (CZG).
      A CZG can only contain elements, which is
      located in the BZ, because the BZ is the zone in which a
      possible connection can be made. However a CZG only contains
      those elements from the BZ, which lead to a connection.
 \end{minipage}
 \hfill
  \begin{minipage}{0.45 \textwidth}
  \centering
 \def\svgwidth{\textwidth}
 \input{path_Image/cz_ep_tangential.pdf_tex}
 \caption{Gray Basis, yellow CZG elements, $ep = 3, CZ_{mind} = 4$.}    % Bildunterschrift 
 \label{fig_ep_cz_tangen}          % Label für Verweise 
 \end{minipage}
\end{figure} 
%
 \begin{figure} [!h]
 \begin{minipage}{0.45 \textwidth}
  The
      sum of all the CZGs is going to be called
      the Connection Zone (CZ), which also has
      to be located within the BZ. \\
      
             After two technical terms have 
      been explained, it shall be mentioned
      that the CZGs are extracted after 10 iterations,
      in order to give enough time for
      the CZGs to arise, before identifying them and
      manipulating their
      properties.
      With that being said,
      the procedure of finding CZGs shall be introduced.
            Since the CZ lies within the BZ, it is only required to consider the BZ elements during the search.
       Furthermore a \textit{CZ-threshold} is employed: if the
        considered BZ element fulfils the \textit{CZ-threshold} it
         is taken as a CZ-element.
         Next it must be checked, if the considered
          element is a neighbour of an element that is already in a CZG.
            In case it is the first element, which was found, a new CZG
            for himmust be generated and in every 
 \end{minipage}
 \hfill
  \begin{minipage}{0.45 \textwidth}
 \centering
 \def\svgwidth{\textwidth}
 \input{path_Image/cz_mind_max.pdf_tex}
 \caption{Orange CZ\textsubscript{max}, purple CZ\textsubscript{mind} and
 gray penal area \emph{AVA}.}    % Bildunterschrift 
 \label{fig_penal_domain}          % Label für Verweise 
 \end{minipage}
\end{figure}
 other case, the elements must be examined.
 This examination
      is 
      as already mentioned
      based on neighbourhood: it verifies whether the current element is a 
      neighbour element of the already found elements, which are stored in
      CZGs.
       By taking advantage of the numbering convention
       (square and regular mesh)
        of the elements, which is from left to right and top to bottom, only 
        maximal 4 search directions need to be taken into 
       account. The search directions can be obtained in figure \ref{fig_look_around},
       where the gray
       element can be taken as the current element in the BZ
       and the green arrows as the 4 search directions-neighbour-elements.
       Each of the neighbour element is obligated to be 
       within the BZ and  meet the \textit{CZ-threshold}.
       
       The reason why only maximal 
       4 search directions need to be taken into account,
       is that two for loops are used, the first one changes the x-number 
       of the element
       and the second loop changes the y-number of the considered element
       in the matrix. 
       Because the x-loop comes first and the y-loop comes second, the column 
       stays constant and after all rows with the constant column
       were examined the column will change to $column_{new} = column +1$.
       Thus, the elements without arrows inside
       (figure \ref{fig_look_around})
        have not been checked ye and rhus can not be
        n any CZG and therefore 
       there is no need for a neighbourhood-examination. In case the
       current element was assigned to a group with the
       first direction of the neighbourhood search, the remaining
       search directions are not going to be considered any more
       in order to save computational time. Neighbour groups will
       be merged subsequently. 
            \begin{figure}[!h]
 \centering
 \def\svgwidth{0.45\textwidth}
 \input{path_Image/look_around.pdf_tex}
 \caption{Search directions for the CZG examination }    % Bildunterschrift 
 \label{fig_look_around}          % Label für Verweise 
      \end{figure}

       \subsection{Geometrical and weighted centre}
       \label{subsection_geometrical_weighted}
       After it is possible to save the CZGs, the centre of the CZG needs to be 
       located
       in order to define the maximal tangential thickness 
       CZ\textsubscript{mind}.
       The centre of a 
       CZG is the point, where the two radii
       CZ\textsubscript{mind} and CZ\textsubscript{max} are
       generated in order to penalize all the elements
       within CZ\textsubscript{max} and outside CZ\textsubscript{}mind
       ( \emph{AVA}, see figure \ref{fig_penal_domain}).
       There are two possible ways, which are going to be described.\\
       
       The first method in order to calculate the centre of
       each CZG can be expressed as 
       the geometrical centre. To calculate the geometrical centre, two mean 
       values are required. Each CZG consists of elements, which
       have x and y-coordinates and with this coordinates the x-mean
       and y-mean can be obtained.
       \begin{align*}
       x_{mean} = \dfrac{1}{N} \sum_{i=1}^N x_i, \quad 
       y_{mean} = \dfrac{1}{N} \sum_{i=1}^N y_i,
       \end{align*}
       where \textbf{$N$} is the number of the elements which are stored
       in the considered CZG\textsubscript{i}, 
       \textbf{$x_i$} and \textbf{$y_i$} are the x- and 
       y-coordinates of the CZG elements.\\
       
       The second method is a weighted average method and can be expressed
       as follows:
          \begin{align*}
       x_{mean} = \dfrac{1}{\sum_{i=1}^N \rho_i} \sum_{i=1}^N x_i  \rho_i, \quad 
       y_{mean} = \dfrac{1}{\sum_{i=1}^N \rho_i} \sum_{i=1}^N y_i \rho_i,
       \end{align*}
		where \textbf{$N$} is the number of the elements which are stored
       in the considered CZG\textsubscript{i},
        \textbf{$x_i$} and \textbf{$y_i$} are the x- and 
       y-coordinates of the CZG elements and \textbf{$\rho_i$} are their
       densities.
       The advantage of the weighted average method is, that the elements 
       densities are considered in order to calculate the mean x- and y-coordinates.
       Therefore the weighted average method allows to come
       closer to the centre of the elements with a higher density. The
       elements with a higher density have a bigger impact on increasing
       the stiffness then the low densities. Since the
       maximal tangential thickness is defined with CZ\textsubscript{mind},
       and the elements outside of CZ\textsubscript{mind} and
       inside CZ\textsubscript{max} are penalized,
       the closer CZ\textsubscript{mind} gets to the centre of the
       densities with higher values the better the impact
       on increasing the stiffness becomes. 
      
       \subsection{Penalization of elements within\textit{ CZ\textsubscript{mind}} and \textit{CZ\textsubscript{max}}}
       \label{subsection_distance_i}
       
       After having found the CZGs and their centre ($x_{mean}$ and $y_{mean}$)
       all elements which
       lie between \textit{CZ\textsubscript{mind}} and
        \textit{CZ\textsubscript{max}}
        or within AVA
       must be penalized (see figure \ref{fig_penal_domain}. The
       reason why there is no difference between penalizing
       between \textit{CZ\textsubscript{mind}} and
        \textit{CZ\textsubscript{max}} and AVA is based
        on the definition of
        $CZ_{max} = AVA + CZ_{mind}$, where
        CZ\textsubscript{mind} and AVA are
        parameter which are defined by the operator of
        the Adapt and CZ\textsubscript{max} is calculated
        then by a summation. In case of $AVA = 0$,
        $CZ_{max} = CZ_{mind}$ and then the penalization area
        would not exist (see figure
        \ref{fig_penal_domain}).\\

       However, to reduce the sensitivities
       within AVA, the
       penalization task needs to be 
       accomplished after the sensitivity analysis and
       it must be performed 
       for each CZG with its respective x- and y-mean
        coordinates. In order to penalize the AVA 
        elements
		the distance ($dist_i$) between each element in
		 CZG and the centre$(x_{mean}|y_{mean})$ of the CZG needs
		to be $CZ_{mind} < dist_i \leq CZ_{max}$.
		
		\begin{align*}
		dist_i = \sqrt{(x_{mean_i}-x_i)^2+(y_{mean_i}-y_i)^2} \quad,
		\end{align*}
where \textbf{$x_{mean_{i}}$} and\textbf{ $y_{mean_{i}}$} are the centre coordinates 
in x- respectively
y-direction of the current CZG\textsubscript{i},  \textbf{$x_i$ }
and \textbf{$y_i$} are the coordinates in x- and y-direction of the
 current CZG\textsubscript{i} element.\\
 
 To penalize the AVA elements $(CZ_{mind} < dist_i \leq CZ_{max})$ the penalization function from
\cite{Dienemann.2018} is deployed.

\begin{align*}
P_i(d_i) = \dfrac{1- \tanh(a \; [2 \; d_i / CZ_{mind} -1)}{1+ \tanh(a)} \quad,
\end{align*}

where \textbf{$d_i = \dfrac{CZ_{mind}}{2}$}
and a defines the discreetness of the function. A low discreetness leads to
insufficient penalization. A value of $a = 4$ will be used
as recommended in \cite{Dienemann.2018}.\\


Analogously to finding elements between \textit{CZ\textsubscript{mind} }
and \textit{CZ\textsubscript{max}}, elements 
within \textit{CZ\textsubscript{mind}} can be
located with $dist_i \leq  CZ_{mind}$ with
a if statement that checks, whether the distance $d_i < CZ_{mind}$.
This can be useful, when it is desired to increase the sensitivity of
the mentioned elements.
However, there was no need to implement the former yet.

\section{Feature 2: Definition of minimal distance between Connection Zone Groups (CZGs)}
\label{section_dealing_with_overlap}
In this section it is going to be explained how to control
the minimal distance between all CZGs. The idea is it to
find overlapping CZGs and in case of an overlapping, the
overlapping CZGs are combined to one CZG. In
case the CZGs do not overlap, the distance between the
CZGs is greater then the minimal distance between 
CZGs parameter.\\

It can be observed that some overlapping CZGs can
cause unintentional penalization.
In general a distinction between two types of 
overlapping can be made. To provide a good understanding, the figure \ref{fig_CZG_overlapping_1}
  shows two CZGs, which do not overlap. 
  The figure \ref{fig_CZG_overlapping_2} shows the first type
  of an overlapping of two CZGs. The first case results in penalizing the elements twice, once by the
left CZG and once by the right CZG within AVA.
\vspace{0.2cm}
 \begin{figure} [!h]
 \def\svgwidth{\textwidth}
 \input{path_Image/overlapping_1.pdf_tex}
 \caption{Overlapping of CZGs type 1.}    % Bildunterschrift 
 \label{fig_CZG_overlapping_1}          % Label für Verweise 
\end{figure} 
%
 \begin{figure} [!h]
 \centering
 \def\svgwidth{\textwidth}
 \input{path_Image/overlapping_2.pdf_tex}
 \caption{Two CZGs with no overlapping.}    % Bildunterschrift 
 \label{fig_CZG_overlapping_2}          % Label für Verweise 
\end{figure} 
 \begin{figure} [!h]
 \centering
 \def\svgwidth{\textwidth}
 \input{path_Image/overlapping_3.pdf_tex}
 \caption{Overlapping of CZGs type 2 and calcualting new centre of two overlapped CZGs.}    % Bildunterschrift 
 \label{fig_CZG_overlapping_3}          % Label für Verweise 
\end{figure} 
However, since no \textit{CZ\textsubscript{mind}} element is punished,
 the maximum tangential thickness of the CZGs
is not changed. The AVA elements of each CZG is 
penalized, in case of a complete identical AVA
as shown in figure \ref{fig_CZG_overlapping_2}.
The complete AVA
is the whole overlapping area, which is then penalized twice
and in case only some parts of the AVAs overlaps 
only these overlapping parts are penalized twice,
the rest of the AVAs is penalized once.
The second type of overlapping CZG occurs, when the \textit{CZ\textsubscript{max}} of at least two CZGs overlaps,
which would penalize elements within \textit{CZ\textsubscript{mind}} (see figure \ref{fig_CZG_overlapping_3}). Penalizing \textit{CZ\textsubscript{mind}} elements affects
the maximum tangential thickness of the CZG. In this way the parameter \textit{CZ\textsubscript{mind}} is not respected
any more and does not describe the maximum tangential thickness of a CZG. In order to avoid
such an occurrence, the distance ($dist_{cz}$) of all CZGs is calculated. If
$dist_i < 2 \; CZ_{mind} +AVA $
at least two \textit{CZ\textsubscript{max}} are overlapping,
which results in penalizing \textit{CZ\textsubscript{mind}} elements. Therefore if
$dist_i < 2 \; CZ_{mind} +AVA$,
new $x_{mean}$ and $y_{mean}$ of the overlapping CZGs are calculated.
Because of the mentioned $dist_i$-condition
it is possible to define a minimal distance
between CZGs. The defining parameter is 
  This \emph{AVA}, which lets define
  \emph{AVA} not only the penalization area,
  but also the minimal distance between CZGs.
  The calculation of the centre
can be accomplished by means of the geometrical or weighted centre,
 introduced in subsection \ref{subsection_geometrical_weighted}.
Again, because of the already 
named benefits the calculation is suggested with the weighted centre.\\

Due to its simple comprehension, the figure 
\ref{fig_CZG_overlapping_3} shows a geometrical centre calculation of a new
CZG\textsubscript{mind\textsubscript{1\_2}} and a
 new CZG\textsubscript{max\textsubscript{1\_2}} of two overlapped CZGs. The new CZG\textsubscript{1\_2} is replaced in the
working storage by the two overlapped CZGs,
CZG1 and CZG2 in order to save memory space,
keep a better overview of the CZGs and it also facilitates the later required modifications on the
BZ-filter, which will be explained in section 
\ref{section_cgz_bz_filter}.
 The new CZG\textsubscript{1\_2} contains all the informations
of the two overlapped CZGS, CZG\textsubscript{1}
 and CZG\textsubscript{2}, with the difference that it is one CZG and its centre
of gravity consists of the coordinates of the overlapping CZGs. \\

The figure \ref{fig_matlab_1_CZG} shows a struct from
Matlab, which stores all the found CZGs with
 the names of the CZGs (\textit{Gruppe\_\textsubscript{i}}),
  the x-, y-coordinates and 
 the densities of their elements as a vector and the center 
of gravity in x and y-direction of each found CZG as \emph{Doubles}
named \textit{X\_mean}
and \textit{Y\_mean}. These informations are sufficient in order to
collect all essential information about the elements
in a CZG\textsubscript{i}. The figure \ref{fig_matlab_2_CZG} shows
the same struct after finding overlapping CZGs: in this case
three overlapping CZGS are found. Therefore CZGS, CZG\textsubscript{4},
CZG\textsubscript{5} and CZG\textsubscript{6} were replaced by the 
CZG\textsubscript{4\_5\_6}.

\begin{figure}[!h]
\begin{minipage}{0.45\textwidth}
\centering
  \includegraphics[width= \textwidth]
  {path_Image/pngs/Aufgabe_3/matlab_1.png}
	\caption{List of found CZGs.} 
	\label{fig_matlab_1_CZG}
\end{minipage}
\hfill
\begin{minipage}{0.45\textwidth}
\centering
  \includegraphics[width= \textwidth]
  {path_Image/pngs/Aufgabe_3/matlab_2_overlapped.png}
	\caption{List of found CZGs after overlapping examination.} 
	\label{fig_matlab_2_CZG}
\end{minipage}
\end{figure} 
%
In order to get a better overview of
the progress Matlabs plot function
in 2D is used.
This allows to project 
\textit{CZ\textsubscript{mind}} and 
\textit{CZ\textsubscript{max}} on the Adapt at the centre of gravity of
 CZGs, which
is shown in figure \ref{fig_circles_CZG}. The dark gray coloured structure
represents the Basis, the lighter gray color stands for the BZ
($\rho_e < 0.1$), the Adapt without
the Basis is rainbow coloured, where dark blue stands for
void material.

\begin{figure}[!h]
\centering
  \includegraphics[width = 0.75\textwidth]
  {path_Image/pngs/Aufgabe_3/czgs_bsp_1.png}
	\caption{Circles for \textit{CZ\textsubscript{mind}} and 
	\textit{CZ\textsubscript{max}}.} 
	\label{fig_circles_CZG}
\end{figure}

\section{ Feature 3: Prohibition of local reinforcements}
\label{section_feature_3}
The Adapt is coded in order to create connections
between the Basis and the AMs.
 As a further restriction, the Adapt will be given to apply material 
 within the BZ, only if it contributes to a \emph{used} connection.
 A \emph{used} connection does not cause local
 reinforcements and is given when the Basis is connected to
 the REs through the BZ, wherein the REs have a
 struts like developed AMs structure.
  The aim is to allow material to be applied to the BZ,
  if there 
  is a  struts like developed AM structure in the vicinity of the found CZG.
  The reason for the extension of this feature is:
  in practise it would be an additional effort to
  manufacture local reinforcements.
  In case there is no formed AM structure, this would only result
  into thickening of the Basis. The term \emph{unused connection}
  is going to be used as local reinforcement.\\
  
  In order to find an unused connection the centre of the
  CZGs is required. This serves as the starting point of the examination,
  because it makes investigations within \textit{AVA} easier.
  Within \textit{AVA} of the CZGs each element must
  be tested, whether it belongs to the REs and in case it does, 
 the density of the matching element needs to be compared
 with a user predefined RE-threshold. If the densities of the matched 
 elements are lower then the RE-threshold, then all the elements inside
 the CZG or the CZG itself needs to be penalized in order
 to prohibit its development. Because \textit{AVA} is a small area, it is 
 sufficient to find one single element, which does not fulfil
 the RE-threshold in order to penalize the CZG.
To make certain that the penalization is 
sufficient, the penalization factor is held variable. With some
if statement it is possible to implement a method, which
can perform the penalization after a number of iterations and
change the beginning magnitude of the penalization repeatedly.
Such if statements are beneficial when the Adapt
faces
unused connections in which elements exhibits high sensitivities.
Therefore high and variable increasable penalisation factors
are required in order to remove unused connections.
The beginning penalisation
of the sensitivity is set to 0.5, which is
performed after having found CZGs. As mentioned the search for
CZGs is performed from the 10th iteration and after each 5 iterations
the RE penalization factor is increased by the factor \emph{0.5} or 
the penalization is increased by the factor 2
(decreasing the 
sensitivity results in a increasing punishment).\\

The figures \ref{fig_nr_iteration_13} to \ref{fig_nr_iteration_16} shows
a topology optimization by the Adapt employing the explained 
method in order to prohibit the occurrence of unused CZGs. 
The small circles represents \textit{CZ\textsubscript{mind}}, the
big circles are \textit{CZ\textsubscript{max}}.
The figures shows that the unused connections 
or the CZGS are removed
in iteration 16 (figure \ref{fig_nr_iteration_16}).
 
\begin{figure}[!h]
\begin{minipage}{0.48\textwidth}
\centering
  \includegraphics[width= \textwidth]{path_Image/pngs/Aufgabe_3/13.png}
	\caption{Iteration number 13.} 
	\label{fig_nr_iteration_13}
\end{minipage}
\hfill
\begin{minipage}{0.48\textwidth}
\centering
  \includegraphics[width= \textwidth]{path_Image/pngs/Aufgabe_3/14.png}
	\caption{Iteration number 14.} 
	\label{fig_nr_iteration_14}
\end{minipage}\\

\vspace{0.3cm}
\begin{minipage}{0.48\textwidth}
\centering
  \includegraphics[width= \textwidth]{path_Image/pngs/Aufgabe_3/14.png}
	\caption{Iteration number 15.} 
	\label{fig_nr_iteration_15}
\end{minipage}
\hfill
\begin{minipage}{0.48\textwidth}
\centering
  \includegraphics[width= \textwidth]{path_Image/pngs/Aufgabe_3/16.png}
	\caption{Iteration number 16.} 
	\label{fig_nr_iteration_16}
\end{minipage}
\end{figure}

\section{Modifications on the BZ-filter}
\label{section_cgz_bz_filter}
Since all the CZGs are located in the BZ, the BZ-filter needs to be modified.
The important parameter for the filtering is 
\textit{CZ\textsubscript{mind}}: it defines the maximal
tangential thickness of the CZGs. As long as $CZ_{mind} \geq ep$ 
the rest of BZ elements, which are within AVA or even further away
 can be neglected. It is also essential not to filter the 
 CZGs which consist a unused connection. This can be performed
 by removing these CZGs or explicitly not including them in
 the filtering process. Furthermore a CZG can exhibit elements, which
 are located within \textit{CZ\textsubscript{mind}}, however,
 are not stored in the CZG. The reason is that, these missing elements
 do not fulfil the CZ-threshold.\\
 To avoid a checkerboard,
    each element within \textit{CZ\textsubscript{mind}} is filtered.
   To find these elements, the distance between the centre of gravity
    of the CZG\textsubscript{i} and its elements is required. 
    The calculation already has been explained in 
    subsection \ref{subsection_distance_i}.
   
   
\section{Comparison results with and without features}
\label{section_compare_results_with_without_feature}

This section shows results with 2D adapts
 with and without the implemented features.
 In particular, it is examined what the Adapt
  performs when its load differs from that of the Basis. 
  For this purpose two versions were presented, 
  the Adapt version with the 
  features and the Adapt version without the features. 
 In order to compare the Adapt versions
  all parameters, such as \textit{ep, r\textsubscript{min}, r\textsubscript{b}},
 the  mesh resolution and the load-case, 
  are equal for the examination with and without the 
  features. However, it is important to understand,
  that the load cases are only equally selected 
  for the two versions of the Adapts, not 
  for the Basis and the Adapt, since the main
  object of this section is to see, what
  the Adapt performs in case its load
  case is not equal with the
  Basis load case. However, the Basis is equal
  for all the Adapt results.\\
  
 The figures on the left show the results without
   the integration of the features and the figures on the
    right show the results with the integration of the features.
    The Basis is gray, the blue elements
    represents void material and the AMs are
    rainbow coloured.\\
    
Figure \ref{fig_comp_fea_l10_o} and \ref{fig_comp_fea_l10_m}: 
 the Basis and the Adapts are obtained with the 
same load in y direction (see figure \ref{fig_loadcase_canti_2d}),
 in x direction, where x represents the
 \emph{horizontal} direction and y stands for
 the \emph{vertical} direction. 
 The Adapts, however, receive a different load. 
The load in x direction is defined as +10\% (tensile load) of the x load,
which results in an increase of the resulting load.
The left figure \ref{fig_comp_fea_l10_o}  shows the result
 without the features and the right figure 
 \ref{fig_comp_fea_l10_m} with the features.\\
 
The sole difference 
between figures \ref{fig_comp_fea_l10_o} and 
\ref{fig_comp_fea_l10_m} to
the figures \ref{fig_comp_fea_ln10_o} and \ref{fig_comp_fea_ln10_m} 
is that the load in the figures
\ref{fig_comp_fea_ln10_o} and
\ref{fig_comp_fea_ln10_m}
 is selected in negative x direction.
 The figures \ref{fig_comp_fea_l50_o} and
\ref{fig_comp_fea_l50_m} are obtained with nearly the same
conditions as in the introductory load case
(figures \ref{fig_comp_fea_l10_o} 
and \ref{fig_comp_fea_l10_m},
 however, the only
difference is that the load in x direction is defined as 50\% of the y load
(tensile load).
 The figures \ref{fig_comp_fea_ln50_o} and
\ref{fig_comp_fea_ln50_m} show the latter Adapts (50\%)
with a negative x-direction. \\

The figures \ref{fig_comp_fea_l79_o} to \ref{fig_comp_fea_l81_m}
 present the results for slightly changing the 
 load position, but with the same load magnitude. The results from figure
 \ref{fig_comp_fea_l79_o} and \ref{fig_comp_fea_l79_m} are obtained 
 with the a load position of $0.79 L$ instead of $0.8 L$,
 which is the load position for the Basis
  (see figure \ref{fig_loadcase_canti_2d}) and the
 figures \ref{fig_comp_fea_l81_o} and
 \ref{fig_comp_fea_l81_m} are obtained with
 a load position of $0.81 L$. This change of the position
 is performed in x-direction.\\
 
It can be observed that the features fulfil their purpose.
The keeping of the maximum tangential
connection thickness can be seen especially well
when comparing the figures \ref{fig_comp_fea_l10_o}
and \ref{fig_comp_fea_l10_m} .
 At the upper left area, can be seen how
 the area of the connection from the figure
 \ref{fig_comp_fea_l10_o}
decreases in figure
\ref{fig_comp_fea_l10_m}.
The work of the third feature,
removing local reinforcements can be
seen when comparing, the two figures
\ref{fig_comp_fea_ln10_o} and
\ref{fig_comp_fea_ln10_m}.
The upper left local reinforcement in
figure \ref{fig_comp_fea_ln10_o}
disappears in figure \ref{fig_comp_fea_ln10_m}, because of the
usage of the features. The same observations are made for the
other examples\\

\vspace{0.75cm}

 \begin{figure}[!h]
 \begin{minipage}{0.45\textwidth}
 \centering
   \includegraphics[width= \textwidth]
   {path_Image/pngs/Aufgabe_3/last_10_ohne.png}
 	\caption{Without features, +10\%.} 
 	\label{fig_comp_fea_l10_o}
 \end{minipage}
 \hfill
 \begin{minipage}{0.45\textwidth}
 \centering
   \includegraphics[width= \textwidth]
   {path_Image/pngs/Aufgabe_3/last_10_mit.png}
 	\caption{With features, +10\%.} 
 	\label{fig_comp_fea_l10_m}
 \end{minipage}\\
 
 \vspace{0.75 cm}
  \begin{minipage}{0.45\textwidth}
 \centering
   \includegraphics[width= \textwidth]
   {path_Image/pngs/Aufgabe_3/last_n10_ohne.png}
 	\caption{Without features, -10\%.} 
 	\label{fig_comp_fea_ln10_o}
 \end{minipage}
 \hfill
 \begin{minipage}{0.45\textwidth}
 \centering
   \includegraphics[width= \textwidth]
   {path_Image/pngs/Aufgabe_3/last_n10_mit.png}
 	\caption{With features, -10\%.} 
 	\label{fig_comp_fea_ln10_m}
 \end{minipage}
 \end{figure}
 \begin{figure}[!h]
 \vspace{0.75 cm}
  \begin{minipage}{0.45\textwidth}
 \centering
   \includegraphics[width= \textwidth]
   {path_Image/pngs/Aufgabe_3/last_50_ohne.png}
 	\caption{Without features, +50\%.} 
 	\label{fig_comp_fea_l50_o}
 \end{minipage}
 \hfill
 \begin{minipage}{0.45\textwidth}
 \centering
   \includegraphics[width= \textwidth]
   {path_Image/pngs/Aufgabe_3/last_50_mit.png}
 	\caption{With features, +50\%.} 
 	\label{fig_comp_fea_l50_m}
 \end{minipage}\\

  \vspace{0.75 cm}
    \begin{minipage}{0.45\textwidth}
 \centering
   \includegraphics[width= \textwidth]
   {path_Image/pngs/Aufgabe_3/last_n50_ohne.png}
 	\caption{Without features, -50\%.} 
 	\label{fig_comp_fea_ln50_o}
 \end{minipage}
 \hfill
 \begin{minipage}{0.45\textwidth}
 \centering
   \includegraphics[width= \textwidth]
   {path_Image/pngs/Aufgabe_3/last_n50_mit.png}
 	\caption{With features, -50\%.} 
 	\label{fig_comp_fea_ln50_m}
 \end{minipage}\\
  \end{figure}~\\
  
  
  \begin{figure} [!h]
  \begin{minipage}{0.45\textwidth}
 \centering
   \includegraphics[width= \textwidth]
   {path_Image/pngs/Aufgabe_3/last_pos_79_ohne.png}
 	\caption {Without features.} 
 	\label{fig_comp_fea_l79_o}
 \end{minipage}
 \hfill
 \begin{minipage}{0.45\textwidth}
 \centering
   \includegraphics[width= \textwidth]
   {path_Image/pngs/Aufgabe_3/last_pos_79_mit.png}
 	\caption{With features.} 
 	\label{fig_comp_fea_l79_m}
 \end{minipage}\\
 
   \vspace{0.75 cm}
  \begin{minipage}{0.45\textwidth}
 \centering
   \includegraphics[width= \textwidth]
   {path_Image/pngs/Aufgabe_3/last_pos_81_ohne.png}
 	\caption{Without features.} 
 	\label{fig_comp_fea_l81_o}
 \end{minipage}
 \hfill
 \begin{minipage}{0.45\textwidth}
 \centering
   \includegraphics[width= \textwidth]
   {path_Image/pngs/Aufgabe_3/last_pos_81_mit.png}
 	\caption{With features.} 
 	\label{fig_comp_fea_l81_m}
 \end{minipage}
 \end{figure}

 
 \section{Fundamental terms}
 \label{section_fundamentals}
Since the previous presented 
three new
features required some new technical words, the table \ref{table_third_task}
shall serve as a short and fast Information source. 
%_____________________TABELLE 1 ________________________________________
\begingroup
\renewcommand{\arraystretch}{2} % Default value: 1
 \begin{longtable}{L{0.2\textwidth} L{0.8\textwidth}}
 \caption{Brief explanations about often used terms in this chapter.}\\
  \hline 
  \rowcolor{Green}
 \multicolumn{1}{c}{Term}  &  \multicolumn{1}{c}{Explanation}  \\ 
 \hline
   \rowcolor{Gray1}  
 Connection-Zone (CZ) & The CZ is the domain in which the Adapt generates
 connections to the Basis. It is located within the BZ and 
 represents the sum of all the generated connections.
  In order to be regarded as a CZ-element, it must must 
  fulfil the CZ-threshold. Each CZ-element is also a BZ-element, but
  not vice versa.
  Note, 
 BZ elements can have a density smaller than the CZ-threshold \\ 
\hline
 CZ-threshold &  The CZ-threshold is a density-threshold
 parameter, which defines, whether the
 elements in the BZ can be considered as CZ elements or not.\\
 \hline 
    \rowcolor{Gray1}  
 RE-threshold & The RE-threshold is a density-threshold
 parameter, which defines, whether 
 the REs exhibit a local reinforcement, see section
 \ref{section_feature_3}.\\
 \hline 
 Connection-Zone-Group (CZG) & The Adapt generates multiple
 connections in order to connect the AMs to the Basis. In order to
 be able to modify single connections, CZGs are introduced. 
 A CZG is one single connection, which consists of all the elements
 which are in the CZ and
are required to build one connection between
 the Basis and the AMs.  The yellow elements or the yellow zone in figure 
 \ref{fig_ep_cz_tangen} represents a CZG.
 The sum of
 all the CZGs is the CZ.\\

 \hline
     \rowcolor{Gray1}  
CZ\textsubscript{mind} and CZ\textsubscript{max} & \textit{CZ\textsubscript{mind}}
defines the maximal tangential thickness of the connection between AMs and Basis. $CZ_{max} = AVA + CZ_{mind}$ (see figure \ref{fig_penal_domain})
is required in order to penalize all the 
elements between \textit{CZ\textsubscript{max}} and 
\textit{CZ\textsubscript{mind}} or
within \textit{AVA}.
With the two parameter \textit{CZ\textsubscript{max}} and \textit{CZ\textsubscript{mind}} it is possible to control the tangential thickness of 
the connection between AMs and Basis, where
CZ\textsubscript{mind} and AVA are user provided parameter and
CZ\textsubscript{max} is calculated with $CZ_{max} = AVA + CZ_{mind}$ .

It is important to choose $CZ_{mind} \geq ep$, otherwise
BZ elements which
are normal to the Basis and  
 form the connection between
Basis, BZ and AMs, are penalized unintentionally.

The figure \ref{fig_ep_geq_ep} shows a gray coloured Basis,
yellow  elements, which are normal to the Basis and
lies between the Basis and the red AM elements. 
$CZ_{mind} < ep$ and the dark green  coloured area presents the 
penalization area. In case $CZ_{mind} < ep$ a connection
between Basis, BZ elements and AM elements might
be disturbed with void elements, because of the
penalization. Therefore the following must
chosen: $CZ_{mind} \geq ep$.\\

     \rowcolor{Gray1}  
 CZ\textsubscript{mind} and CZ\textsubscript{max} & Without \textit{CZ\textsubscript{max}}  the elements around 
\textit{CZ\textsubscript{mind}} can not be penalized (gray colored in
figure \ref{fig_penal_domain}),
which results in none control over the tangential connection thickness.

Note: the normal thickness or length of the
a CZG is defined by the parameter \textit{ep}. 
\\
 \hline 
AVA & \textit{AVA} (see figure \ref{fig_penal_domain})
 represents the penalization area, within \textit{AVA}, elements
from CZGs will be penalized in order to achieve demanded maximal
tangential CZ thickness \textit{CZ\textsubscript{mind}}. It
is a user predefined parameter and since 
the $dist_i $-condition from 
section \ref{section_dealing_with_overlap}, is defined as 
$dist_i < 2 \; CZ_{mind} +AVA $
\textit{AVA} controls the minimal distance between all
overlapped CZGs.
The effect of \textit{AVA} can be seen in figure 
\ref{fig_CZG_overlapping_2} .  \\
\hline
\label{table_third_task}
\end{longtable}
\endgroup 

\begin{figure}[!h]
  \centering
 \def\svgwidth{0.45\textwidth}
 \input{path_Image/cz_geq_ep.pdf_tex}
 \caption{ $ CZ_{mind} < ep$.}    % Bildunterschrift 
 \label{fig_ep_geq_ep}          % Label für Verweise 
\end{figure}

%_______________
 
  
 \section{Finite element reanalysis of the Adapt with features}
 \label{section_fea_features}
 
The reasons why a FE reanalysis can be considered as meaningful 
were already mentioned in chapter \ref{chapter_fea_reanalysis}
and will therefore not be discussed again. Section
\ref{section_compare_results_with_without_feature} offers
6 different load cases, however, one Adapt with features
will be used for the FE reanalysis. Since the figures
\ref{fig_comp_fea_l79_o} and \ref{fig_comp_fea_l79_m} exhibit
the strongest influence of the implemented features, these
Adapt results will
be used for
the FE reanalysis.\\

 The Adapt is restricted to three features.
These features help to increase the manufacturability
 of the Adapt structures in practice.
 The restrictions can have a positive as 
 well as a negative effect on the compliance.
 In case of the requirement of a maximum tangential
  thickness of the CZGs, a connection larger than
   \textit{CZ\textsubscript{mind }} is usually broken down into several 
   smaller connections, whereby it must meet the
    minimum distance of \textit{AVA}. The reason why 
    one thick connection is broken into several smaller
     ones instead of applying the obtained material 
     elsewhere is that the sensitivity in the vicinity of 
     the connection location may be higher than in another location.
     Furthermore, this requirement does not prohibit 
     the appearance of any structure completely, it is limited in 
     its tangential thickness by  \textit{CZ\textsubscript{mind }} and
      in its normal thickness by \textit{ep}. 
      Thus it is possible that several smaller 
      formed compounds increase stiffness. \\
       
       In the case of the requirement that no unused 
       connection should be created or that the 
       base should not only be thickened, the 
       creation of a local reinforcement is completely
        forbidden. Since these structures, due to 
        their high sensitivities, have required high
         punishment factors to be removed, it is 
         very likely that the obtained material
          cannot be applied in a better place . 
          Therefore, the application of this feature
           is expected to result in a loss of stiffness.\\
                        
       Both figure \ref{fig_comp_fea_rean_alt} and 
       \ref{fig_comp_fea_rean_neu} do not  exhibit any obvious
       differences in the maximum strain (red areas).
        Both also  display no sharp transitions from high to low 
        stress and vice versa. Furthermore, the stress
         curves are in the middle range of the
         stress range, which 
         is  desirable.
          Through the two figures it  can be 
          observed that both AMs, with and without the features,
           are exposed to stress, 
           which leads to the conclusion that they 
           reduce the compliance of the Adapt structure.\\
           
  \begin{figure} [!h]
 \centering
   \includegraphics[width= \textwidth]
   {path_Image/pngs/Aufgabe_3/079_nachrech_alt.png}
 	\caption {FE reanalysis of a Adapt without features.} 
 	\label{fig_comp_fea_rean_alt}
 	 \end{figure}
 
  \begin{figure} [!h]
 \centering
   \includegraphics[width= \textwidth]
   {path_Image/pngs/Aufgabe_3/079_nachrech_neu.png}
 	\caption{FE reanalysis of a Adapt with features.} 
 	\label{fig_comp_fea_rean_neu}
 \end{figure}
 
% _______________________________________





%
%%%5) Bei vorgegebener Basis FE-Analyse

\chapter{Topology optimization with an external 2D and 3D Basis}
\label{chapter_external_basis}

Until now, it was only possible to
 take structures created with
  Matlab as a Basis. Since the Adapt is
  supposed to optimise an entire vehicle, it 
  is desirable
   to include  any given  structure as
   the Basis. For this purpose, it is going to be explained, how to
    import a 2D or 3D structure, generated with a common
     CAD sofwares, e.g.CATIA , into Matlab and use the imported
     structure as the Basis. Afterwards it is going to be clarified,
     why optimising with an external Basis requires
     the already presented features and finally 
     two applications are shown.
     Note that the CAD program must be able to save the CAD drawing as
      a .stl file and the discretization takes place exclusively as
      squared elements in 2D and as voxels in 3D. 
      
      \section{External Basis}
      \begin{figure}[H]
      \begin{minipage}{0.45\textwidth}
            In the previous chapter \ref{chapter_features_adapt} three Adapt
      features were presented. These features enable to get the control
      of the maximal tangential thickness of the CZGs, remove
      locally reinforcements and define a minimal distance between
      the CZGs. The reason why these features were implemented,
      is because, it was desired to use the Adapt for slightly different and 
      new load cases, which can be seen in figures
      \ref{fig_slighlty_different_load} and
      \ref{fig_new_load}. An external Basis as shown in figure
      \ref{fig_ext_bas_1} is obtained without any predefined load-case.
      The Basis, which was used up to now, was always
      generated with a Basis load-case though the topology
      optimisation codes \cite{Andreassen.2011} and \cite{Liu.2014}.
          \end{minipage}
      \hfill
            \begin{minipage}{0.45\textwidth}
      	\centering
  \includegraphics[width= \textwidth]{path_Image/pngs/Aufgabe_3/grbegri_5.png}
	\caption{Random external Basis.} 
	\label{fig_ext_bas_1}
      \end{minipage}
      \end{figure}
      
      \begin{figure}[!h]
      \begin{minipage}{0.45\textwidth}
\centering
  \includegraphics[width= \textwidth]{path_Image/pngs/Aufgabe_3/grbegri_6.png}
	\caption{Load close to loadpaths.} 
	\label{fig_ext_bas_2}
\end{minipage}
\hfill
\begin{minipage}{0.45\textwidth}
	\centering
  \includegraphics[width= \textwidth]{path_Image/pngs/Aufgabe_3/grbegri_7.png}
	\caption{Load far from loadpaths.} 
	\label{fig_ext_bas_3}
\end{minipage}
      \end{figure}
	An external
      Basis is not generated
      though topology optimisation and
      therefore it does not have a load-case and also there is no such
      event, in which the Adapt could have the same load-case and loadpaths as
      an external Basis. However, there is something similar to
      a slightly different load-case, the event is called
     \emph{ load close to the load paths} and can be seen in
      the figure \ref{chapter_external_basis}.
      The 
      previous new load is similar to a load, which is\emph{ far
      away from the load paths} of the external Basis, which
      can be obtained from the figure
      \ref{fig_ext_bas_3}. Since in case of a slightly different load-case
      and without the features,
      the Adapt produces
      results with wide connections and local reinforcements
      and therefore a calculation with an external Basis
      is requiring the features in order to prohibit the
      occurrence of wide connections and local reinforcements.
	The following sections are going to explain
	which method is used
      in order to load an external Basis into Matlab.
      
\section{Discretize external 2D and 3D Basis}
The idea for working with an external Basis is to load any
\emph{.stl}-file in Matlab. This is done by
discretizing the \emph{.stl} 2D or 3D object.
The code for this
purpose is called Mesh voxelisation and 
 is available free of charge on MathWorks File Exchange. Note
 the Mesh voxelisation is not written by the author of this bachelor thesis.
 MathWorks File Exchange also explains  the principles of the code,
 therefore the reader is referred to it.
  In order to use Mesh voxelisation in Adapt, 
  the Matrix-Ouput, which is 
  obtained after running
  Mesh voxelisation,
  must be converted from the type  \textit{Boolean}
   to type \textit{Double}. The
    matrix can then be regarded as a
    completely usual Matlab matrix consisting 
    exclusively of 0 and 1. 
    In order to discretize in 2D, the depth parameter must be set to \textit{1}.

\section{Results with external Basis}
  \begin{figure}[!h]
  \begin{minipage}{0.45\textwidth}
  The selected external 2D and 3D Basis
 have been drawn with CATIA. The 3D Basis
 is extruded in the depth, which is the only difference between
 the 2D and the 3D Basis. The 2D Basis
 is presented in
 \ref{fig_externe_2d_basis}.
  It consists of a frame, which occupies 20\%
   of the available construction space 
  in both 2D and 3D. 
  \end{minipage}
  \hfill
    \begin{minipage}{0.5\textwidth}
    \centering
    \includegraphics[width=  \textwidth]
    {path_Image/pngs/Aufgabe_4/externe2d_basis.png}
  	\caption{Gray external 2D Basis.} 
  	\label{fig_externe_2d_basis}
  \end{minipage}
  \end{figure}
  
  The figures \ref{fig_externe_basis_02} and
  \ref{fig_externe_basis_04} are obtained with the 
  load case in figure \ref{fig_loadcase_canti_2d} and the
  use of the features, where \ref{fig_externe_basis_02}
  show the results with a display threshold of \emph{0.2} and
  the figure \ref{fig_externe_basis_04} show the
  results with a threshold of \emph{0.4}. The Basis is gray and 
  the AMs are coloured with
  the colormap \emph{Blue to Red Rainbow}. Since the features are
  not implemented in 3D,
  the latest 3D Adapt, which prohibits the appearance of
  connections at the edges is employed. The load case is
  similar to Cantilever\textsubscript{3}, with the sole difference,
  that the force is applied on only the central node in the depth
  direction.
  The figures \ref{fig_externe_basis3d_02}
  and \ref{fig_externe_basis3d_02_2} show the results
  with a threshold of \emph{0.2} and
  the figures \ref{fig_externe_basis3d_04_2}
  and \ref{fig_externe_basis3d_04_2}
  show the figure with a threshold of \emph{0.4}.
  
    \begin{figure}[!h]
  \begin{minipage}{0.45\textwidth}
  \centering
    \includegraphics[width= \textwidth]
    {path_Image/pngs/Aufgabe_4/externe2d_basis_th_02.png}
  	\caption{Threshold: 02, gray external 2D Basis.} 
  	\label{fig_externe_basis_02}
  \end{minipage}
  \hfill
  \begin{minipage}{0.45\textwidth}
  \centering
    \includegraphics[width= \textwidth]
        {path_Image/pngs/Aufgabe_4/externe2d_basis_th_04.png}
  	\caption{Threshold: 04, gray external 2D Basis.} 
  	\label{fig_externe_basis_04}
  \end{minipage}\\
  
  \vspace{0.5cm}
  \begin{minipage}{0.45\textwidth}
  \centering
    \includegraphics[width= \textwidth]
    {path_Image/pngs/Aufgabe_4/externe3d_basis_th_02.png}
  	\caption{Threshold: 02, gray external 3D Basis..} 
  	\label{fig_externe_basis3d_02}
  \end{minipage}
  \hfill
  \begin{minipage}{0.45\textwidth}
  \centering
    \includegraphics[width= \textwidth]
        {path_Image/pngs/Aufgabe_4/externe3d_basis_th_04.png}
  	\caption{Threshold: 04, gray external 3D Basis.} 
  	\label{fig_externe_basis3d_04}
  \end{minipage}
    \end{figure}
  
      \begin{figure}[!h]
    \begin{minipage}{0.35\textwidth}
  \centering
    \includegraphics[width= \textwidth]
    {path_Image/pngs/Aufgabe_4/externe3d_basis_th_02_2.png}
  	\caption{Threshold: 02, gray external 3D Basis..} 
  	\label{fig_externe_basis3d_02_2}
  \end{minipage}
  \hfill
  \begin{minipage}{0.35\textwidth}
  \centering
    \includegraphics[width= \textwidth]
        {path_Image/pngs/Aufgabe_4/externe3d_basis_th_04_2.png}
  	\caption{Threshold: 04, gray external 3D Basis..} 
  	\label{fig_externe_basis3d_04_2}
  \end{minipage}
  \end{figure}~\\
%%
\chapter{Summary and outlook}

The aim of this work was
to extend an existing Topology optimization method for
the adaptation of structures in different
ways. First it was extended for the optimization of 3D structures
and the performance of the generated structures was analysed using FEA. Secondly the adaptation
method was extended for slightly different and new load cases.
Also the performance of these results were analysed using FEA.
Finally the optimiser was extend to import an external Basis.
The goals could all be achieved,
a very powerful 2D and 3D post processor was found, which crucial for
the development of the different methods.\\

The Adapt has been improved in many places, but there is still a lot of room for improvement. It is particularly important to be able to count on large models when it comes to optimizing an entire vehicle. Not only with large models, but ideally also with a reasonable computing time. For this purpose a different programming language like C++ is better suited. But just changing the programming language is not enough for the big projects. It will be necessary
to parallelize the Adapt.
In \cite{Aage.2015} already a parallel version of a comparable version of the 88 lines or 169 lines of code is shown. \\

Before parallelizing, it is even more important to extend the features to 3D and after this is done, it would make sense to  think about other features, which
are desired to be implemented.
If all the features are implemented, then
the Adapt should be parallelized for a Cluster. Note Matlab offers a parallelization
toolkit, which is probably easier to use than MPI, BLAS and LAPACK.\\

Furthermore, it is advisable to use an iterative FEM-solver when it comes
to big models, since they are faster. However this only counts
for big models, as long as the model is not much greater than
the used
mesh resolutions in this work, a direct solver is much faster. The
implementation of a 2D fem solver can be found in \cite{Amir.2014}. The
FEM solver is the bottleneck of the computational time. Another
suggestion is to use a different sensitivity filter. The standard
sensitivity filter can be replaced by the conv2 in 2D or convn in 3D
functions from Matlab as shown in \cite{Andreassen.2011}
%
%
\chapter{Appendix}

%______________matlab 2d BZ_____________
\begin{lstlisting}[
style=Matlab-editor,
basicstyle=\mlttfamily,
escapechar=`,
label=lst_finding_BZ,
caption={2D Matlab code to find the Boundary Zone}
]
% Main Step 1 - Find neighbour-elements
BZ = zeros (nely,nelx);  % Boundary-Zone-matrix
for i = 1:nelx
    for j = 1:nely
        if Basis (j,i) > Basis_recognise_threshold              
            if i-ep > 0                     % Previous el in x
                BZ (j,i-ep) = 1;
            end
            if i+ep < nelx                  % Next el in x
                BZ (j,i+ep) = 1;
            end
            if j-ep > 0                     % Previous in y
                BZ (j-ep,i) = 1;      
            end
            if j+ep < nely                  % Next el in y
                BZ (j+ep,i) = 1;
            end
        end
    end
end

% Main Step 2 - Remove Basis entries
for j = 1:nely              %Number of elements in y
    for i = 1:nelx          %Number of elements in x
        if Basis (j,i) > Basis_recognise_threshold
            BZ(j,i) = 0;
        end
    end
end
\end{lstlisting}~\\

\newpage

%______________matlab 3d BZ_____________
\begin{lstlisting}[
style=Matlab-editor,
basicstyle=\mlttfamily,
escapechar=`,
label=lst_finding_BZ_3D,
caption={3D Matlab code to find the Boundary Zone}
]
% Main Step 1 - Find neighbour-elements in 3D
    BZ = zeros (nely,nelx,nelz);                % Boundary-Zone-matrix
    for i = 1:nelx
       for j = 1:nely
           for k = 1:nelz
            if Basis(j,i,k) > Basis_recognise_threshold        
                if i-ep > 0                     % Previous element in x
                    BZ (j,i-ep,k) = 1;
                end
                if i+ep < nelx                  % Next element in x
                    BZ (j,i+ep,k) = 1;
                end
                if j-ep > 0                     % Previous element in y
                    BZ (j-ep,i,k) = 1;      
                end
                if j+ep < nely                  % Next element in y
                    BZ (j+ep,i,k) = 1;
                end
                if k-ep > 0                    % Previous element in z
                    BZ (j,i,k-ep) = 1;
                end
                if k+ep < nelz                 % Next element in z
                    BZ (j,i,k+ep) = 1;
                end 
               end
           end
        end
    end
    
% Main Step 2 - Remove Basis entries
    for j = 1:nely
        for i = 1:nelx
            for k = 1:nelz
             if basis (j,i,k) > Basis_recognise_threshold   
                BZ(j,i,k) = 0;
             end
            end
        end
    end
\end{lstlisting} ~\\


%
%______RE 2D Matlab____________-

\begin{lstlisting}[
style=Matlab-editor,
basicstyle=\mlttfamily,
escapechar=`,
label=lst_2d_RE,
caption={2D Matlab code to find the Remaining Elements (RE)}
]
% As we know: RE = DS -BZ
RE = DE;
% Set overlapping elements to 0
for i = 1:nelx
    for j = 1:nely
        if BE(j,i) == 1
         RE (j,i)= 0; 
        end
    end
end
\end{lstlisting}~\\
%______________matlab 3d RE_____________
\begin{lstlisting}[
style=Matlab-editor,
basicstyle=\mlttfamily,
escapechar=`,
label=lst_3d_RE,
caption={3D Matlab code to find the Remaining Elements (RE)}
]
% As we know: RE = DS -BZ
RE = DE;
% Set overlapping elements to 0
for i = 1:nelx
    for j = 1:nely
%         Taking the third axis into account
        for k = 1:nelz
            if BE(j,i,k) == 1
             RE (j,i,k)= 0; 
            end
        end
    end
end
\end{lstlisting} ~\\



\section{Matlab and 3D displaying}
\label{section_paraview}
Matlab is a very powerful commercial software and for 2D topology results it is equipped with a lot of features. However, for 3D topology optimisation Matlab often meets its limits, because
it has difficulties in rotating a 3D object.
Matlab is capable of rotating and zooming in and out, but
as the number of the elements increases, Matlab needs
up to minutes to perform one rotation. Any 3D object is meant to be rotated, zoomed in and out, instantly, therefore as an alternative Post-processor ParaView is introduced.

\section{ParaView as postprocessor}   
%supported formats
ParaView is an application for interactive and scientific visualization, is available for Windows, Linux and Macos and is mostly used for CFD-solutions (Computer Fluid Dynamics). It is often said, that ParaView wound be the most powerful open-source post-processor. And compared to Matlab this was already the first big advantage which Praview brings along, it is free. ParaView's size is beyond the scope of this bachelor thesis, nevertheless,
some beneficial features regarding 3D topology optimisation are presented.

%You can roate, zoom in, zoom out, overlapp all fluently
\subsection{ParaView General Features }
Features, e.g.  Rotation and Zooming in real time, working with big models, the possibility of a portable installation (especially helpful, if admin-rights are missing) and the numerous of supported File-Readers. However, there are other function, which demands a bit of explanation. Note, all the explanations are based on the Version 5.6.0.

\subsection{Load a .vtk in ParaView}
Run ParaView (inside the bin folder, click on paraview) click on File/Open and choose a .vtk-file, or simply use Drag and Drop into the Pipeline Browser (see figure \ref{fig_para_thres}). Search for the Properties Tab, which is located beneath the Pipeline Browser and click on 'Apply'. Now the default  colormap is set to 'solid' (see the blue framed color-map in figure \ref{fig_Para_solid_color} and  in figure\ref{fig_para_thres} the colormap is set to adapt). Change the solid to your predefined colormap (see subsection \ref{subsection_generate_vtk}) and finally change \emph{Outline} (see the yellow framed view-option in figure \ref{subsection_generate_vtk}) to Surface or Surface with Edges.

\begin{figure}[!h]
\begin{minipage}{0.55 \textwidth}
\subsection{ParaView feature 1:  Make use of the threshold-filter}
\label{subsection_paraview_threshold}
SIMP aims to penalize intermediate densities, in such a manner, that the low densities are penalized stronger than high densities (see subsection  \ref{subsection_SIMP}), in order to improve the convergence behaviour. However, SIMP still cannot prohibit intermediate densities completely. Therefore a tool, which obeys a threshold-restriction.
 This restriction enables it to only plot, e.g. after reaching a trigger point $\rho \geq 0.1$ or  inside an interval  $0.2 \leq \rho \leq 0.7$ (see figures \ref{fig_threshold_05} and \ref{thres_hold_02_07}.) \\

\textit{Instruction:} In order to apply the threshold-filter, load any .vtk-file with Drag and Drop or by File/Open, make sure that the eye-icon in the Pipeline Browser is checked (see figure \ref{fig_para_thres}), click on the red framed (see figure \ref{fig_para_thres}) threshold-icon. Now, beneath the Pipeline Browser in the Properties Tab, the user can choose between the Scalars and a minimum and maximum value and has to confirm by clicking on\emph{ Apply}.
\end{minipage}
\hfill
\begin{minipage}{0.4 \textwidth}
  \includegraphics[scale = 1]
  {path_Image/pngs/Paraview/thres_hold.png}
	\caption{ParaView - blue framed eye-icon and red framed threshold.} 
	\label{fig_para_thres}
\end{minipage}

\end{figure}

\begin{figure}[!h]
\begin{minipage}{0.45\textwidth}
\centering
  \includegraphics[width = \textwidth]
  {path_Image/pngs/Paraview/paraview_thr_01.png}
	\caption{Threshold between 0.2 and 0.7} 
	\label{thres_hold_02_07}
\end{minipage}
\hfill
\begin{minipage}{0.45\textwidth}
\centering
  \includegraphics[width = \textwidth]{path_Image/pngs/Paraview/paraview_thr_02.png}
	\caption{Threshold between 0.1 and 1} 
	\label{fig_threshold_05}
\end{minipage}
\end{figure}~\\


\begin{figure}[!h]
\begin{minipage}{0.55 \textwidth}
\subsection{ParaView feature 2: Colormaps}
Since the Adapt results are composed of intermediate densities a colourful highlighting of different densities can be benefaction,in order to study the  Adapt results precisely or stress specific regions. Therefore ParaView provides a completely adjustable colorbar. \\

\textbf{Instructions:} Assuming having loaded a .vtk file, click on the desired .vtk file, make certain the blue eye-icon is set to visible (see figure \ref{fig_para_thres}) then beneath the Pipeline Browser, click in the in Properties/Coloring at the Edit-button (see figure \ref{fig_Para_edit_color}).
\end{minipage}
\hfill
\begin{minipage}{0.45\textwidth}
 \centering
   \includegraphics[width = 0.85 \textwidth]{path_Image/pngs/Paraview/ParaV_edit_col.png}
 	\caption{ParaView - red framed colorbar edit button.} 
 	\label{fig_Para_edit_color}
\end{minipage}
 \end{figure} 
 
 

\subsection{ParaView feature 3: Progress-Animation-View}
Topology optimisation is a numerically progress, it performs many tasks in order to obtain the first solution and because the first solution has to satisfy a abort-criterion (see flowchart \ref{fig_flowchart_88}), otherwise the progress revise all the steps. Topology optimisation is based on iterations and each iteration exhibits results. For the purpose of understanding the decisions, which were made by the Adapt, a animation can be helpful. This animation is meant to store  all the information about each iteration and show them like animation.\\

\textbf{Instructions:} The user has to generate for each iteration a .vtk-file. This can  be done by implementing the code, presented in Listing \ref{lst_vtk}, in a loop. it is necessary to follow a certain naming pattern  \cite{.29.11.2018b}. After having generated all the needed vtk-files, the series should be moved to a empty folder. In contrast to a single .vtk-file a series of .vtk-files cannot be loaded into ParaView by Drag and Drop, in order to play the the .vtk-animation, the user has to click on File/Open and search the .vtk-series (the .vtk-series is going to be showed as a group). 

\begin{figure}[!h]
 \centering
   \includegraphics[width =  \textwidth]{path_Image/pngs/Paraview/ParaV_play.png}
 	\caption{ParaView - red framed colorbar play button and blue framed  solid-colormap.} 
 	\label{fig_Para_solid_color}
 \end{figure} 
 
% ::::::::::::::::::::::::::____________________________

\subsection{ParaView feature 4: Seeing inner structures,  CT-Scan, Cut and Volume}
In 3D Topology optimisation  the structure is able to be filled with holes and since these holes can be at surface, as well as inside the structure, these holes may not be identified. In the following, three method are presented, in order to get a look at the inner structure.\\

After having loaded a .vtk-file the user can click on the yellow framed
\emph{Outline} arrow (figure \ref{fig_Para_solid_color},
then select \emph{Volume}. The second method is to chose \emph{clip}, press
\textit{SHIFT+CTRL+ENTER} and then type \textit{clip}. This function enables
a cut-view in a desired axes. A similar function is called \emph{slice}, press
\textit{SHIFT+CTRL+ENTER} and then type \textit{slice}. With this function it is possible to get
a 2D insight of a 3D model. In order to have a better control of
the a slice-view, to see
the 3D results like in a CT-scan, open a new layout by clicking
on the plus symbol (at the bottom of the fgure \ref{fig_Para_solid_color}), then
scroll down and click on slice view. 
 
% ::::::::::::______________________________________

\subsection{Matlab implementation: Generate .vtk-files}
\label{subsection_generate_vtk}
After having explained, why ParaView can bee seen as a reasonable post-processor, the reader
 probably would like to know, how to convert a matrix from Matlab to a readable ParaView-file. According to \cite{.29.11.2018}, ParaView has 73 different file-Reader and one of these file-Reader can read .stl-files. However, for the aim to export a Matlab 3D figure into ParaView, the data format .vtk is required.\\

There are a lot of ways to obtain a vtk-writer and in the following, one method is introduced. Every reader, free of charge, can download a vtk-writer from \cite{.}. After having said that, two  modifications in the vtk-writer.m file needs to be done. \\

The first one is at line 105, the 'nx, ny, nz' needs to be replaced with 'nx+1, ny+1, nz+1' and the second adjustment is located at line 11, the expression \emph{POINT}\_DATA must be replaced with \emph{CELL}\_DATA.
With the command in the Listing \ref{lst_vtk} the vtkwriter.m-file will generate a .vtk-file. The first input parameter is a string, it must end with a .vtk-file-extension, the second parameter needs to be set as shown in  the Listing \ref{lst_vtk}, the third parameter  defines the name of the ParaView scalars and also the name of the colormap. It is recommended to name is like the matrix and the last parameter is the matrix from Matlab, which is desired to be plotted in ParaView. The described procedure is valid for 2D and 3D matrices.

\begin{lstlisting}[
style=Matlab-editor,
basicstyle=\mlttfamily,
escapechar=`,
label=lst_vtk,
caption={Code to run the vtkwriter.m-file, for 2D and 3D.}
]
vtkwrite_basis('save_name_DE.vtk','structured_points','DE',DE_matrix);
\end{lstlisting}
 
 ParaView is able to store multiple set of scalars respectively Matlab matrices, in one .vtk-file. This comes handy, when it is necessary to have the Basis and the Adapt in one file.



\bibliography{2_Latex_Files/bib_tex}
%
%%____STATUTORY_DECLARATION_________
%%Eidestattliche Erklärung 

% STATUTORY_DECLARATION

\chapter*{STATUTORY DECLARATION}

I Javed Butt declare that I have authored this thesis independently, that I have not used other than the declared
sources / resources, and that I have explicitly marked all material which has been quoted either
literally or by content from the used sources\\
\\
Munich, \today\\
\\
\begin{center}
(signature)
\\
Javed Butt
\end{center}



\end{document}

